\documentclass[10pt,a4paper,american]{exam}
\usepackage{../misc/macros/classhandout}

\title{Applied Cryptography - Project E: Post-Quantum Cryptography Migration}
\author{Nadim Kobeissi}
\subject{Lab project exploring practical migration strategies from classical to post-quantum cryptography using NIST standardized algorithms ML-KEM and ML-DSA.}
\keywords{post-quantum cryptography, quantum-resistant algorithms, ML-KEM, Kyber, ML-DSA, Dilithium, hybrid cryptography, cryptographic migration, NIST PQC standards}

\begin{document}
\classhandoutheader
\section*{Project E: Post-Quantum Cryptography Migration}

\subsection*{Overview}
In this project, you will explore the practical challenges of migrating existing systems to post-quantum cryptography. You'll work with NIST's standardized post-quantum algorithms, integrating ML-KEM (Kyber) for key encapsulation and ML-DSA (Dilithium) for digital signatures. This project guides you through hybrid approaches that combine classical and post-quantum algorithms, ensuring security against both current and future quantum threats. You will start by choosing a target protocol that you wish to devise a post-quantum migration strategy for. Then, you'll analyze performance impacts, message size increases, and integration challenges when replacing pre-quantum primitives (such as RSA or ECDH) with quantum-resistant alternatives. By completing this lab, you'll gain practical experience in implementing post-quantum designs, developing migration strategies for existing systems, and evaluating the trade-offs between different post-quantum algorithm choices for various use cases.

\subsection*{Learning Objectives}
After completing this project, you should be able to:
\begin{itemize}
	\item Integrate NIST standardized post-quantum algorithms (ML-KEM and ML-DSA) through the use of existing implementations and cryptographic libraries.
	\item Design hybrid cryptographic schemes that provide security against both classical and quantum adversaries.
	\item Analyze performance and compatibility trade-offs in post-quantum migration.
	\item Develop practical migration strategies for transitioning existing systems to post-quantum security.
\end{itemize}

\subsection*{Background}
The advent of quantum computers poses a significant threat to current public-key cryptography. Post-quantum cryptography provides security against both classical and quantum adversaries:
\begin{itemize}
	\item ML-KEM (Kyber) provides quantum-resistant key encapsulation based on lattice problems.
	\item ML-DSA (Dilithium) offers quantum-resistant digital signatures also based on lattice cryptography.
	\item Hybrid approaches, such as X-Wing\footnote{Manuel Barbosa, Deirdre Connolly, Jo\~ao Diogo Duarte, Aaron Kaiser, Peter Schwabe, Karolin Varner and Baas Westerban, \href{https://appliedcryptography.page/paper/\#xwing-hybrid}{\textit{X-Wing: The Hybrid KEM You've Been Looking For}}, IACR Communications in Cryptology, 2024.} combine classical and post-quantum algorithms to hedge against implementation vulnerabilities.
	\item Migration strategies must consider performance impacts, message size increases, and backward compatibility.
\end{itemize}

\subsection*{Requirements}
Your post-quantum migration project must implement the following core functionality:

\begin{enumerate}
	\item \textbf{Protocol Selection and Analysis:}
	      \begin{itemize}
		      \item Choose a target protocol (e.g., TLS, SSH, VPN, or custom protocol) for migration.
		      \item Document the current cryptographic primitives used in the protocol.
		      \item Analyze the protocol's security requirements and constraints.
	      \end{itemize}

	\item \textbf{Post-Quantum Integration:}
	      \begin{itemize}
		      \item Integrate ML-KEM (Kyber) for key encapsulation mechanisms.
		      \item Integrate ML-DSA (Dilithium) for digital signature operations.
		      \item Create hybrid schemes combining classical and post-quantum algorithms.
	      \end{itemize}

	\item \textbf{Migration Strategy:}
	      \begin{itemize}
		      \item Design a backward-compatible migration path.
		      \item Implement protocol negotiation for algorithm selection.
		      \item Ensure graceful fallback to classical algorithms when necessary.
	      \end{itemize}

	\item \textbf{Performance Analysis:}
	      \begin{itemize}
		      \item Measure and compare key generation, encapsulation, and signature times.
		      \item Analyze message size increases and bandwidth impacts.
		      \item Evaluate memory usage and computational requirements.
		      \item Document performance trade-offs for different security levels.
	      \end{itemize}
\end{enumerate}

\subsection*{Implementation Guidelines}

\subsubsection*{Step 1: Protocol Analysis}
Begin by thoroughly analyzing your chosen protocol:
\begin{itemize}
	\item What cryptographic primitives does it currently use? (RSA, ECDH, ECDSA, etc.)
	\item What are the performance requirements and constraints?
	\item What are the message size limitations?
	\item How does the protocol handle algorithm negotiation?
\end{itemize}

Document your findings and justify your choice of target protocol.

\subsubsection*{Step 2: Post-Quantum Integration}
Implement the post-quantum algorithms:
\begin{itemize}
	\item Integrate ML-KEM (Kyber) for key exchange operations.
	\item Integrate ML-DSA (Dilithium) for signature operations.
	\item Design and implement hybrid modes that combine classical and post-quantum algorithms.
\end{itemize}

\subsubsection*{Step 3: Migration Design}
Develop a comprehensive migration strategy:
\begin{itemize}
	\item Protocol negotiation mechanisms for algorithm selection.
	\item Backward compatibility with non-post-quantum implementations.
	\item Phased migration approach with risk mitigation.
\end{itemize}

\subsubsection*{Step 4: Performance Evaluation}
Conduct thorough performance analysis:
\begin{itemize}
	\item Benchmark all cryptographic operations.
	\item Measure protocol overhead and latency impacts.
	\item Analyze scalability under various load conditions.
\end{itemize}

\subsubsection*{Step 5: Security Analysis}
Evaluate the security of your migration:
\begin{itemize}
	\item Analyze the security levels provided by different parameter sets.
	\item Consider cryptographic agility and future algorithm updates.
	\item Evaluate risks during the migration period.
\end{itemize}

\subsection*{Deliverables}
Submit the following:

\begin{enumerate}
	\item Source code for your post-quantum migration implementation.
	\item Design document including:
	      \begin{itemize}
		      \item Detailed protocol analysis and migration strategy.
		      \item Description of hybrid cryptographic schemes implemented.
		      \item Justification for algorithm choices and parameter selections.
	      \end{itemize}
	\item Performance analysis report with:
	      \begin{itemize}
		      \item Benchmarking results comparing pre- and post-quantum implementations.
		      \item Analysis of message size and bandwidth impacts.
		      \item Recommendations for different use cases.
	      \end{itemize}
	\item Security analysis discussing:
	      \begin{itemize}
		      \item Security guarantees of the migrated protocol.
		      \item Potential vulnerabilities during migration.
		      \item Long-term cryptographic agility considerations.
	      \end{itemize}
\end{enumerate}

\subsection*{Evaluation Criteria}
Your project will be evaluated based on:

\begin{itemize}
	\item Correctness of post-quantum algorithm integration.
	\item Quality and practicality of the migration strategy.
	\item Thoroughness of performance analysis.
	\item Security considerations and risk mitigation.
	\item Quality of code and documentation.
\end{itemize}

\subsection*{Resources}
\begin{itemize}
	\item \href{https://csrc.nist.gov/projects/post-quantum-cryptography/post-quantum-cryptography-standardization}{NIST Post-Quantum Cryptography standardization documents} and \href{https://nvlpubs.nist.gov/nistpubs/FIPS/NIST.FIPS.203.pdf}{FIPS 203 (ML-KEM)}, \href{https://nvlpubs.nist.gov/nistpubs/FIPS/NIST.FIPS.204.pdf}{FIPS 204 (ML-DSA)}.
	\item Reference implementations: \href{https://github.com/pq-crystals/kyber}{Kyber reference implementation}, \href{https://github.com/symbolicsoft/kyber-k2so}{Kyber Go implementation}, \href{https://github.com/pq-crystals/dilithium}{Dilithium reference implementation}, and \href{https://github.com/open-quantum-safe/liboqs}{Open Quantum Safe library}.
	\item Course materials on post-quantum cryptography.
\end{itemize}

\subsection*{Submission Guidelines}
\begin{itemize}
	\item Submit your code as a ZIP archive or through a Git repository.
	\item Include all documentation in PDF or Markdown format.
	\item Presentations: Prepare a 10-minute presentation demonstrating your post-quantum migration strategy and implementation.
\end{itemize}

\end{document}
