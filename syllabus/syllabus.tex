\documentclass[10pt,a4paper,american]{exam}
\usepackage{../misc/macros/classhandout}

\title{Applied Cryptography - Course Syllabus}
\author{Nadim Kobeissi}
\subject{Syllabus for an Applied Cryptography course covering theoretical foundations and practical implementations of modern cryptographic systems.}
\keywords{cryptography, security, protocols, encryption, zero-knowledge proofs, post-quantum cryptography, secure systems, high-assurance cryptography, formal methods, secure messaging, provable security}

\begin{document}
\classhandoutheader
\urldef{\urlcodeofconduct}\url{https://www.aub.edu.lb/SAO/Documents/student%20code%20of%20conduct.pdf}
\section*{Course Syllabus}

\textit{Applied Cryptography} explores the core theory of modern cryptography and how to apply these fundamental principles to build and analyze real-world secure systems. We start with foundational concepts—such as Kerckhoff's Principle, computational hardness, and provable security—before moving on to key cryptographic primitives like pseudorandom generators, block ciphers, and hash functions. Building on this solid groundwork, we will survey how these technologies power critical real-world deployments such as TLS, secure messaging protocols (e.g., Signal), and post-quantum cryptography. We will also delve into specialized topics like high-assurance cryptographic implementations, elliptic-curve-based systems, and zero-knowledge proofs to give you a complete understanding of contemporary cryptography's scope and impact. By the end of the semester, you will have gained both a rigorous theoretical perspective and practical hands-on experience, enabling you to evaluate, design, and implement cryptographic solutions.

\section{Course Objectives \& Outcomes}
This course is designed to bridge the theoretical foundations of cryptography with its practical applications in contemporary secure systems. By engaging with lectures, projects, problem sets, and project work, you will develop a thorough understanding of modern cryptographic concepts and gain the hands-on skills needed to implement, assess, and communicate security solutions.

Upon successful completion of this course, a student should be able to:

\begin{itemize}
	\item Understand the reasoning behind the mathematical underpinnings of modern cryptography.
	\item Analyze and prove the security properties of cryptographic constructions.
	\item Understand how cryptographic constructions can be composed to build secure protocols and systems.
	\item Discern between how cryptography is approached mathematically versus from an engineering perspective.
	\item Critically assess security implementations and evaluate real-world cryptographic protocols.
	\item Gain an understanding about the future of cryptography and its role in emerging technologies.
\end{itemize}

\section{Course Prerequisites}
This course is intended for \textbf{senior undergraduate} students. \textbf{Graduate students} are also welcome to register provided that they are working on a research topic that is relevant to this course. The following prerequisites are \textbf{optional but recommended}:

\begin{itemize}
	\item \textbf{CMPS 215:} Theory of Computation
\end{itemize}

If you want to understand whether you have the sufficient background for this course, review this revision chapter and try to do all the exercises: \url{https://joyofcryptography.com/pdf/chap0.pdf}

\section{Resources}
\begin{itemize}
	\item Mike Rosulek, {\href{https://joyofcryptography.com}{\textit{The Joy of Cryptography}}}, Oregon State University, 2021.\footnote{\textit{The Joy of Cryptography} is available free of charge at \url{https://joyofcryptography.com}.}
	\item Jean-Philippe Aumasson, {\href{https://nostarch.com/serious-cryptography-2nd-edition}{\textit{Serious Cryptography, $2^{nd}$ Edition}}}, No Starch Press, 2024.
	\item Handouts will be made available during the course and on the course website.
\end{itemize}

\section{Course Schedule}
The course schedule is available on the course website, where it is always kept up-to-date, including details about the lecture topics, materials, resources and easy access to slides: \url{https://appliedcryptography.page}

\section{Assessment Items \& Grading Criteria}
In this course, your performance will be evaluated through multiple components designed to measure both your theoretical understanding and your practical skills in cryptography. By staying current with the readings, attending and participating in lectures and projects, and completing all assigned work, you will gain a thorough mastery of the material.

Overall, these graded components are designed to ensure that you not only grasp the theoretical underpinnings of cryptography but also develop the practical expertise needed to implement, analyze, and innovate within the field.

\subsection{This Is Your Classroom}
An essential facet in this course's design is encouraging students to produce work that is entirely theirs, without resorting to AI technologies. This ensures that students are embracing their full learning potential and makes grading more fair throughout the classroom. As such, the class sessions and projects will favor \textbf{engagement}:

\begin{itemize}
	\item \textbf{Interactive lectures:} Classes will incorporate interactive elements such as in-class exercises, discussions, and collaborative problem-solving to encourage active participation.
	\item \textbf{Real-time feedback:} Students will regularly have opportunities to demonstrate their understanding through low-stakes activities, receiving immediate feedback to guide their learning.
	\item \textbf{Peer teaching:} Students will occasionally be invited to explain concepts to their peers, reinforcing their own understanding while creating a collaborative learning environment.
	\item \textbf{Project progress discussions:} Students will be encouraged to openly discuss their progress on projects as they work, allowing for real-time problem-solving and knowledge sharing.
	\item \textbf{Security strategy workshops:} Throughout projects, students will present their approaches to implementing security goals, receiving feedback from peers and instructors to refine their solutions.
	\item \textbf{Collaborative troubleshooting:} Labs will incorporate structured time for students to collectively address challenges, fostering a community where security insights and implementation techniques are freely exchanged.
\end{itemize}

\subsection{Problem Sets}
Problem sets will be assigned periodically throughout the semester to reinforce and deepen your understanding of the lecture material. Each set will include a range of exercises—some focused on theoretical proofs and problem-solving, others requiring short coding tasks or computational experiments. These assignments are designed to bridge the gap between abstract cryptographic concepts and their concrete applications. You are encouraged to start working on each problem set early and to seek guidance during office hours or projects if you encounter difficulties.

\subsection{Projects}
Take-home projects are intended to serve as a hands-on complement to the lectures. Each group of (one to three) students is expected to pick one (or, if you're feeling adventurous, at most two) project topics and work on them throughout the entire course. During each project, you will experiment with real-world libraries, and even simulate attacks or vulnerabilities to understand why certain security practices are necessary. These sessions will also help you become comfortable with relevant tools and environments, including formal analysis tools. Take-home projects are mandatory.

Example projects include:

\begin{itemize}
	\item \textbf{Designing a Password Manager:} Design and implement a secure password manager application, learning about secure password storage techniques, key derivation functions, and encryption methods for sensitive data. You'll implement features such as master password protection, secure password generation, and encrypted storage while analyzing potential vulnerabilities and implementing countermeasures against common attacks.
	\item \textbf{Designing a Secure Messenger:} Build a secure messaging application implementing end-to-end encryption using cryptographic libraries for key exchange protocols, message encryption, and authentication mechanisms. This project covers essential features like perfect forward secrecy, deniability, and secure group messaging while exploring practical challenges such as key verification and metadata protection.
	\item \textbf{Protocol Modeling and Verification with ProVerif:} Design and formally verify a Transport Layer Security (TLS)-like protocol using ProVerif, a formal verification tool for cryptographic protocols. This project provides practical experience with cryptographic protocol design, formal verification, and security property specification for protocols that provide confidentiality, integrity, and authentication for network communications.
	\item \textbf{Designing a Battleship Game Using Zero-Knowledge Systems:} Design and implement a zero-knowledge battleship game using RISC Zero, a zero-knowledge virtual machine (zkVM). This project demonstrates how players can validate moves without revealing the entire game state, providing practical experience with zero-knowledge proofs, zkVMs, and cryptographic protocol design.
	\item \textbf{Post-Quantum Cryptography Migration:} Explore the practical challenges of migrating existing systems to post-quantum cryptography using NIST's standardized algorithms, integrating ML-KEM (Kyber) for key encapsulation and ML-DSA (Dilithium) for digital signatures. You'll analyze performance impacts, message size increases, and integration challenges when replacing pre-quantum primitives with quantum-resistant alternatives.
	\item \textbf{Private Set Intersection for Contact Discovery:} Build a privacy-preserving contact discovery system using Private Set Intersection (PSI), implementing an Oblivious Pseudorandom Function (OPRF) based approach. This project covers designing secure wire protocols, implementing server and client components, and addressing security challenges like preventing offline enumeration attacks through peppered hashing and rate limiting.
	\item \textbf{Privacy-Preserving Age Verification:} Build a privacy-preserving age verification system that proves age thresholds without revealing personal information using cryptographic commitments and signatures. You'll implement zero-knowledge proofs that verify statements like "I am 18 or older" while tackling challenges including credential revocation, preventing credential sharing, and ensuring efficient mobile verification.
	\item \textbf{Building a Time-Locked Message Capsule:} Build a practical timelock encryption system that enables messages to be encrypted such that they can only be decrypted after a specific future time. You'll implement the tlock protocol using the League of Entropy's distributed randomness beacon, exploring real-world applications including scheduled disclosures, commitment schemes for auctions, and "digital time capsules" for personal messages.
\end{itemize}

\subsection{Exams}
There will be a midterm exam and a final exam. The exams will test your command of topics discussed throughout the semester. You are expected to come to each exam prepared, having thoroughly reviewed lecture notes.

\subsection{Grading Breakdown}
The final course grade will be computed using the following breakdown:

\begin{center}
	\renewcommand{\arraystretch}{2}
	\begin{tabular}{|p{2.5in}|c|}
		\hline
		\textbf{Category}           & \textbf{Percentage} \\
		\hline
		Attendance \& Participation & $10\%$              \\
		\hline
		Problem Sets                & $22.5\%$            \\
		\hline
		Take-Home Project           & $22.5\%$            \\
		\hline
		Midterm Exam                & $22.5\%$            \\
		\hline
		Final Exam                  & $22.5\%$            \\
		\hline
	\end{tabular}
\end{center}

\section{The Key Exchange}
\textit{The Key Exchange} is an optional weekly gathering intended to provide additional career training for students intending to become researchers or career professionals in cryptography. Once a week, we will come together to discuss cutting-edge research papers, work on our presentation skills by practicing presenting complex ideas, develop scientific writing skills, and explore career paths in cryptography. Whether you're debugging your first encryption algorithm or designing novel protocols, \textit{The Key Exchange} provides a supportive environment where questions are encouraged, mistakes become learning opportunities, and everyone has something valuable to contribute. The only prerequisite is your curiosity!

What makes \textit{The Key Exchange} special isn't just what we learn, but how we learn together. In our relaxed, coffee-shop-style sessions, you'll gain skills that textbooks can't teach: how to read research papers efficiently, present technical concepts clearly, write like a cryptographer, and navigate career decisions with confidence. One week you might be discussing the elegance of elliptic curves with a visiting researcher from Signal, the next you're getting feedback on your presentation about side-channel attacks, and the following week you're collaborating on a blog post explaining zero-knowledge proofs to a general audience. With our rotating format of paper discussions, student presentations, writing workshops, and career cafés, you'll build a portfolio of skills while forming lasting connections with peers who share your passion for cryptography. After all, the best cryptographic protocols are built on collaboration, and \textit{The Key Exchange} is where your journey as a cryptographer truly begins!

\subsection{Four-Week Rotating Schedule}

\subsubsection*{Week 1: Paper Deep Dive}
\textbf{Theme:} \textit{Reading and Understanding Research}

Join us for collaborative discussions of seminal and cutting-edge cryptography papers. Learn efficient paper reading strategies (abstract $\rightarrow$ contributions $\rightarrow$ methodology $\rightarrow$ conclusion), practice identifying key contributions and limitations, and engage in collaborative annotation. We'll sometimes explore classics but more often react to the most recent papers from CRYPTO/Eurocrypt/CCS, and accessible papers with clear practical applications.

\textbf{Skills developed:} Critical analysis, technical reading comprehension, identifying research gaps

\subsubsection*{Week 2: Student Presentations}
\textbf{Theme:} \textit{Communicating Complex Ideas}

Practice your presentation skills with 15-minute talks on cryptographic topics, followed by constructive peer feedback. Topics include explaining cryptographic constructions, analyzing recent attacks or vulnerabilities, presenting project progress, or proposing research ideas. We'll also have ``lightning talks'' for 3-minute research pitches and practice with visual aids and demos.

\textbf{Skills developed:} Public speaking, slide design, handling Q\&A, time management

\subsubsection*{Week 3: Writing Workshop}
\textbf{Theme:} \textit{Scientific Writing and Documentation}

Develop your technical writing through collaborative exercises and peer review sessions. Projects include abstract writing for imaginary papers, blog posts explaining crypto concepts, security audit reports, and research proposals. We'll also cover LaTeX tutorials for papers and security disclosure writing practices.

\textbf{Skills developed:} Technical writing clarity, academic writing style, documentation best practices, peer review skills

\subsubsection*{Week 4: Career Café}
\textbf{Theme:} \textit{Professional Development and Pathways}

Explore cryptography career paths with industry guest speakers (virtual or in-person), graduate school application workshops, interview preparation sessions, and portfolio/CV reviews. Topics include career paths in cryptography, industry vs. academia decisions, required skills for different roles, and networking strategies.

\textbf{Skills developed:} Professional networking, interview skills, career planning, industry awareness

\subsection{Special Sessions}

\subsubsection*{Crypto Conference Simulation (Once per semester)}
Experience a full academic conference simulation where students present ``papers'' complete with program committee, reviews, and awards. Professional attire encouraged, with a keynote by faculty or guest speaker. This immersive experience provides insight into the academic conference process and helps build presentation confidence.

\section{Course Policies}
Students are expected to strictly observe the following course policies:

\subsection{Attendance}
\textbf{Do not register for this course if you do not plan to attend all classes and exams.} You are expected to abide by the university's rules on attendance. You are expected to attend lectures and to be on time for all sessions and activities related to this course. Lectures are a sequence. Missing one lecture will almost certainly mean that you will not be able to keep up with the following lectures without studying the material covered in the missed lecture. Catching up with missed lectures is your responsibility and is done on your own time. You are responsible for all work, even when absent. Attendance may be recorded at every class session. Excessive absence, defined at the discretion of the instructor, will not be tolerated and will result in being dropped from the course.

\subsection{Academic Misconduct \& Plagiarism}
Lectures start on time. You may not be allowed to come into the room once class has started. Any class conduct that disturbs the learning atmosphere may be deemed misbehavior and will not be tolerated.

This course has a strict \textbf{zero tolerance policy for cheating}. Any instance of cheating will result in an immediate, non-negotiable grade of 0 on the pertinent assignment and a report to the university faculty:
\begin{itemize}
	\item Your work has to be your own. No copying work (or rewriting it line by line based on someone else's work) will be tolerated.
	\item Any sharing of any answers on any assignment is considered cheating.
	\item Coaching another student by helping them writing their answers line by line is also cheating.
	\item Copying answers or code from the Internet or hiring someone to write your answers for you is cheating.
\end{itemize}

Explaining how to use systems or tools and helping others with high-level design issues is not cheating.

\textbf{Regarding AI Tools:} Any use of AI tools to produce answers to class assignments or projects is considered cheating.
\begin{itemize}
	\item Using AI tools like ChatGPT, GitHub Copilot, or similar to generate code, proofs, or written answers constitutes cheating.
	\item Submitting AI-generated work without proper attribution and explanation is considered plagiarism.
	\item The course will employ AI-detection tools and manual review techniques to identify AI-generated submissions.
	\item Using AI to enhance your learning experience (e.g. asking ChatGPT questions about the material) is not considered cheating.
\end{itemize}

The Student Code of Conduct\footnote{\urlcodeofconduct} acts as the main reference in determining instances of misconduct.

\subsection{Communication Policy}
You are requested to check your e-mail and the course website regularly. You are responsible for all the information communicated to you via these tools. \textbf{Bookmark the course website and visit it regularly. All course news will be kept up to date on the website.}

\section{Note for Special Needs Students}
AUB strives to make learning experiences as accessible as possible. If you anticipate or experience academic barriers due to a disability (including mental health, chronic or temporary medical conditions), please inform the course instructor immediately so that we can privately discuss options. In order to help establish reasonable accommodations and facilitate a smooth accommodations process, you are encouraged to contact the Accessible Education Office: \href{mailto:accessibility@aub.edu.lb}{accessibility@aub.edu.lb}; +961-1-350000, x3246; West Hall, 314.

\section{Nondiscrimination}
AUB is committed to facilitating a campus free of all forms of discrimination including sex/gender-based harassment prohibited by Title IX. The University's non-discrimination policy applies to, and protects, all students, faculty, and staff. If you think you have experienced discrimination or harassment, including sexual misconduct, we encourage you to tell someone promptly. If you speak to a faculty or staff member about an issue such as harassment, sexual violence, or discrimination, the information will be kept as private as possible, however, faculty and designated staff are required to bring it to the attention of the University's Title IX Coordinator. Faculty can refer you to fully confidential resources, and you can find information and contacts at \url{https://www.aub.edu.lb/titleix}. To report an incident, contact the University's Title IX Coordinator Trudi Hodges at 01-350000 ext. 2514, or
\href{mailto:titleix@aub.edu.lb}{titleix@aub.edu.lb}. An anonymous report may be submitted online via EthicsPoint at \url{https://www.aub.ethicspoint.com}.

\end{document}
