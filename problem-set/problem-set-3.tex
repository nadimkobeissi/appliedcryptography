\documentclass[10pt,a4paper,american]{exam}
\usepackage{../misc/macros/classhandout}

\title{Applied Cryptography - Problem Set 3: Asymmetric Cryptography}
\author{Nadim Kobeissi}
\subject{Problem set covering asymmetric cryptography concepts including cryptographic hardness, Diffie-Hellman protocol, elliptic curve cryptography, and practical security applications.}
\keywords{asymmetric cryptography, public key cryptography, Diffie-Hellman, elliptic curves, discrete logarithm problem, cryptographic hardness, man-in-the-middle attacks, ECDSA, Ed25519, post-quantum cryptography}

\begin{document}
\classhandoutheader
\section*{Problem Set 3: Asymmetric Cryptography}

\begin{tcolorbox}[colframe=OliveGreen!30!white,colback=OliveGreen!5!white]
	\textbf{Instructions:} This problem set covers topics in provable security from topics 1.7\footnote{\url{https://appliedcryptography.page/slides/\#1-7}} and 1.8\footnote{\url{https://appliedcryptography.page/slides/\#1-8}} of the course. Submit your solutions as a neatly formatted PDF. You are encouraged to collaborate with classmates in studying the material, but your submitted solutions must be your own work. For proofs, clearly state your assumptions, steps, and conclusions.
\end{tcolorbox}

\section{Cryptographic Hardness and Real-World Implications}
\begin{questions}
	\question[10] \textbf{Breaking Cryptography: Attack Scenarios}

	\begin{parts}
		\part[5] \textbf{The Cryptographic Apocalypse Scenario:}
		Imagine you wake up tomorrow to headlines: \textit{``Breakthrough Algorithm Solves P vs NP - Computer Scientists Prove P = NP!''}
		\begin{subparts}
			\subpart As the Chief Security Officer of a major bank, write a crisis response memo outlining which systems fail immediately and what emergency measures you would implement.
			\subpart Design an alternative security model for online banking that could work in a post-P=NP world. What assumptions would you rely on instead?
		\end{subparts}

		\part[5] \textbf{The Weak DH Parameters Problem:}
		A security researcher discovers that a popular cryptographic library has been generating Diffie-Hellman parameters where the prime $p$ satisfies $p-1$ having many small factors, making 75\% of generated groups vulnerable to Pohlig-Hellman attacks.
		\begin{subparts}
			\subpart Evaluate whether this discovery completely breaks Diffie-Hellman or only partially weakens it. Consider both the mathematical impact and practical deployment consequences.
			\subpart Design a strategy for systems using this library: should they immediately regenerate all parameters, implement parameter validation, or pursue a different approach?
		\end{subparts}
	\end{parts}

	\question[10] \textbf{Discrete Logarithm Security Architecture}

	\begin{parts}
		\part[5] \textbf{The Weak Parameter Disaster:}
		Your security audit discovers that a legacy system has been using $p = 2047$ (which factors as $23 \times 89$) for Diffie-Hellman key exchange, and the generator $g = 2$.
		\begin{subparts}
			\subpart Analyze exactly why this parameter choice is catastrophically weak. Estimate how long it would take an attacker with a modern laptop to break this system.
			\subpart Design an emergency response plan: how do you migrate users to secure parameters while maintaining service availability?
		\end{subparts}

		\part[5] \textbf{Elliptic Curve vs. Finite Field Trade-off Analysis:}
		You're designing a cryptographic protocol for IoT devices with severe computational and bandwidth constraints.
		\begin{subparts}
			\subpart Compare elliptic curve and finite field DLP for your use case: which offers better security per bit of key size, and which offers better computational performance?
			\subpart Analyze why index calculus attacks work against finite fields but not elliptic curves. How does this fundamental difference affect your security margins?
		\end{subparts}
	\end{parts}
\end{questions}

\section{Diffie-Hellman in Hostile Environments}
\begin{questions}
	\question[10] \textbf{Attack and Defense Scenarios}

	\begin{parts}
		\part[5] \textbf{The Perfect Man-in-the-Middle:}
		An attacker has complete control over the network between Alice and Bob, can modify any message, and can initiate connections that appear to come from either party.
		\begin{subparts}
			\subpart Design the most effective man-in-the-middle attack against unauthenticated Diffie-Hellman. Your attack should be undetectable to Alice and Bob during the key exchange.
			\subpart Alice and Bob have never met but each has the other's public key fingerprint written on a piece of paper. Design an authentication protocol that defeats your attack using only these fingerprints.
			\subpart Compare your fingerprint-based solution to certificate authorities and web-of-trust models. What are the usability and security trade-offs?
		\end{subparts}

		\part[5] \textbf{The Paranoid Whistleblower Scenario:}
		A whistleblower needs to securely communicate with a journalist. They assume the government monitors all internet traffic, has compromised most Certificate Authorities, and can perform man-in-the-middle attacks on any connection.
		\begin{subparts}
			\subpart Design a key exchange protocol for this scenario using only methods available to ordinary civilians (no specialized hardware or pre-shared secrets).
			\subpart Analyze what happens if the government can also compromise one of their devices after the key exchange. How can you provide forward secrecy?
		\end{subparts}
	\end{parts}

	\question[10] \textbf{Protocol Design Challenge}

	\textbf{SSH Trust-on-First-Use Analysis:}
	Your organization wants to deploy SSH across 10,000 servers, but the current TOFU model creates security and usability problems at scale.
	\begin{parts}
		\part[5] Analyze specific attack scenarios where the TOFU model fails in practice. When are users most vulnerable?
		\part[5] Design an improved authentication model that maintains SSH's simplicity while providing better security guarantees than pure TOFU.
	\end{parts}
\end{questions}

\section{Elliptic Curve Security Engineering}
\begin{questions}
	\question[15] \textbf{Curve Selection Under Pressure}

	\begin{parts}
		\part[8] \textbf{The Government Backdoor Controversy:}
		You're the security architect for a new messaging app. Cryptographers are debating whether NIST P-256 contains a government backdoor, while Curve25519 offers better security properties but less widespread hardware support.
		\begin{subparts}
			\subpart Analyze the specific concerns about NIST curves: what would a backdoor look like, and how could it be exploited without breaking the underlying mathematical problems?
			\subpart Design a risk assessment framework for choosing between P-256 and Curve25519. What factors should influence your decision?
			\subpart Your legal team reports that several countries require NIST-compliant cryptography for government sales. Propose a solution that addresses both the backdoor concerns and the compliance requirements.
		\end{subparts}

		\part[7] \textbf{The Invalid Curve Attack Scenario:}
		A security researcher discovers that your ECDH implementation doesn't validate input points, making it vulnerable to invalid curve attacks.
		\begin{subparts}
			\subpart Design a specific attack exploiting this vulnerability. What information can an attacker extract?
			\subpart Develop a comprehensive input validation strategy that prevents this attack class. What performance impact does your solution have?
		\end{subparts}
	\end{parts}

	\question[15] \textbf{Implementation Vulnerability Analysis}

	\begin{parts}
		\part[8] \textbf{Side-Channel Attack Laboratory:}
		You're tasked with testing an embedded device's ECDSA implementation for side-channel vulnerabilities.
		\begin{subparts}
			\subpart Design a timing attack against variable-time scalar multiplication. What information would you measure, and how would you extract the private key?
			\subpart Develop countermeasures that maintain performance while resisting your attack. What constant-time techniques would you implement?
		\end{subparts}

		\part[7] \textbf{The Ed25519 Validation Crisis:}
		You discover that two widely-used Ed25519 libraries accept different signatures as valid for the same message and public key, breaking interoperability.
		\begin{subparts}
			\subpart Investigate what causes this inconsistency: what validation steps do different implementations handle differently?
			\subpart Analyze the security implications: could an attacker exploit these differences to create practical attacks?
		\end{subparts}
	\end{parts}
\end{questions}

\section{Applied Cryptography Case Studies}
\begin{questions}
	\question[15] \textbf{Cryptocurrency Signature Scheme Analysis:}
	A new cryptocurrency project is choosing between ECDSA and Ed25519 for transaction signatures. The system requirements include:
	\begin{itemize}
		\item High transaction throughput (thousands of signatures per second)
		\item Long-term security (system should remain secure for decades)
		\item Compatibility with hardware wallets and mobile devices
		\item Deterministic transaction signing for reproducibility
	\end{itemize}
	Analyze this decision:
	\begin{parts}
		\part[5] Compare ECDSA and Ed25519 for each requirement above. Which algorithm better meets each criterion and why?
		\part[5] Discuss the implications of signature malleability. How does this affect each algorithm and why might it matter for cryptocurrency applications?
		\part[5] Make a final recommendation with justification, considering both technical and practical factors.
	\end{parts}

	\question[15] \textbf{Secure Communication System Architecture:}
	You are architecting a secure communication system for a large organization (10,000+ employees) that needs to protect against both external attackers and potential insider threats. The system must support real-time messaging, file sharing, and voice calls.
	Design and analyze a complete solution:
	\begin{parts}
		\part[5] Specify your cryptographic algorithm choices for:
		\begin{itemize}
			\item Key exchange protocols
			\item Digital signature schemes
			\item Symmetric encryption algorithms
			\item Hash functions and MACs
		\end{itemize}
		\part[5] Describe your key management architecture. How do you bootstrap trust, distribute keys, and handle key rotation?
		\part[5] Analyze your system's security properties against various attack scenarios:
		\begin{itemize}
			\item Network eavesdropping
			\item Server compromise
			\item Endpoint compromise
			\item Insider attacks
		\end{itemize}
	\end{parts}
\end{questions}

\begin{tcolorbox}[colframe=EarthBrown!30!white,colback=EarthBrown!5!white]
	\section*{Bonus Question}
	\begin{questions}
		\bonusquestion[20] The transition to post-quantum cryptography will require replacing current elliptic curve systems with quantum-resistant alternatives. Research and analyze one of the following aspects of this transition:
		\begin{parts}
			\part \textbf{NIST Post-Quantum Standards:} Analyze the recently standardized ML-KEM and ML-DSA algorithms. How do their key sizes, performance characteristics, and security assumptions compare to current ECC systems?
			\part \textbf{Hybrid Classical/Post-Quantum Systems:} Describe approaches for combining classical and post-quantum algorithms during the transition period. What are the benefits and challenges of hybrid systems?
			\part \textbf{Migration Timeline and Challenges:} Analyze the practical challenges of migrating existing systems (browsers, mobile apps, IoT devices) from ECC to post-quantum cryptography. What factors determine the migration timeline?
		\end{parts}
		Your answer should include: current standardization status, performance comparisons with existing systems, deployment challenges, and recommendations for practitioners preparing for the post-quantum transition. Check the Optional Readings under the topic listing for the Post-Quantum Cryptography on the course website for helpful references!
	\end{questions}
\end{tcolorbox}

\end{document}
