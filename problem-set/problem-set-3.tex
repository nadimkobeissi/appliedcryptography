\documentclass[10pt,a4paper,american]{exam}
\usepackage{../misc/macros/classhandout}

\title{Applied Cryptography - Problem Set 3: Asymmetric Cryptography}
\author{Nadim Kobeissi}
\subject{Problem set covering asymmetric cryptography concepts including cryptographic hardness, Diffie-Hellman protocol, elliptic curve cryptography, and practical security applications.}
\keywords{asymmetric cryptography, public key cryptography, Diffie-Hellman, elliptic curves, discrete logarithm problem, cryptographic hardness, man-in-the-middle attacks, ECDSA, Ed25519, post-quantum cryptography}

\begin{document}
\classhandoutheader
\section*{Problem Set 3: Asymmetric Cryptography}

\begin{tcolorbox}[colframe=OliveGreen!30!white,colback=OliveGreen!5!white]
	\textbf{Instructions:} This problem set covers topics in provable security from topics 1.7\footnote{\url{https://appliedcryptography.page/slides/\#1-7}} and 1.8\footnote{\url{https://appliedcryptography.page/slides/\#1-8}} of the course. Submit your solutions as a neatly formatted PDF. You are encouraged to collaborate with classmates in studying the material, but your submitted solutions must be your own work. For proofs, clearly state your assumptions, steps, and conclusions.
\end{tcolorbox}

\section{Cryptographic Hardness and Real-World Implications}
\begin{questions}
	\question[10] \textbf{Breaking Cryptography: Attack Scenarios}

	\textbf{The Cryptographic Apocalypse Scenario:}
	Imagine you wake up tomorrow to headlines: \textit{``Breakthrough Algorithm Solves P vs NP - Computer Scientists Prove P = NP!''}
	\begin{parts}
		\part[5] As the Chief Security Officer of a major bank, write a crisis response memo outlining which systems fail immediately, which have grace periods, and what emergency measures you would implement.
		\part[5] Design an alternative security model for online banking that could work in a post-P=NP world. What assumptions would you rely on instead? Why wouldn't NP-complete problems save us in this scenario?
	\end{parts}

	\question[10] \textbf{Discrete Logarithm Security}

	\textbf{The Weak Parameter Disaster:}
	Your security audit discovers that a legacy system has been using $p = 2047$ (which factors as $23 \times 89$) for Diffie-Hellman key exchange, and the generator $g = 2$.
	\begin{parts}
		\part[5] Analyze exactly why this parameter choice is catastrophically weak. Estimate how long it would take an attacker with a modern laptop to break this system.
		\part[5] Design an emergency response plan: how do you migrate users to secure parameters while maintaining service availability? Compare the security implications if the system had instead used a proper 2048-bit prime but with a generator that only generates a small subgroup.
	\end{parts}
\end{questions}

\section{Diffie-Hellman in Hostile Environments}
\begin{questions}
	\question[15] \textbf{Attack and Defense Scenarios}

	\textbf{The Perfect Man-in-the-Middle:}
	An attacker has complete control over the network between Alice and Bob, can modify any message, and can initiate connections that appear to come from either party.
	\begin{parts}
		\part[5] Design the most effective man-in-the-middle attack against unauthenticated Diffie-Hellman. Your attack should be undetectable to Alice and Bob during the key exchange.
		\part[5] Alice and Bob have never met but each has the other's public key fingerprint written on a piece of paper. Design an authentication protocol that defeats your attack using only these fingerprints.
		\part[5] The attacker now has quantum capabilities. How does this change your attack and defense strategies? Compare your solution to certificate authorities and web-of-trust models.
	\end{parts}
\end{questions}

\section{Elliptic Curve Security Engineering}
\begin{questions}
	\question[15] \textbf{Curve Selection and Implementation}

	\begin{parts}
		\part[8] \textbf{The Government Backdoor Controversy:}
		You're the security architect for a new messaging app. Cryptographers are debating whether NIST P-256 contains a government backdoor, while Curve25519 offers better security properties but less widespread hardware support.
		\begin{subparts}
			\subpart Analyze the specific concerns about NIST curves: what would a backdoor look like, and how could it be exploited without breaking the underlying mathematical problems?
			\subpart Design a risk assessment framework for choosing between P-256 and Curve25519. What factors should influence your decision?
			\subpart Your legal team reports that several countries require NIST-compliant cryptography for government sales. Propose a solution that addresses both the backdoor concerns and the compliance requirements.
		\end{subparts}

		\part[7] \textbf{The Small Subgroup Attack on Curve25519:}
		A security audit reveals that your Curve25519 ECDH implementation doesn't properly handle low-order points, potentially leaking information about secret keys.
		\begin{subparts}
			\subpart Explain how an attacker could exploit Curve25519's cofactor of 8 to mount a small subgroup attack. What specific low-order points could be used, and what information about the secret key could be recovered?
			\subpart Design a defense strategy that prevents small subgroup attacks while maintaining Curve25519's performance advantages. Should you reject low-order points, or use a different approach? Justify your choice.
		\end{subparts}
	\end{parts}
\end{questions}

\section{Applied Cryptography Case Study}
\begin{questions}
	\question[20] \textbf{Cryptocurrency Signature Scheme Analysis:}
	A new cryptocurrency project is choosing between ECDSA and Ed25519 for transaction signatures. The system requirements include:
	\begin{itemize}
		\item High transaction throughput (thousands of signatures per second)
		\item Long-term security (system should remain secure for decades)
		\item Compatibility with hardware wallets and mobile devices
		\item Deterministic transaction signing for reproducibility
	\end{itemize}
	Analyze this decision:
	\begin{parts}
		\part[7] Compare ECDSA and Ed25519 for each requirement above. Which algorithm better meets each criterion and why?
		\part[7] Discuss the implications of signature malleability. How does this affect each algorithm and why might it matter for cryptocurrency applications?
		\part[6] Analyze the quantum resistance of both options. What migration path would you recommend for long-term security? Make a final recommendation with justification.
	\end{parts}
\end{questions}

\begin{tcolorbox}[colframe=EarthBrown!30!white,colback=EarthBrown!5!white]
	\section*{Bonus Question}
	\begin{questions}
		\bonusquestion[20] The transition to post-quantum cryptography will require replacing current elliptic curve systems with quantum-resistant alternatives. Research and analyze one of the following aspects of this transition:
		\begin{parts}
			\part \textbf{NIST Post-Quantum Standards:} Analyze the recently standardized ML-KEM and ML-DSA algorithms. How do their key sizes, performance characteristics, and security assumptions compare to current ECC systems?
			\part \textbf{Hybrid Classical/Post-Quantum Systems:} Describe approaches for combining classical and post-quantum algorithms during the transition period. What are the benefits and challenges of hybrid systems?
			\part \textbf{Migration Timeline and Challenges:} Analyze the practical challenges of migrating existing systems (browsers, mobile apps, IoT devices) from ECC to post-quantum cryptography. What factors determine the migration timeline?
		\end{parts}
		Your answer should include: current standardization status, performance comparisons with existing systems, deployment challenges, and recommendations for practitioners preparing for the post-quantum transition. Check the Optional Readings under the topic listing for the Post-Quantum Cryptography on the course website for helpful references!
	\end{questions}
\end{tcolorbox}

\end{document}
