\documentclass[aspectratio=169, lualatex, handout]{beamer}
\makeatletter\def\input@path{{theme/}}\makeatother\usetheme{cipher}

\title{Applied Cryptography - 1.1: Introduction}
\author{Nadim Kobeissi}
\subject{Introduction to Applied Cryptography covering cryptographic primitives, protocols, security goals, and real-world applications from basic concepts to advanced topics like TLS and zero-knowledge proofs.}
\keywords{cryptography, security, encryption, authentication, hash functions, TLS, protocols, provable security}
\institute{American University of Beirut}
\instituteimage{images/aub_white.png}
\date{\today}
\coversubtitle{CMPS 297AD/396AI\\Fall 2025}
\coverpartname{Part 1: Provable Security}
\covertopicname{1.1: Introduction}
\coverwebsite{https://appliedcryptography.page}

\begin{document}
\begin{frame}[plain]
	\titlepage
\end{frame}

\section{Defining Cryptography}

\begin{frame}{Defining cryptography}
	\begin{columns}[c]
		\begin{column}{0.5\textwidth}
			\definitionbox{What is Cryptography?}{\textit{``The science of enabling secure and private computation, communication, verification, and delegation in the presence of untrusted parties, adversarial behavior, and mutually distrustful participants.''}}
		\end{column}

		\begin{column}{0.5\textwidth}
			\imagewithcaption{caesar.png}{Source: Serious Cryptography, 2nd Edition}
		\end{column}
	\end{columns}
\end{frame}

\begin{frame}{Defining cryptography}
	\begin{columns}[c]
		\begin{column}{0.5\textwidth}
			\definitionbox{What is Cryptography?}{\textit{``The science of enabling secure and private computation, communication, verification, and delegation in the presence of untrusted parties, adversarial behavior, and mutually distrustful participants.''}}
		\end{column}

		\begin{column}{0.5\textwidth}
			\imagewithcaption{vigenere.png}{Source: Serious Cryptography, 2nd Edition}
		\end{column}
	\end{columns}
\end{frame}

\begin{frame}{Cryptography is everywhere}
	\begin{columns}[c]
		\begin{column}{0.5\textwidth}
			\begin{itemize}[<+->]
				\item Banking
				\item Buying stuff from the store
				\item Any digital payment system
				\item Messaging (WhatsApp, Signal, iMessage, Telegram)
				\item Voice calls
				\item Government and military systems
				\item SSH
				\item VPN access
				\item Visiting most websites (HTTPS)
			\end{itemize}
		\end{column}
		\begin{column}{0.5\textwidth}
			\begin{itemize}[<+->]
				\item Disk encryption
				\item Cloud storage
				\item Video conferencing
				\item Unlocking your (newer) car
				\item Identity card systems
				\item Ticketing systems
				\item DRM solutions
				\item Private contact discovery
				\item Cryptocurrencies
				\item That Apple Photos feature that detects similar photos
			\end{itemize}
		\end{column}
	\end{columns}
\end{frame}

\begin{frame}{How it's made}
	\bigimagewithcaption{fischer.png}{Fischer et al., The Challenges of Bringing Cryptography from Research Papers to Products: Results from an Interview Study with Experts, USENIX Security 2024}
\end{frame}

\begin{frame}{How it's made}
	\begin{center}
		\bigimagewithcaption{fischer_sectioned.png}{Fischer et al., The Challenges of Bringing Cryptography from Research Papers to Products: Results from an Interview Study with Experts, USENIX Security 2024}
	\end{center}
\end{frame}

\begin{frame}{Cryptographic building blocks}
	\begin{columns}[c]
		\begin{column}{0.5\textwidth}
			\textbf{Components}
			\begin{itemize}[<+->]
				\item Cryptography manifests as a set of primitives, from which we
				      build protocols intended to accomplish well-defined security goals.
				\item \textbf{Primitives}: AES, RSA, SHA-2, DH\ldots
				\item \textbf{Protocols}: TLS, Signal, SSH, FileVault 2, BitLocker\ldots
			\end{itemize}
		\end{column}

		\begin{column}{0.5\textwidth}
			\textbf{Examples}
			\begin{itemize}[<+->]
				\item \textbf{AES}: Symmetric encryption
				      \begin{itemize}
					      \item $\mathsf{Enc}(k, m) = c$, $\mathsf{Dec}(k, c) = m$.
				      \end{itemize}
				\item \textbf{SHA-2}: Hash function
				      \begin{itemize}
					      \item $\mathsf{H}(m) = h$.
				      \end{itemize}
				\item \textbf{Diffie-Hellman}: Public key agreement
				      \begin{itemize}
					      \item Allows two parties to agree on a secret key $k$.
				      \end{itemize}
			\end{itemize}
		\end{column}
	\end{columns}
\end{frame}

\begin{frame}{Cryptographic building blocks}
	\begin{columns}[c]
		\begin{column}{0.5\textwidth}
			\textbf{Security goals}
			\begin{itemize}[<+->]
				\item \textbf{Confidentiality}: Data exchanged between Client and Server
				      is only known to those parties.
				\item \textbf{Authentication}: If Server receives data from Client,
				      then Client sent it to Server.
				\item \textbf{Integrity}: If Server modifies data owned by Client,
				      Client can find out.
			\end{itemize}
		\end{column}

		\begin{column}{0.5\textwidth}
			\textbf{Examples}
			\begin{itemize}[<+->]
				\item \textbf{Confidentiality}: When you send a private message on Signal,
				      only you and the recipient can read the content.
				\item \textbf{Authentication}: When you receive an email from your boss,
				      you can verify it actually came from them.
				\item \textbf{Integrity}: Your computer can verify that software update
				      downloads haven't been tampered with during transmission.
			\end{itemize}
		\end{column}
	\end{columns}
\end{frame}

\begin{frame}{Security goals: more examples}
	\begin{itemize}[<+->]
		\item \textbf{TLS (HTTPS)} ensures that data exchanged between the client
		      and the server is confidential and that parties are authenticated.
		      \begin{itemize}
			      \item Allows you to log into gmail.com without your ISP learning your password.
		      \end{itemize}
		\item \textbf{FileVault 2} ensures data confidentiality and integrity on
		      your MacBook.
		      \begin{itemize}
			      \item Prevents thieves from accessing your data if your MacBook is stolen.
		      \end{itemize}
		\item \textbf{Signal} implements post-compromise security, an advanced security
		      goal.
		      \begin{itemize}
			      \item Allows a conversation to ``heal'' in the event of a temporary key
			            compromise.
			      \item More on that later in the course.
		      \end{itemize}
	\end{itemize}
\end{frame}

\begin{frame}{Why bother?}
	\begin{itemize}[<+->]
		\item Can't we just use access control?
		\item Strictly speaking, usernames and passwords can be implemented
		      without cryptography\ldots
		\item Server checks if the password matches, or if the IP address matches,
		      etc. before granting access.
		\item What's so bad about that?
	\end{itemize}
	\definitionbox{The problem with traditional access control}{
		\begin{itemize}[<+->]
			\item Requires trusting the server completely
			\item No protection during transmission
			\item No way to verify integrity
			\item No way to establish trust between strangers
		\end{itemize}
	}
\end{frame}

\begin{frame}[c]{The magic of cryptography}
	\begin{center}
		\Large\textbf{Cryptography lets us achieve what seems impossible}
		\vspace{1cm}
		\begin{itemize}[<+->]
			\item Secure communication over insecure channels
			\item Verification without revealing secrets
			\item Proof of computation without redoing it
		\end{itemize}
	\end{center}
\end{frame}

\section{Examples of the Magic in Cryptography}

\begin{frame}{Hard problems}
	\begin{itemize}[<+->]
		\item Cryptography is largely about equating the security of a system to the
		      difficulty of solving a math problem that is thought to be computationally
		      very expensive.
		\item With cryptography, we get security systems that we can literally
		      mathematically prove as secure (under assumptions).
		\item Also, this allows for actual magic.
		      \begin{itemize}[<+->]
			      \item Alice and Bob meet for the first time in the same room as you.
			      \item You are listening to everything they are saying.
			      \item Can they exchange a secret without you learning it?
		      \end{itemize}
	\end{itemize}
\end{frame}

\begin{frame}{Time for actual magic}
	\bigimagewithcaption{dh.png}{}
\end{frame}

\begin{frame}{No known feasible computation}
	\begin{itemize}[<+->]
		\item The discrete logarithm problem:
		      \begin{itemize}
			      \item Given a finite cyclic group $G$, a generator $g \in G$, and an element
			            $h \in G$, find the integer $x$ such that $g^{x}=h$
		      \end{itemize}
		\item In more concrete terms:
		      \begin{itemize}
			      \item Let $p$ be a large prime and let $g$ be a generator of the multiplicative
			            group $\mathbb{Z}_{p}^{*}$ (all nonzero integers modulo $p$).

			      \item Given:
			            \begin{itemize}
				            \item $g \in \mathbb{Z}_{p}^{*}$, $h \in \mathbb{Z}_{p}^{*}$

				            \item Find $x \in \{0, 1, \ldots, p-2\}$ such that $g^{x} \equiv h \pmod
					                  {p}$
			            \end{itemize}

			      \item This problem is believed to be computationally hard when $p$ is large
			            and $g$ is a primitive root modulo $p$.
			            \begin{itemize}
				            \item ``Believed to be'' = we don't know of any way to do it that doesn't
				                  take forever, unless we have a strong, stable quantum computer (Shor's
				                  algorithm)
			            \end{itemize}
		      \end{itemize}
	\end{itemize}
\end{frame}

\begin{frame}{Signal's double ratchet: DH everywhere}
	\begin{columns}[c]
		\begin{column}{0.7\textwidth}
			\begin{itemize}[<+->]
				\item \textbf{Initial key exchange}: Uses X3DH (Extended Triple DH)
				      \begin{itemize}
					      \item Combines \textbf{three} DH key exchanges for security.
					      \item Works even when recipient is offline (\textit{``asynchronous''} protocol).\footnote{Everything on this slide will be covered in much more detail later in the course.}
				      \end{itemize}
				\item \textbf{Ongoing communication}: Uses Double Ratchet
				      \begin{itemize}
					      \item New DH key exchange for every message!
					      \item Provides ``forward secrecy'' and ``post-compromise security''.
					      \item If your phone gets compromised today, yesterday's messages remain secure.
					      \item If your phone recovers from compromise, tomorrow's messages are secure again.
				      \end{itemize}
			\end{itemize}
		\end{column}
		\begin{column}{0.3\textwidth}
			\imagewithcaption{signal.jpg}{Signal uses DH key exchange dozens, hundreds of times per conversation.}
		\end{column}
	\end{columns}
\end{frame}

\begin{frame}{Hard problems}
	\begin{columns}[c]
		\begin{column}{0.5\textwidth}
			\textbf{Asymmetric Primitives}
			\begin{itemize}[<+->]
				\item Diffie-Hellman, RSA, ML-KEM, etc.
				\item ``Asymmetric'' because there is a ``public key'' and a ``private
				      key'' for each party.
				\item Algebraic, assume the hardness of mathematical problems (as seen
				      just now.)
			\end{itemize}
		\end{column}

		\begin{column}{0.5\textwidth}
			\textbf{Symmetric Primitives}
			\begin{itemize}[<+->]
				\item AES, SHA-2, ChaCha20, HMAC\ldots
				\item ``Symmetric'' because there is one secret key.
				\item Not algebraic but unstructured, but on their understood
				      resistance to $n$ years of cryptanalysis.
				\item Can act as substitutes for assumptions in security proofs!
				      \begin{itemize}
					      \item Example: hash function assumed to be a ``random oracle''
				      \end{itemize}
			\end{itemize}
		\end{column}
	\end{columns}
\end{frame}

\begin{frame}{Kerckhoff's principle}
	\begin{itemize}[<+->]
		\item \textit{``A cryptosystem should be secure even if everything about
			      the system, except the key, is public knowledge.''} — Auguste Kerckhoffs,
		      1883
		\item \textbf{Why it matters}:
		      \begin{itemize}[<+->]
			      \item No ``security through obscurity''
			      \item The key is the only secret: the rest can be audited, tested,
			            trusted
			      \item Encourages open standards and peer review
			      \item If your system's security depends on nobody knowing how it works,
			            it's not secure.
		      \end{itemize}
	\end{itemize}
\end{frame}

\begin{frame}{Symmetric primitive example: hash functions}
	\begin{columns}[c]
		\begin{column}{0.55\textwidth}
			\definitionbox{Hash Function Properties}{
				\begin{itemize}\item Takes input of \textbf{any size}
					\item Produces output of \textbf{fixed size}
					\item Is \textbf{deterministic} (same input $\rightarrow$ same output)
					\item Even a \textbf{tiny change} in input creates completely different output
					\item Is \textbf{efficient} to compute\end{itemize}
			}
		\end{column}
		\begin{column}{0.45\textwidth}
			\begin{tcolorbox}
				[colback=black!5!white,colframe=ciphergray] $\mathsf{SHA256}(\texttt{hello}) =$ \\ {\small\texttt{2cf24dba5fb0a30e26e83b2ac5}\\ \texttt{b9e29e1b161e5c1fa7425e7304}\\
				\texttt{3362938b9824}}\\[1em]
				$\mathsf{SHA256}(\texttt{hullo}) =$ \\ {\small\texttt{7835066a1457504217688c8f5d}\\
				\texttt{06909c6591e0ca78c254ccf174}\\ \texttt{50d0d999cab0}}
			\end{tcolorbox}
			\textcolor{cipherprimary}{\textbf{Note:} \small One character change $\rightarrow$
				completely different hash!}
		\end{column}
	\end{columns}
\end{frame}

\begin{frame}{Expected properties of a hash function}
	\begin{columns}[c]
		\begin{column}{0.6\textwidth}
			\begin{itemize}[<+->]
				\item \textbf{Collision resistance}: computationally infeasible to find
				      two different inputs producing the same hash.
				\item \textbf{Preimage resistance}: given the output of a hash function,
				      it is computationally infeasible to reconstruct the original input.
				\item \textbf{Second preimage resistance}: given an input and an output,
				      it's computationally infeasible to find another different input
				      producing the same output.
			\end{itemize}
		\end{column}
		\begin{column}{0.4\textwidth}
			\imagewithcaption{sha2.png}{SHA-2 compression function. Source: Wikipedia}
		\end{column}
	\end{columns}
\end{frame}

\begin{frame}{Hash functions: what are they good for?}
	\begin{itemize}[<+->]
		\item \textbf{Password storage}: Store the hash of the password on the server,
		      not the password itself. Then check candidate passwords against the hash.
		\item \textbf{Data integrity verification}: Hash a file. Later hash it
		      again and compare hashes to check if the file has changed, suffered storage
		      degradation, etc.
		\item \textbf{Proof of work}: Server asks client to hash something a lot of
		      times before they can access some resource. Useful for anti-spam, Bitcoin
		      mining, etc.
		\item \textbf{Zero knowledge proofs}: time for more actual magic
	\end{itemize}
\end{frame}

\begin{frame}{Time for more actual magic}
	\begin{columns}[c]
		\begin{column}{0.6\textwidth}
			\begin{itemize}[<+->]
				\item \textbf{Zero-knowledge proofs} allow you to prove that you know
				      a secret without revealing any information about it.
				\item They built ``zero-knowledge virtual machines'' where you can execute
				      an entire program that runs as a zero-knowledge proof.
				\item ZKP battleship game: server proves to the players that its
				      output to their battleship guesses is correct, without revealing any
				      additional information (e.g. ship location).
			\end{itemize}
		\end{column}
		\begin{column}{0.4\textwidth}
			\imagewithcaption{battleship.jpg}{Battleship board game. Source: Hasbro}
		\end{column}
	\end{columns}
\end{frame}

\begin{frame}{Evaluating a hash function's quality}
	\begin{columns}[c]
		\begin{column}{0.6\textwidth}
			\begin{itemize}[<+->]
				\item \textbf{Recall}:
				      \begin{itemize}[<+->]
					      \item \textbf{Asymmetric primitives} are based on mathematical
					            problems, can be mathematically proven secure (given assumptions!)
					      \item \textbf{Symmetric primitives} (encryption, hashing\ldots)
					            are statistically, empirically, heuristically shown to be secure,
					            not proven secure.
					      \item The more cryptanalysis they survive, the higher confidence
					            we have in their security.
				      \end{itemize}
			\end{itemize}
		\end{column}

		\begin{column}{0.4\textwidth}
			\imagewithcaption{qiao.png}{Cryptanalysis of AES.}
		\end{column}
	\end{columns}
\end{frame}

\begin{frame}{What about encryption?}
	\begin{columns}[c]
		\begin{column}{0.6\textwidth}
			\begin{itemize}[<+->]
				\item Symmetric primitive of choice for encryption: \textbf{AES}.
				\item Not that far off in terms of design process from hash functions,
				      but:
				      \begin{itemize}[<+->]
					      \item AES is a PRP (pseudorandom permutation)
					      \item HMAC-SHA256 is a PRF (pseudorandom function)
				      \end{itemize}
			\end{itemize}
		\end{column}

		\begin{column}{0.4\textwidth}
			\imagewithcaption{aes_subbytes.png}{AES's SubBytes operation. Source: Wikipedia}
		\end{column}
	\end{columns}
\end{frame}

\begin{frame}{PRF versus PRP}
	\begin{columns}[c]
		\begin{column}{0.5\textwidth}
			\textbf{Pseudo-Random Function (SHA-2)}
			\begin{itemize}[<+->]
				\item \textbf{Input} is arbitrary-length,
				\item \textbf{Output} is fixed-length, looks random (as discussed
				      earlier).
				\item Indistinguishable from a truly random function by an adversary with
				      limited computational power.
			\end{itemize}
		\end{column}

		\begin{column}{0.5\textwidth}
			\textbf{Pseudo-Random Permutation (AES)}
			\begin{itemize}[<+->]
				\item \textbf{Input and output} are the same length, forming a permutation.
				\item Each input maps uniquely to one output, allowing invertibility.
				\item Indistinguishable from a truly random permutation by an adversary
				      with limited computational power.
			\end{itemize}
		\end{column}
	\end{columns}
\end{frame}

\begin{frame}{$\mathsf{PRF}: F_{k}= X \rightarrow Y$}
	\begin{columns}[c]
		\begin{column}{0.4\textwidth}
			\begin{itemize}
				\item We want the mapping to be:
				      \begin{itemize}
					      \item One-way
					      \item ``Randomized''
					      \item Relations between inputs not reflected in outputs
				      \end{itemize}
			\end{itemize}
		\end{column}

		\begin{column}{0.8\textwidth}
			\begin{tikzpicture}[scale=0.38]
				% Define colors
				\definecolor{domaingreen}{RGB}{102, 170, 68}
				\definecolor{rangegreen}{RGB}{170, 187, 136}
				\definecolor{circlecolor}{RGB}{235, 137, 85}
				\definecolor{purplearrow}{RGB}{160, 78, 160}
				\definecolor{redarrow}{RGB}{237, 50, 36}

				% Input space (domain) X - made square
				\draw[dashed, thick, domaingreen, fill=domaingreen]
				(0,0) rectangle (8,8);
				\node[text width=6.5cm, align=center, font=\normalsize]
				at
				(4,-0.8)
				{Size: infinite!};
				\node[font=\small] at (4,9) {Input space (domain) $X$};

				% Output (range) Y - made square - moved more to the right
				\draw[thick, rangegreen, fill=rangegreen] (15,2) rectangle (20,7);
				\node[text width=4cm, align=center, font=\normalsize]
				at
				(17.5,1.2)
				{Size: fixed};
				\node[font=\small] at (17.5,8.5) {Output (range) $Y$};
				% Input dots - adjusted positions for square domain
				\filldraw[circlecolor] (2,7) circle (0.3);
				\pause
				\draw[-{Stealth[length=6mm, width=4mm]}, thick, purplearrow]
				(2,7) -- (16.2,6.4);
				\pause
				\filldraw[circlecolor] (16.2,6.4) circle (0.3);
				\pause

				\filldraw[circlecolor] (3,6) circle (0.3);
				\pause
				\draw[-{Stealth[length=6mm, width=4mm]}, thick, purplearrow]
				(3,6) -- (18.6,5.3);
				\pause
				\filldraw[circlecolor] (18.6,5.3) circle (0.3);
				\pause

				\filldraw[circlecolor] (2,5) circle (0.3);
				\pause
				\draw[-{Stealth[length=6mm, width=4mm]}, thick, purplearrow]
				(2,5) -- (16.8,4.2);
				\pause
				\filldraw[circlecolor] (16.8,4.2) circle (0.3);
				\pause

				\filldraw[circlecolor] (4,3.5) circle (0.3);
				\pause
				\draw[-{Stealth[length=6mm, width=4mm]}, thick, purplearrow]
				(4,3.5) -- (18.4,3.2);
				\pause
				\filldraw[circlecolor] (18.4,3.2) circle (0.3);
				\pause

				\filldraw[circlecolor] (2,2) circle (0.3);
				\pause
				\draw[-{Stealth[length=6mm, width=4mm]}, thick, purplearrow]
				(2,2) -- (17.1,2.7);
				\pause
				\filldraw[circlecolor] (17.1,2.7) circle (0.3);
				\pause

				\filldraw[circlecolor] (3,1) circle (0.3);
				\pause
				\draw[-{Stealth[length=6mm, width=4mm]}, ultra thick, redarrow]
				(3,1) -- (16.8,4.2);
				\node[redarrow, font=\scriptsize\bfseries, rotate=14]
				at
				(10,3)
				{Collisions are inevitable};
			\end{tikzpicture}
		\end{column}
	\end{columns}
\end{frame}

\begin{frame}{$\mathsf{PRP}: F_{k}= X \rightarrow X$}
	\begin{columns}[c]
		\begin{column}{0.4\textwidth}
			\begin{itemize}
				\item \textbf{Bijective} (two-way)
				      \begin{itemize}
					      \item \textbf{Injective}: no two inputs map to same output (no
					            collisions)
					      \item \textbf{Surjective}: Every output has one corresponding input
				      \end{itemize}
				\item ``Randomized''
				\item Relations between inputs not reflected in outputs
			\end{itemize}
		\end{column}

		\begin{column}{0.8\textwidth}
			\begin{tikzpicture}[scale=0.38]
				% Define colors
				\definecolor{domaingreen}{RGB}{102, 170, 68}
				\definecolor{rangegreen}{RGB}{102, 170, 68}
				\definecolor{circlecolor}{RGB}{235, 137, 85}
				\definecolor{purplearrow}{RGB}{160, 78, 160}

				% Input space (domain) X - made square
				\draw[dashed, thick, domaingreen, fill=domaingreen]
				(0,0) rectangle (8,8);
				\node[text width=6.5cm, align=center, font=\normalsize]
				at
				(4,-0.8)
				{Size: fixed};
				\node[font=\normalsize] at (4,9) {Input space (domain) $X$};

				% Output (range) Y - made square, same size as domain, moved left
				\draw[thick, rangegreen, fill=rangegreen] (12,0) rectangle (20,8);
				\node[text width=6.5cm, align=center, font=\normalsize]
				at
				(16,-0.8)
				{Size: fixed};
				\node[font=\normalsize] at (16,9) {Output (range) $X$};
				% Input dots - adjusted positions for square domain
				\filldraw[circlecolor] (2,7) circle (0.3);
				\pause
				\draw[-{Stealth[length=6mm, width=4mm]}, thick, purplearrow]
				(2,7) -- (14.2,7.4);
				\pause
				\filldraw[circlecolor] (14.2,7.4) circle (0.3);
				\pause

				\filldraw[circlecolor] (3,6) circle (0.3);
				\pause
				\draw[-{Stealth[length=6mm, width=4mm]}, thick, purplearrow]
				(3,6) -- (18.6,5.3);
				\pause
				\filldraw[circlecolor] (18.6,5.3) circle (0.3);
				\pause

				\filldraw[circlecolor] (2,5) circle (0.3);
				\pause
				\draw[-{Stealth[length=6mm, width=4mm]}, thick, purplearrow]
				(2,5) -- (13.8,4.2);
				\pause
				\filldraw[circlecolor] (13.8,4.2) circle (0.3);
				\pause

				\filldraw[circlecolor] (4,3.5) circle (0.3);
				\pause
				\draw[-{Stealth[length=6mm, width=4mm]}, thick, purplearrow]
				(4,3.5) -- (17.4,2.2);
				\pause
				\filldraw[circlecolor] (17.4,2.2) circle (0.3);
				\pause

				\filldraw[circlecolor] (2,2) circle (0.3);
				\pause
				\draw[-{Stealth[length=6mm, width=4mm]}, thick, purplearrow]
				(2,2) -- (16.1,6.7);
				\pause
				\filldraw[circlecolor] (16.1,6.7) circle (0.3);
				\pause

				\filldraw[circlecolor] (3,1) circle (0.3);
				\pause
				\draw[-{Stealth[length=6mm, width=4mm]}, thick, purplearrow]
				(3,1) -- (19.0,1.4);
				\pause
				\filldraw[circlecolor] (19.0,1.4) circle (0.3);
			\end{tikzpicture}
		\end{column}
	\end{columns}
\end{frame}

\begin{frame}{AES is a block cipher}
	\begin{itemize}[<+->]
		\item AES takes a 16-byte input, produces a 16-byte output.
		\item Key can be 16, 24 or 32 bytes.
		\item OK, so what if we want to encrypt more than 16 bytes?
		\item \textbf{Proposal}: split the plaintext into 16 byte chunks, encrypt
		      each of them with the same key.
	\end{itemize}
\end{frame}

\begin{frame}{Block cipher examples}
	\begin{columns}
		\begin{column}{0.33\textwidth}
			\imagewithcaption{tux_plaintext.png}{What we start with}
		\end{column}
		\pause
		\begin{column}{0.33\textwidth}
			\imagewithcaption{tux_encrypted_ecb.png}{What we get}
		\end{column}
		\pause
		\begin{column}{0.33\textwidth}
			\imagewithcaption{tux_encrypted_ctr.png}{What we actually want}
		\end{column}
	\end{columns}
\end{frame}

\begin{frame}{Block cipher modes of operation}
	\bigimagewithcaption{block_cipher_modes.png}{Source: Wikipedia}
\end{frame}

\begin{frame}{Cryptographic building blocks}
	\begin{columns}[c]
		\begin{column}{0.5\textwidth}
			\textbf{Security goals}
			\begin{itemize}
				\item \textbf{Confidentiality}: Data exchanged between Client and Server
				      is only known to those parties.

				\item \textbf{Authentication}: If Server receives data from Client,
				      then Client sent it to Server.

				\item \textbf{Integrity}: If Server modifies data owned by Client,
				      Client can find out.
			\end{itemize}
		\end{column}

		\begin{column}{0.5\textwidth}
			\textbf{Examples}
			\begin{itemize}[<+->]
				\item \textbf{Confidentiality}: When you send a private message on Signal,
				      only you and the recipient can read the content.
				\item \textbf{Authentication}: When you receive an email from your boss,
				      you can verify it actually came from them.
				\item \textbf{Integrity}: Your computer can verify that software update
				      downloads haven't been tampered with during transmission.
			\end{itemize}
		\end{column}
	\end{columns}
\end{frame}

\begin{frame}{Security goals: more examples}
	\begin{itemize}[<+->]
		\item \textbf{TLS (HTTPS)} ensures that data exchanged between the client
		      and the server is confidential and that parties are authenticated.
		      \begin{itemize}
			      \item Allows you to log into gmail.com without your ISP learning your password.
		      \end{itemize}
		\item \textbf{FileVault 2} ensures data confidentiality and integrity on
		      your MacBook.
		      \begin{itemize}
			      \item Prevents thieves from accessing your data if your MacBook is stolen.
		      \end{itemize}
		\item \textbf{Signal} implements post-compromise security, an advanced security
		      goal.
		      \begin{itemize}
			      \item Allows a conversation to ``heal'' in the event of a temporary key
			            compromise.

			      \item More on that later in the course.
		      \end{itemize}
	\end{itemize}
\end{frame}

\begin{frame}{TLS 1.3: high-level sketch}
	\bigimagewithcaption{tls_13_sketch}{Source: Mostafa Ibrahim}
\end{frame}

\begin{frame}{TLS 1.3: high-level sketch}
	\begin{columns}[c]
		\begin{column}{0.5\textwidth}
			\begin{itemize}[<+->]
				\item \textbf{Public key agreement} (eg. Diffie-Hellman) is used to establish
				      a shared secret between the client and the server.
				\item \textbf{AES} is used for encrypting data in transit.
				\item \textbf{SHA-2} is used for hashing (checking certificates, etc.)
			\end{itemize}
		\end{column}

		\begin{column}{0.5\textwidth}
			\bigimagewithcaption{tls_13_sketch}{Source: Mostafa Ibrahim}
		\end{column}
	\end{columns}
\end{frame}

\begin{frame}{TLS 1.3: high-level sketch}
	\begin{columns}[c]
		\begin{column}{0.5\textwidth}
			\begin{itemize}[<+->]
				\item Through the design, we accomplish the desired \textbf{security
					      goals} under a well-specified \textbf{threat model}:
				\item \textbf{Security goals}: confidentiality of data, authentication
				      of the server towards the client\ldots
				\item \textbf{Threat model}: malicious Internet Service Provider (ISP),
				      etc.
			\end{itemize}
		\end{column}

		\begin{column}{0.5\textwidth}
			\bigimagewithcaption{tls_13_sketch}{Source: Mostafa Ibrahim}
		\end{column}
	\end{columns}
\end{frame}

\begin{frame}{How TLS 1.3 was made}
	\bigimagewithcaption{fischer}{}
\end{frame}

\begin{frame}{How TLS 1.3 was made}
	\bigimagewithcaption{fischer_tls_13_bubbles}{}
\end{frame}

\begin{frame}{From hard problems to real-world security}
	\begin{center}
		\Large\textbf{The journey we'll trace}
	\end{center}
	\vspace{0.5cm}
	\begin{enumerate}[<+->]
		\item \textbf{Mathematical insight}: Discrete logarithm is hard to compute.
		\item \textbf{Cryptographic innovation}: Diffie-Hellman key exchange leverages this hardness.
		\item \textbf{Real-world impact}: Secure communication for billions of people daily.
	\end{enumerate}
	\vspace{1cm}
	\textbf{This is the power of applied cryptography}: transforming abstract mathematical problems into tools that help people and protect our digital lives.
\end{frame}

\section{Course Goals}

\begin{frame}{Course goals}
	\begin{itemize}[<+->]
		\item Understand the reasoning behind the math of modern cryptography.
		\item Analyze and prove the security of cryptographic constructions.
		\item Understand how cryptographic constructions can be composed to build real-world
		      secure protocols and systems.
		\item Discern between theoretical cryptography and applied cryptography from
		      an engineering perspective.
		\item Critically assess security implementations and evaluate real-world cryptographic
		      protocols.
		\item Gain an understanding of the future of cryptography and its role in emerging
		      technologies.
	\end{itemize}
\end{frame}

\begin{frame}{Course prerequisites}
	\begin{itemize}
		\item Good but optional: CMPS 215 (Theory of Computation)
		\item If you want to understand whether you have the sufficient background for this course, review this revision chapter and try to do all the exercises: \url{https://joyofcryptography.com/pdf/chap0.pdf}
	\end{itemize}
\end{frame}

\begin{frame}{Class resources}
	\begin{itemize}[<+->]
		\item \textbf{Joy of Cryptography}: learn how to reason about and prove systems secure.
		\item \textbf{Attack papers, codebases, labs}: hard engineering perspective.
		      \vspace{1cm}
		\item \textbf{Always keep an eye on the website:} Course news, updates,
		      materials, slides will all be posted there.
		      \url{https://appliedcryptography.page}
		\item I am aiming for the most engaging course possible!
	\end{itemize}
\end{frame}

\subsection{Lab Projects}

\begin{frame}{Lab projects: hands-on learning}
	\begin{itemize}[<+->]
		\item \textbf{Weekly lab sessions}: Hands-on complement to lectures.
		\item \textbf{Team structure}: Groups of 1-3 students.
		\item \textbf{Project selection}: Pick at most \textbf{two lab topics} to work on throughout the entire course.
		\item \textbf{What you'll do}:
		      \begin{itemize}
			      \item Experiment with real-world cryptographic libraries.
			      \item Simulate attacks and vulnerabilities.
			      \item Understand why certain security practices are necessary.
			      \item Use formal analysis tools.
		      \end{itemize}
	\end{itemize}
	\vspace{0.1cm}
	\begin{center}
		\Large\textcolor{cipherprimary}{\textbf{Lab project sheets available on the course website. \\ Pick one or suggest your own!}}
	\end{center}
\end{frame}

\begin{frame}{Lab Project A: Designing a Password Manager}
	\begin{columns}[c]
		\begin{column}{0.5\textwidth}
			\definitionbox{Build Your Digital Vault!}{
				Create a fortress for your passwords using real cryptographic techniques!
				\begin{itemize}
					\item Master password protection with key derivation
					\item Encrypted storage that even you can't break
					\item Secure password generation algorithms
				\end{itemize}
			}
		\end{column}
		\begin{column}{0.5\textwidth}
			\textbf{What you'll learn:}
			\begin{itemize}[<+->]
				\item How password managers \textit{actually} work
				\item Defend against password cracking attacks
				\item Memory scraping countermeasures
				\item Why ``forgot master password'' button doesn't exist
			\end{itemize}
			\vspace{0.5cm}
			\textcolor{cipherprimary}{\textbf{Never forget a password again... except one!}}
		\end{column}
	\end{columns}
\end{frame}

\begin{frame}{Lab Project B: Designing a Secure Messenger}
	\begin{columns}[c]
		\begin{column}{0.5\textwidth}
			\definitionbox{Build the Next Signal!}{
				Create your own end-to-end encrypted messaging app from scratch!
				\begin{itemize}
					\item Implement the Double Ratchet protocol
					\item Perfect forward secrecy by default
					\item Group messaging that stays private
				\end{itemize}
			}
		\end{column}
		\begin{column}{0.5\textwidth}
			\textbf{Real cryptography in action:}
			\begin{itemize}[<+->]
				\item Key exchange protocols that feel like magic
				\item Deniable authentication (plausible deniability!)
				\item Metadata protection techniques
				\item Secure key storage on real devices
			\end{itemize}
			\vspace{0.5cm}
			\textcolor{cipherprimary}{\textbf{Your messages, truly private!}}
		\end{column}
	\end{columns}
\end{frame}

\begin{frame}{Lab Project C: Protocol Modeling with ProVerif}
	\begin{columns}[c]
		\begin{column}{0.5\textwidth}
			\definitionbox{Prove It Secure!}{
				Design and mathematically verify your own TLS-like protocol!
				\begin{itemize}
					\item Formal verification with ProVerif
					\item Specify security properties rigorously
					\item Find attacks before hackers do
				\end{itemize}
			}
		\end{column}
		\begin{column}{0.5\textwidth}
			\textbf{Become a protocol designer:}
			\begin{itemize}[<+->]
				\item Design protocols that are \textit{provably} secure
				\item Learn how TLS was formally verified
				\item Discover subtle attack patterns
				\item Master formal methods used by pros
			\end{itemize}
			\vspace{0.5cm}
			\textcolor{cipherprimary}{\textbf{If you can't prove it, don't ship it!}}
		\end{column}
	\end{columns}
\end{frame}

\begin{frame}{Lab Project D: Zero-Knowledge Battleship Game}
	\begin{columns}[c]
		\begin{column}{0.5\textwidth}
			\definitionbox{Cryptographic Gaming!}{
				Build Battleship where cheating is mathematically impossible!
				\begin{itemize}
					\item Zero-knowledge proofs with RISC Zero
					\item Prove hits without revealing ship positions
					\item Verifiable gameplay, zero trust needed
				\end{itemize}
			}
		\end{column}
		\begin{column}{0.5\textwidth}
			\textbf{The magic of ZK proofs:}
			\begin{itemize}[<+->]
				\item Build a zkVM from scratch
				\item Create proofs that feel impossible
				\item Learn by building an actual game
				\item Impress your friends with crypto magic!
			\end{itemize}
			\vspace{0.5cm}
			\textcolor{cipherprimary}{\textbf{You sunk my battleship... provably!}}
		\end{column}
	\end{columns}
\end{frame}

\begin{frame}{Lab Project E: Post-Quantum Migration}
	\begin{columns}[c]
		\begin{column}{0.5\textwidth}
			\definitionbox{Quantum-Proof the Future!}{
				Prepare systems for the quantum computing era!
				\begin{itemize}
					\item Implement NIST's ML-KEM (Kyber)
					\item Deploy ML-DSA (Dilithium) signatures
					\item Design hybrid classical-quantum systems
				\end{itemize}
			}
		\end{column}
		\begin{column}{0.5\textwidth}
			\textbf{Tomorrow's cryptography today:}
			\begin{itemize}[<+->]
				\item Migrate real protocols to PQC
				\item Analyze performance impacts
				\item Handle message size explosions
				\item Future-proof existing systems
			\end{itemize}
			\vspace{0.5cm}
			\textcolor{cipherprimary}{\textbf{Beat quantum computers at their own game!}}
		\end{column}
	\end{columns}
\end{frame}

\begin{frame}{Lab Project F: Private Contact Discovery}
	\begin{columns}[c]
		\begin{column}{0.5\textwidth}
			\definitionbox{Find Friends with Privacy!}{
				Build WhatsApp-style contact discovery that's actually private!
				\begin{itemize}
					\item Private Set Intersection (PSI)
					\item Oblivious PRFs in action
					\item Zero leakage to servers
				\end{itemize}
			}
		\end{column}
		\begin{column}{0.5\textwidth}
			\textbf{Real-world privacy tech:}
			\begin{itemize}[<+->]
				\item Protect address books from servers
				\item Prevent enumeration attacks
				\item Handle thousands of contacts efficiently
				\item Used by Signal and WhatsApp!
			\end{itemize}
			\vspace{0.5cm}
			\textcolor{cipherprimary}{\textbf{Your contacts, your business!}}
		\end{column}
	\end{columns}
\end{frame}

\begin{frame}{Lab Project G: Privacy-Preserving Age Verification}
	\begin{columns}[c]
		\begin{column}{0.5\textwidth}
			\definitionbox{Prove Your Age Privately!}{
				Build ID checks that don't track you!
				\begin{itemize}
					\item Anonymous credentials
					\item Zero-knowledge age proofs
					\item Unlinkable verifications
				\end{itemize}
			}
		\end{column}
		\begin{column}{0.5\textwidth}
			\textbf{Digital ID done right:}
			\begin{itemize}[<+->]
				\item Prove ``over 18'' without revealing birthdate
				\item Prevent tracking across services
				\item Mobile-friendly verification
				\item Pedersen commitments \& Schnorr proofs
			\end{itemize}
			\vspace{0.5cm}
			\textcolor{cipherprimary}{\textbf{Old enough to enter, young enough for privacy!}}
		\end{column}
	\end{columns}
\end{frame}

\begin{frame}{Lab Project H: Time-Locked Message Capsule}
	\begin{columns}[c]
		\begin{column}{0.5\textwidth}
			\definitionbox{Send Messages to the Future!}{
				Create digital time capsules that can't be opened early!
				\begin{itemize}
					\item Timelock encryption with drand beacon
					\item Threshold BLS signatures
					\item Unstoppable future reveals
				\end{itemize}
			}
		\end{column}
		\begin{column}{0.5\textwidth}
			\textbf{Time travel for data:}
			\begin{itemize}[<+->]
				\item Messages that decrypt themselves
				\item Perfect for auctions \& commitments
				\item Build on League of Entropy infrastructure
				\item No key escrow, no early access!
			\end{itemize}
			\vspace{0.5cm}
			\textcolor{cipherprimary}{\textbf{Encrypt today, decrypt tomorrow!}}
		\end{column}
	\end{columns}
\end{frame}

\begin{frame}{Create Your Own Cryptographic Adventure!}
	\begin{columns}[c]
		\begin{column}{0.5\textwidth}
			\definitionbox{Be the Innovator!}{
				Design your own cryptographic project from scratch!
				\begin{itemize}
					\item Novel protocol implementation
					\item Security analysis of real systems
					\item Formal verification challenges
					\item Your creative crypto idea!
				\end{itemize}
			}
		\end{column}
		\begin{column}{0.5\textwidth}
			\textbf{The sky's the limit:}
			\begin{itemize}[<+->]
				\item Follow your cryptographic passion
				\item Solve a real problem you care about
				\item Combine techniques in new ways
				\item Potentially publishable research!
			\end{itemize}
			\vspace{0.5cm}
			\textcolor{cipherprimary}{\textbf{Your idea could change cryptography!}}
		\end{column}
	\end{columns}
\end{frame}

\subsection{Problem Sets}

\begin{frame}{Problem sets: theory meets practice}
	\begin{itemize}[<+->]
		\item \textbf{Already available}: All problem sets are posted on the course website.
		\item \textbf{Purpose}: Reinforce and deepen understanding of lecture material.
		\item \textbf{What to expect}:
		      \begin{itemize}
			      \item Theoretical proofs and problem-solving.
			      \item Short coding tasks and computational experiments.
			      \item Bridge abstract concepts with concrete applications.
		      \end{itemize}
		\item \textbf{When are they due?} I'll tell you weeks in advance.
	\end{itemize}
	\vspace{0.5cm}
	\begin{center}
		\Large\textcolor{cipherprimary}{\textbf{Check the website for all problem sets! \\ \url{https://appliedcryptography.page}}}
	\end{center}
\end{frame}

\subsection{Graduate Students: Special Instructions}

\begin{frame}{Graduate students: thesis integration}
	\definitionbox{Special Requirements for Graduate Students}{
		Graduate students are expected to discuss their thesis topic with me to develop an individual project that integrates with their research.
	}
	\vspace{0.1cm}
	\begin{itemize}[<+->]
		\item \textbf{Individual project}: Tailored to your thesis research.
		\item \textbf{Integration}: Apply course concepts to your specific research area.
		\item \textbf{Consultation}: Schedule a meeting early in the semester to discuss your thesis and project ideas.
		\item \textbf{Outcome}: Deepen your understanding of cryptography within your research context.
	\end{itemize}
\end{frame}

\subsection{Exams}

\begin{frame}{Exam transparency: no surprises!}
	\begin{columns}[c]
		\begin{column}{0.65\textwidth}
			\begin{itemize}[<+->]
				\item \textbf{What's covered}: Specific topics, chapters, and concepts clearly outlined.
				\item \textbf{Format details}: Question types, duration, and grading scheme all disclosed.
				\item \textbf{Practice materials}: Sample questions are available through the problem sets.
				\item \textbf{No stress approach}: Focus on understanding concepts, not guessing what might appear.
				\item \textbf{If it wasn't mentioned in the slides or problem sets, then it's not on the exam.}
			\end{itemize}
		\end{column}
		\begin{column}{0.35\textwidth}
			\imagewithcaption{website_midterm.png}{Check the website for complete exam information! \\ \url{https://appliedcryptography.page}}
		\end{column}
	\end{columns}
\end{frame}

\begin{frame}{Self-assessment quizzes: test your understanding}
	\begin{columns}[c]
		\begin{column}{0.65\textwidth}
			\definitionbox{Optional Learning Quizzes}{
				\textit{Each class topic comes with an optional quiz to help you gauge your understanding.}
				\begin{itemize}
					\item \textbf{Not graded}: purely for self-assessment
					\item \textbf{Immediate feedback}: check your answers right away
					\item \textbf{Fun and engaging}: designed to reinforce key concepts
				\end{itemize}
			}
		\end{column}
		\begin{column}{0.35\textwidth}
			\imagewithcaption{website_quiz.png}{Each topic has its own quiz on the website (except this introduction): \\ \url{https://appliedcryptography.page}}
		\end{column}
	\end{columns}
\end{frame}

\subsection{The Key Exchange}

\begin{frame}{The Key Exchange: optional weekly gathering}
	\begin{columns}[c]
		\begin{column}{0.5\textwidth}
			\definitionbox{What is The Key Exchange?}{
				\textit{An optional weekly gathering for students intending to become researchers or professionals in cryptography.}
				\begin{itemize}
					\item Discuss cutting-edge research papers
					\item Practice presentation skills
					\item Develop scientific writing
					\item Explore career paths
				\end{itemize}
			}
		\end{column}
		\begin{column}{0.5\textwidth}
			\textbf{Why join?}
			\begin{itemize}[<+->]
				\item Learn to read research papers efficiently
				\item Present technical concepts clearly
				\item Write like a cryptographer
				\item Navigate career decisions with confidence
				\item Build lasting connections with peers
			\end{itemize}
			\vspace{0.5cm}
			\textcolor{cipherprimary}{\textbf{The only prerequisite is your curiosity!}}
		\end{column}
	\end{columns}
\end{frame}

\begin{frame}{The Key Exchange: four-week rotating schedule}
	\begin{columns}[c]
		\begin{column}{0.5\textwidth}
			\begin{itemize}
				\item \textbf{Week 1: Paper Deep Dive}
				      \begin{itemize}
					      \item Read and analyze research papers
					      \item Learn efficient reading strategies
					      \item Discuss recent CRYPTO/Eurocrypt papers
				      \end{itemize}
				\item \textbf{Week 2: Student Presentations}
				      \begin{itemize}
					      \item 15-minute talks on crypto topics
					      \item Practice with visual aids and demos
					      \item Receive constructive peer feedback
				      \end{itemize}
			\end{itemize}
		\end{column}
		\begin{column}{0.5\textwidth}
			\begin{itemize}
				\item \textbf{Week 3: Writing Workshop}
				      \begin{itemize}
					      \item Technical writing exercises
					      \item Blog posts and security reports
					      \item LaTeX tutorials for papers
				      \end{itemize}
				\item \textbf{Week 4: Career Café}
				      \begin{itemize}
					      \item Industry guest speakers
					      \item Graduate school applications
					      \item Interview preparation
				      \end{itemize}
			\end{itemize}
		\end{column}
	\end{columns}
	\vspace{0.1cm}
	\begin{center}
		\Large\textcolor{cipherprimary}{\textbf{When should we meet? \\ Let's find a time that works for everyone!}}
	\end{center}
\end{frame}

\subsection{About Me}

\begin{frame}{About me}
	\begin{columns}[c]
		\begin{column}{0.5\textwidth}
			\begin{itemize}
				\item \textbf{Academic Background}
				      \begin{itemize}[<+->]
					      \item Ph.D. Computer Science, Inria Paris (2018)
					      \item Thesis: \textit{``Formal Verification for Real-World Cryptographic Protocols''}
					      \item Advisors: Karthikeyan Bhargavan, Bruno Blanchet
				      \end{itemize}
			\end{itemize}
			\begin{itemize}
				\item \textbf{Current Roles}
				      \begin{itemize}[<+->]
					      \item Visiting Professor, AUB (2025-2026)
					      \item Senior Applied Cryptography Auditor, Cure53
					      \item Director, Symbolic Software
				      \end{itemize}
			\end{itemize}
		\end{column}
		\begin{column}{0.5\textwidth}
			\begin{itemize}
				\item \textbf{Research \& Practice}
				      \begin{itemize}[<+->]
					      \item 250+ security audits for major clients
					      \item Created Verifpal, Noise Explorer
					      \item Published at IEEE S\&P, USENIX Security, RWC
					      \item Program Committee: CCS, NDSS, PETS
				      \end{itemize}
			\end{itemize}
			\begin{itemize}
				\item \textbf{Fun Facts}
				      \begin{itemize}[<+->]
					      \item Active in cryptography for \approx 15 years now
					      \item Published indie games on Steam \& Nintendo
				      \end{itemize}
			\end{itemize}
		\end{column}
	\end{columns}
\end{frame}

\begin{frame}{My goals for this course}
	\begin{columns}[c]
		\begin{column}{0.5\textwidth}
			\definitionbox{My Commitment}{
				\textit{``To provide you with an excellent applied cryptography education that prepares you for success in this field.''}
				\begin{itemize}
					\item High standards and quality teaching
					\item Your success is my priority
					\item We'll work together to achieve this
				\end{itemize}
			}
		\end{column}
		\begin{column}{0.5\textwidth}
			\textbf{What I aim to deliver}
			\begin{itemize}[<+->]
				\item Strong foundation in cryptographic science and engineering
				\item Practical skills for research and industry
				\item Clear technical communication abilities
				\item A learning environment driven by \textbf{curiosity} and \textbf{intellectual excitement}
				\item Deep understanding over memorization
			\end{itemize}
		\end{column}
	\end{columns}
\end{frame}

\begin{frame}{My expectations: be intellectually present}
	\begin{columns}[c]
		\begin{column}{0.5\textwidth}
			\definitionbox{Minimum Requirements}{
				\begin{itemize}
					\item Attend all classes, labs, and exams, \textbf{on time}.
					\item Complete assignments, \textbf{on time}.
				\end{itemize}
			}
			\begin{itemize}
				\item \textbf{But I want you to do more:}
				      \begin{itemize}[<+->]
					      \item \textbf{Be curious}: Think deeply about what we're learning.
					      \item \textbf{Engage actively}: This is a conversation, not a monologue.
					      \item \textbf{Work hard}: Not because this is hard, but because you stand to learn so much!
				      \end{itemize}
			\end{itemize}
		\end{column}
		\begin{column}{0.5\textwidth}
			\begin{itemize}
				\item \textbf{Please, SPEAK UP!}
				      \begin{itemize}[<+->]
					      \item Don't understand something? \textbf{Ask!}
					      \item Have a question? \textbf{Voice it!}
					      \item Afraid of looking stupid? \textbf{Don't be!}
				      \end{itemize}
				\item \textbf{Why?}
				      \begin{itemize}[<+->]
					      \item I sometimes wasted weeks because my ego didn't allow me to ask ``stupid'' questions.
					      \item Don't worry: I've managed to make myself look dumber in front of the international cryptography community than you ever will.
				      \end{itemize}
			\end{itemize}
		\end{column}
	\end{columns}
\end{frame}

\subsection{About You}

\begin{frame}{This is YOUR learning journey}
	\begin{columns}[c]
		\begin{column}{0.5\textwidth}
			\textbf{What I need from you}
			\begin{itemize}[<+->]
				\item Think about what YOU want from this course.
				\item Express your interests and goals.
				\item Participate in shaping our discussions.
				\item I didn't come here just to watch you stare at me from a corner!
			\end{itemize}
		\end{column}
		\begin{column}{0.5\textwidth}
			\textbf{We're learning together}
			\begin{itemize}[<+->]
				\item This isn't just another course to check off.
				\item Applied cryptography is an exciting, specialized field.
				\item Your questions make the class better for everyone.
				\item Your engagement shapes the experience.
			\end{itemize}
		\end{column}
	\end{columns}
	\begin{center}
		\textcolor{cipherprimary}{\Large\textbf{Let's make this journey extraordinary together!}}
	\end{center}
\end{frame}

\begin{frame}{Let's get to know each other!}
	\begin{columns}[c]
		\begin{column}{0.5\textwidth}
			\definitionbox{Please introduce yourself!}{
				I'd love to learn more about each of you. Please share:
				\begin{itemize}
					\item Your name
					\item Your major and level (undergrad/grad)
					\item Your career aspirations
					\item What you hope to gain from this course
				\end{itemize}
			}
		\end{column}
		\begin{column}{0.5\textwidth}
			\textbf{Why this matters}
			\begin{itemize}[<+->]
				\item Helps me tailor examples to your interests
				\item Enables better project matching
				\item Builds our classroom community
				\item Allows me to connect course material to your goals
			\end{itemize}
		\end{column}
	\end{columns}
\end{frame}

\begin{frame}[plain]
	\titlepage
\end{frame}
\end{document}
