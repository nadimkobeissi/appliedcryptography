\documentclass[aspectratio=169, lualatex, handout]{beamer}
\makeatletter\def\input@path{{theme/}}\makeatother\usetheme{cipher}

\title{Applied Cryptography - 1.7: Hard Problems \& Diffie-Hellman}
\author{Nadim Kobeissi}
\subject{An introduction to computational complexity, hard problems, and the Diffie-Hellman key exchange protocol in cryptography.}
\keywords{cryptography, computational complexity, P vs NP, hard problems, discrete logarithm, Diffie-Hellman, key exchange, NP-complete, cryptographic protocols}
\institute{American University of Beirut}
\instituteimage{images/aub-white.png}
\date{\today}
\coversubtitle{CMPS 297AD/396AI\\Fall 2025}
\coverpartname{Part 1: Provable Security}
\covertopicname{1.7: Hard Problems \&\\Diffie-Hellman}
\coverwebsite{https://appliedcryptography.page}

\begin{document}
\begin{frame}[plain]
	\titlepage
\end{frame}

\begin{frame}{How it's made}
	\bigimagewithcaption{fischer.png}{Fischer et al., The Challenges of Bringing Cryptography from Research Papers to Products: Results from an Interview Study with Experts, USENIX Security 2024}
\end{frame}

\begin{frame}{Cryptographic building blocks}
	\begin{columns}[c]
		\begin{column}{0.5\textwidth}
			\textbf{Security goals}
			\begin{itemize}[<+->]
				\item \textbf{Confidentiality}: Data exchanged between Client and Server
				      is only known to those parties.
				\item \textbf{Authentication}: If Server receives data from Client,
				      then Client sent it to Server.
				\item \textbf{Integrity}: If Server modifies data owned by Client,
				      Client can find out.
			\end{itemize}
		\end{column}

		\begin{column}{0.5\textwidth}
			\textbf{Examples}
			\begin{itemize}[<+->]
				\item \textbf{Confidentiality}: When you send a private message on Signal,
				      only you and the recipient can read the content.
				\item \textbf{Authentication}: When you receive an email from your boss,
				      you can verify it actually came from them.
				\item \textbf{Integrity}: Your computer can verify that software update
				      downloads haven't been tampered with during transmission.
			\end{itemize}
		\end{column}
	\end{columns}
\end{frame}

\begin{frame}{Security goals: more examples}
	\begin{itemize}[<+->]
		\item \textbf{TLS (HTTPS)} ensures that data exchanged between the client
		      and the server is confidential and that parties are authenticated.
		      \begin{itemize}
			      \item Allows you to log into gmail.com without your ISP learning your password.
		      \end{itemize}
		\item \textbf{FileVault 2} ensures data confidentiality and integrity on
		      your MacBook.
		      \begin{itemize}
			      \item Prevents thieves from accessing your data if your MacBook is stolen.
		      \end{itemize}
		\item \textbf{Signal} implements post-compromise security, an advanced security
		      goal.
		      \begin{itemize}
			      \item Allows a conversation to ``heal'' in the event of a temporary key
			            compromise.
			      \item More on that later in the course.
		      \end{itemize}
	\end{itemize}
\end{frame}

\begin{frame}{Why bother?}
	\begin{itemize}[<+->]
		\item Can't we just use access control?
		\item Strictly speaking, usernames and passwords can be implemented
		      without cryptography\ldots
		\item Server checks if the password matches, or if the IP address matches,
		      etc. before granting access.
		\item What's so bad about that?
	\end{itemize}
	\definitionbox{The problem with traditional access control}{
		\begin{itemize}[<+->]
			\item Requires trusting the server completely
			\item No protection during transmission
			\item No way to verify integrity
			\item No way to establish trust between strangers
		\end{itemize}
	}
\end{frame}

\begin{frame}[c]{The magic of cryptography}
	\begin{center}
		\Large\textbf{Cryptography lets us achieve what seems impossible}
		\vspace{1cm}
		\begin{itemize}[<+->]
			\item Secure communication over insecure channels
			\item Verification without revealing secrets
			\item Proof of computation without redoing it
		\end{itemize}
	\end{center}
\end{frame}

\section{Hard Problems}

\begin{frame}{Hard problems}
	\begin{itemize}[<+->]
		\item Cryptography is largely about equating the security of a system to the
		      difficulty of solving a math problem that is thought to be computationally
		      very expensive.
		\item With cryptography, we get security systems that we can literally
		      mathematically prove as secure (under assumptions).
		\item Also, this allows for actual magic.
		      \begin{itemize}[<+->]
			      \item Alice and Bob meet for the first time in the same room as you.
			      \item You are listening to everything they are saying.
			      \item Can they exchange a secret without you learning it?
		      \end{itemize}
	\end{itemize}
\end{frame}

\begin{frame}{Time for actual magic}
	\bigimagewithcaption{dh.png}{}
\end{frame}

\begin{frame}{No known feasible computation}
	\begin{itemize}[<+->]
		\item The discrete logarithm problem:
		      \begin{itemize}
			      \item Given a finite cyclic group $G$, a generator $g \in G$, and an element
			            $h \in G$, find the integer $x$ such that $g^{x}=h$
		      \end{itemize}
		\item In more concrete terms:
		      \begin{itemize}
			      \item Let $p$ be a large prime and let $g$ be a generator of the multiplicative
			            group $\mathbb{Z}_{p}^{*}$ (all nonzero integers modulo $p$).

			      \item Given:
			            \begin{itemize}
				            \item $g \in \mathbb{Z}_{p}^{*}$, $h \in \mathbb{Z}_{p}^{*}$

				            \item Find $x \in \{0, 1, \ldots, p-2\}$ such that $g^{x} \equiv h \pmod
					                  {p}$
			            \end{itemize}

			      \item This problem is believed to be computationally hard when $p$ is large
			            and $g$ is a primitive root modulo $p$.
			            \begin{itemize}
				            \item ``Believed to be'' = we don't know of any way to do it that doesn't
				                  take forever, unless we have a strong, stable quantum computer (Shor's
				                  algorithm)
			            \end{itemize}
		      \end{itemize}
	\end{itemize}
\end{frame}

\begin{frame}{Time for more actual magic}
	\begin{columns}[c]
		\begin{column}{0.6\textwidth}
			\begin{itemize}[<+->]
				\item \textbf{Zero-knowledge proofs} allow you to prove that you know
				      a secret without revealing any information about it.
				\item They built ``zero-knowledge virtual machines'' where you can execute
				      an entire program that runs as a zero-knowledge proof.
				\item ZKP battleship game: server proves to the players that its
				      output to their battleship guesses is correct, without revealing any
				      additional information (e.g. ship location).
			\end{itemize}
		\end{column}
		\begin{column}{0.4\textwidth}
			\imagewithcaption{battleship.jpg}{Battleship board game. Source: Hasbro}
		\end{column}
	\end{columns}
\end{frame}

\begin{frame}{Hard problems}
	\begin{columns}[c]
		\begin{column}{0.5\textwidth}
			\textbf{Asymmetric Primitives}
			\begin{itemize}[<+->]
				\item Diffie-Hellman, RSA, ML-KEM, etc.
				\item ``Asymmetric'' because there is a ``public key'' and a ``private
				      key'' for each party.
				\item Algebraic, assume the hardness of mathematical problems (as seen
				      just now.)
			\end{itemize}
		\end{column}

		\begin{column}{0.5\textwidth}
			\textbf{Symmetric Primitives}
			\begin{itemize}[<+->]
				\item AES, SHA-2, ChaCha20, HMAC\ldots
				\item ``Symmetric'' because there is one secret key.
				\item Not algebraic but unstructured, but on their understood
				      resistance to $n$ years of cryptanalysis.
				\item Can act as substitutes for assumptions in security proofs!
				      \begin{itemize}
					      \item Example: hash function assumed to be a ``random oracle''
				      \end{itemize}
			\end{itemize}
		\end{column}
	\end{columns}
\end{frame}

\begin{frame}{Hard problems}
	\begin{itemize}[<+->]
		\item Hard computational problems are the cornerstone of modern cryptography.
		\item These are problems for which even the best algorithms wouldn't find a solution before the sun burns out.
		\item They provide the security foundation for cryptographic schemes.
		\item Without hard problems, most of our encryption systems would collapse.
	\end{itemize}
\end{frame}

\begin{frame}{The rise of computational complexity theory}
	\definitionbox{Computational Complexity Theory}{Complexity theory provides the mathematical framework to understand what makes problems ``hard''.}
	\begin{itemize}
		\item In the 1970s, rigorous study of hard problems led to computational complexity theory.
		\item This field has had dramatic impacts beyond cryptography:
		      \begin{itemize}
			      \item \textbf{Economics}: Computational complexity of finding Nash equilibria in game theory.
			      \item \textbf{Physics}: Simulating quantum many-body systems with exponential complexity.
			      \item \textbf{Biology}: Protein folding prediction and DNA sequence alignment algorithms.
		      \end{itemize}
	\end{itemize}
\end{frame}

\begin{frame}{Computational problems}
	\definitionbox{Computational Problem}{
		A question that can be answered by performing a computation.
		\begin{itemize}
			\item \textbf{Decision problems}: Questions with ``yes'' or ``no'' answers
			      \begin{itemize}
				      \item Example: ``Is 217 a prime number?''
			      \end{itemize}
			\item \textbf{Search problems}: Questions that require finding a specific value
			      \begin{itemize}
				      \item Example: ``How many instances of \textit{`i'}s appear in \textit{`incomprehensibilities'}?''
			      \end{itemize}
		\end{itemize}
	}
	\begin{itemize}[<+->]
		\item Computational problems form the foundation of theoretical computer science.
		\item Different types of problems require different algorithmic approaches.
		\item The difficulty of solving these problems is central to cryptography.
	\end{itemize}
\end{frame}

\begin{frame}{Computational hardness}
	\definitionbox{Computational Hardness}{
		The property of computational problems for which no algorithm exists that can solve the problem in a reasonable amount of time.
		\begin{itemize}
			\item Also called \textbf{intractable problems}.
			\item Hardness is independent of the computing device used.
			\item All standard computing models are equivalent in terms of what they can compute efficiently.
			\item \textbf{Exception}: Quantum computers for certain problems.
		\end{itemize}
	}
	\begin{itemize}[<+->]
		\item Hardness is a fundamental concept in computational complexity theory.
		\item Cryptography deliberately uses hard problems to create security.
		\item What's ``hard'' should remain hard regardless of hardware advances.
	\end{itemize}
\end{frame}

\begin{frame}{Measuring algorithm complexity}
	\begin{columns}[c]
		\begin{column}{0.4\textwidth}
			\begin{itemize}[<+->]
				\item To evaluate computational hardness, we need to measure an algorithm's running time.
				\item We typically use \textbf{asymptotic analysis} to express complexity.
				\item Common notation:
				      \begin{itemize}
					      \item $\mathcal{O}(n)$: Linear time.
					      \item $\mathcal{O}(n^2)$: Quadratic time.
					      \item $\mathcal{O}(2^n)$: Exponential time.
				      \end{itemize}
			\end{itemize}
		\end{column}
		\begin{column}{0.6\textwidth}
			\imagewithcaption{complexity.png}{Complexity classes growth. Source: Serious Cryptography}
		\end{column}
	\end{columns}
\end{frame}

\begin{frame}{Measuring algorithm complexity}
	\begin{itemize}
		\item To evaluate computational hardness, we need to measure an algorithm's running time.
		\item We typically use \textbf{asymptotic analysis} to express complexity.
		\item Common notation:
		      \begin{itemize}
			      \item $\mathcal{O}(n)$: Linear time.
			      \item $\mathcal{O}(n^2)$: Quadratic time.
			      \item $\mathcal{O}(2^n)$: Exponential time.
		      \end{itemize}
		\item We care about how the running time grows as the input size increases.
		\item \textbf{Example}: An algorithm that takes $n^2$ operations for input size $n$ becomes impractical as $n$ grows large.
	\end{itemize}
\end{frame}

\begin{frame}{Categorizing computational hardness}
	\begin{columns}[c]
		\begin{column}{0.5\textwidth}
			\textbf{Easy Problems}
			\begin{itemize}[<+->]
				\item Solvable in polynomial time.
				\item \textbf{Examples}: Sorting, searching.
				\item Running time: $\mathcal{O}(n^c)$ for some constant $c$
				\item Generally scales reasonably with input size.
				\item Class P (Polynomial time).
			\end{itemize}
		\end{column}
		\begin{column}{0.5\textwidth}
			\textbf{Hard Problems}
			\begin{itemize}[<+->]
				\item No known polynomial-time solution.
				\item \textbf{Example}: Factorizing product of two large primes.
				\item Running time: Often exponential, e.g., $\mathcal{O}(2^n)$
				\item Becomes impractical quickly as input grows.
				\item Includes NP-hard, NP-complete classes.
			\end{itemize}
		\end{column}
	\end{columns}
\end{frame}

\begin{frame}{Hard problems in practice}
	\begin{itemize}
		\item Public-key cryptography relies on specific hard problems:
		      \begin{itemize}
			      \item RSA: Integer factorization problem.
			      \item Diffie-Hellman: Discrete logarithm problem.
		      \end{itemize}
		\item Cryptography leverages these problems to maximize security assurance,
		\item The security of these schemes depends on the continued hardness of these problems.
	\end{itemize}
\end{frame}

\begin{frame}{Quantum vulnerability of hard problems}
	\begin{itemize}[<+->]
		\item The hard problems we rely on today (factoring, discrete logarithm) are vulnerable to quantum computers.
		\item Shor's algorithm (1994) can efficiently solve both problems on a sufficiently powerful quantum computer.
		\item This has motivated the search for \textbf{``post-quantum''} hard problems:
		      \begin{itemize}[<+->]
			      \item Lattice-based cryptography (e.g., ML-KEM, formerly CRYSTALS-Kyber).
			      \item Hash-based cryptography.
			      \item Code-based cryptography.
			      \item Multivariate cryptography.
			      \item Isogeny-based cryptography.
		      \end{itemize}
		\item NIST is currently standardizing post-quantum cryptographic algorithms to replace our vulnerable systems.
	\end{itemize}
\end{frame}

\begin{frame}{What is NIST?}
	\begin{columns}[c]
		\begin{column}{0.6\textwidth}
			\begin{itemize}[<+->]
				\item \textbf{NIST} stands for the National Institute of Standards and Technology.
				\item It's a U.S. government agency that develops technology standards.
				\item In cryptography, NIST:
				      \begin{itemize}
					      \item Sets security standards used worldwide.
					      \item Evaluates and approves cryptographic algorithms.
					      \item Currently leading the standardization of post-quantum cryptography.
				      \end{itemize}
				\item When NIST standardizes an algorithm, it often becomes the global industry standard.
			\end{itemize}
		\end{column}
		\begin{column}{0.4\textwidth}
			\imagewithcaption{nist-peanut.png}{NIST's ``Standard Reference Peanut Butter'', available for only \$1,217 USD!}
		\end{column}
	\end{columns}
\end{frame}

\begin{frame}{Funny things standardized by NIST}
	\begin{itemize}[<+->]
		\item \textbf{Standard Reference Peanut Butter}: for calibrating food testing equipment.
		\item \textbf{The ``Odor Unit''}: for standardizing measurements of smell intensity in environmental monitoring.
		\item \textbf{The Standard Banana Equivalent Dose (BED)}: for comparing radiation exposure levels to the natural radiation in a banana.
		\item \textbf{Toilet Paper Testing}: for measuring strength, absorbency, and softness of toilet paper products.
	\end{itemize}
\end{frame}

\begin{frame}{Cryptographic algorithms standardized by NIST}
	\begin{itemize}[<+->]
		\item \textbf{AES (Advanced Encryption Standard)}: Selected in 2001 to replace DES, now the worldwide standard for symmetric encryption.
		\item \textbf{SHA-2 and SHA-3 (Secure Hash Algorithms)}: Cryptographic hash functions used for digital signatures and data integrity.
		\item \textbf{DSA and ECDSA}: Digital Signature Algorithms based on the discrete logarithm problem.
		\item \textbf{Triple DES}: An interim standard before AES that enhanced the security of the original DES.
		\item \textbf{ML-KEM and ML-DSA}: Recently standardized post-quantum public-key cryptography and signature schemes.
	\end{itemize}
\end{frame}

\begin{frame}{Why hard problems matter}
	\begin{itemize}
		\item Hard problems \textbf{create asymmetry between legitimate users and attackers}.
		\item Easy in one direction, difficult in the reverse.
		\item Example: Easy to multiply large primes, hard to factor the product.
		\item This asymmetry is what enables secure communication!
	\end{itemize}
\end{frame}

\begin{frame}{What are complexity classes?}
	\definitionbox{Complexity Class}{
		A group of computational problems that share similar resource requirements (time, memory, etc.).
	}
	\begin{itemize}[<+->]
		\item \textbf{Example}: All problems solvable in $\mathcal{O}(n^2)$ time form one class.
		\item Different classes represent different levels of computational difficulty.
		\item Understanding these classes helps us categorize cryptographic problems.
	\end{itemize}
\end{frame}

\begin{frame}{TIME complexity classes}
	\begin{itemize}[<+->]
		\item \textbf{TIME}$(f(n))$ = class of problems solvable in time $\mathcal{O}(f(n))$
		\item Examples:
		      \begin{itemize}
			      \item \textbf{TIME}$(n^2)$ = problems solvable in $\mathcal{O}(n^2)$ time
			      \item \textbf{TIME}$(n^3)$ = problems solvable in $\mathcal{O}(n^3)$ time
			      \item \textbf{TIME}$(2^n)$ = problems solvable in $\mathcal{O}(2^n)$ time
		      \end{itemize}
		\item \textbf{Key insight}: If you can solve a problem in $\mathcal{O}(n^2)$ time, you can also solve it in $\mathcal{O}(n^3)$ time.
		\item Therefore: \textbf{TIME}$(n^2) \subseteq$ \textbf{TIME}$(n^3) \subseteq$ \textbf{TIME}$(n^4) \subseteq \ldots$
	\end{itemize}
\end{frame}

\begin{frame}{The class P (Polynomial time)}
	\definitionbox{Class P}{
		The union of all \textbf{TIME}$(n^k)$ classes for all constants $k$.
		\begin{itemize}
			\item P = \textbf{TIME}$(n) \cup$ \textbf{TIME}$(n^2) \cup$ \textbf{TIME}$(n^3) \cup \ldots$
			\item Contains all problems solvable in polynomial time.
			\item Generally considered ``efficiently solvable''.
		\end{itemize}
	}
	\begin{itemize}[<+->]
		\item Most practical algorithms we use daily are in class P.
		\item Examples: Sorting, searching, basic arithmetic.
		\item Cryptography often relies on problems \textbf{not} in P!
	\end{itemize}
\end{frame}

\begin{frame}{SPACE complexity classes}
	\begin{itemize}[<+->]
		\item Time isn't everything—memory usage matters too!
		\item A single memory access can be orders of magnitude slower than CPU operations.
		\item \textbf{SPACE}$(f(n))$ = class of problems solvable using $O(f(n))$ bits of memory
		\item Examples:
		      \begin{itemize}
			      \item \textbf{SPACE}$(n)$ = problems using $\mathcal{O}(n)$ memory
			      \item \textbf{SPACE}$(n^2)$ = problems using $\mathcal{O}(n^2)$ memory
		      \end{itemize}
		\item \textbf{PSPACE} = union of all \textbf{SPACE}$(n^k)$ for constants $k$
	\end{itemize}
\end{frame}

\begin{frame}{Relationship between TIME and SPACE}
	\begin{itemize}[<+->]
		\item \textbf{Key insight}: Any algorithm running in time $f(n)$ uses at most $f(n)$ memory.
		\item Why? You can write at most one bit per time unit.
		\item Therefore: \textbf{TIME}$(f(n)) \subseteq$ \textbf{SPACE}$(f(n))$
		\item This gives us: $P \subseteq PSPACE$
		\item \textbf{Important}: Low memory doesn't guarantee fast execution!
		      \begin{itemize}
			      \item Example: Brute-force key search uses little memory but takes forever.
		      \end{itemize}
	\end{itemize}
\end{frame}

\begin{frame}{The class NP (Nondeterministic Polynomial time)}
	\definitionbox{Class NP}{
		The class of decision problems for which you can \textbf{verify} a solution in polynomial time, even if finding the solution is hard.
	}
	\begin{itemize}[<+->]
		\item \textbf{Key insight}: Easy to check, hard to find!
		\item Given a potential solution, you can run a polynomial-time algorithm to verify if it's correct.
		\item You don't need to find the solution efficiently—only verify it efficiently.
		\item \textbf{Relationship}: $P \subseteq NP$ (if you can solve it quickly, you can certainly verify it quickly)
	\end{itemize}
\end{frame}

\begin{frame}{NP: A cryptographic example}
	\textbf{Problem}: Given plaintext $P$ and ciphertext $C$, does there exist a key $K$ such that $C = E(K, P)$?
	\begin{itemize}[<+->]
		\item \textbf{Finding the solution}: Could take exponential time (brute-force key search)
		\item \textbf{Verifying a candidate solution}: Given a potential key $K_0$:
		      \begin{enumerate}
			      \item Compute $E(K_0, P)$
			      \item Check if $E(K_0, P) = C$
			      \item Return ``yes'' if they match, ``no'' otherwise
		      \end{enumerate}
		\item This verification runs in polynomial time!
		\item Therefore, this key recovery problem is in NP.
	\end{itemize}
\end{frame}

\begin{frame}{What's NOT in NP?}
	\begin{itemize}[<+->]
		\item \textbf{Known-ciphertext attacks}: You only have $E(K, P)$ values for random unknown plaintexts $P$.
		      \begin{itemize}
			      \item How do you verify if candidate key $K_0$ is correct?
			      \item You don't know what the plaintexts should be!
			      \item Can't express this as a decision problem with efficient verification.
		      \end{itemize}
		\item \textbf{Proving absence of solutions}: ``Does there exist NO solution to this problem?''
		      \begin{itemize}
			      \item To verify ``no solution exists,'' you might need to check all possible inputs.
			      \item If there are exponentially many inputs, this takes exponential time.
			      \item Therefore, proving non-existence is generally not in NP.
		      \end{itemize}
	\end{itemize}
\end{frame}

\begin{frame}{NP-complete problems}
	\definitionbox{NP-Complete Problems}{
		The hardest decision problems in the class NP.
		\begin{itemize}
			\item No known polynomial-time algorithms exist for worst-case instances.
			\item If any NP-complete problem can be solved efficiently, then \textbf{all} problems in NP can be solved efficiently.
		\end{itemize}
	}
	\begin{itemize}[<+->]
		\item Discovered in the 1970s during the development of complexity theory.
		\item \textbf{Remarkable discovery}: All NP-complete problems are fundamentally equally hard!
		\item Examples: Boolean satisfiability (SAT), traveling salesman problem, graph coloring.
	\end{itemize}
\end{frame}

\begin{frame}{Why are NP-complete problems equally hard?}
	\begin{itemize}[<+->]
		\item \textbf{Key insight}: You can \textit{reduce} any NP-complete problem to any other NP-complete problem.
		\item \textbf{Reduction}: Transform one problem into another in polynomial time.
		      \begin{itemize}
			      \item If you can solve problem B efficiently, you can solve problem A efficiently too.
		      \end{itemize}
		\item \textbf{Mathematical equivalence}: Different NP-complete problems may look completely different but are fundamentally the same from a computational perspective.
		\item \textbf{Consequence}: Solving any single NP-complete problem efficiently would solve \textit{all} problems in NP efficiently!
		      \begin{itemize}
			      \item This would prove that P = NP (one of the biggest open questions in computer science).
		      \end{itemize}
	\end{itemize}
\end{frame}

\begin{frame}{The remarkable equivalence of NP-complete problems}
	\begin{center}
		\Large\textbf{These problems look completely different\ldots}
		\vspace{0.5cm}

		\begin{columns}[c]
			\begin{column}{0.33\textwidth}
				\textbf{Boolean Logic}\\
				\small Can you set variables to make this formula true?\\
				$(x_1 \lor \neg x_2) \land (x_2 \lor x_3) \land \ldots$
			\end{column}
			\begin{column}{0.33\textwidth}
				\textbf{Travel Planning}\\
				\small What's the shortest route visiting all cities exactly once?
			\end{column}
			\begin{column}{0.33\textwidth}
				\textbf{Sudoku Puzzles}\\
				\small Can you fill this 9 \times 9 grid following the rules?
			\end{column}
		\end{columns}
		\vspace{0.5cm}

		\Large\textbf{\ldots but they're computationally identical!}
	\end{center}
\end{frame}

\begin{frame}{Concrete examples of equivalent problems}
	\begin{itemize}[<+->]
		\item \textbf{3-SAT} (Boolean satisfiability): Given a logical formula, can you set the variables to make it true?
		      \begin{itemize}
			      \item Example: $(x_1 \lor \neg x_2 \lor x_3) \land (\neg x_1 \lor x_2 \lor \neg x_3) \land \ldots$
		      \end{itemize}
		\item \textbf{Traveling Salesman Problem}: Given cities and distances, what's the shortest route visiting each city exactly once?
		      \begin{itemize}
			      \item Looks like a geometry/optimization problem!
		      \end{itemize}
		\item \textbf{Graph Coloring}: Can you color a graph's vertices with $k$ colors so no adjacent vertices share a color?
		      \begin{itemize}
			      \item Looks like a combinatorial puzzle!
		      \end{itemize}
		\item \textbf{Subset Sum}: Given a set of integers, is there a subset that sums to exactly $k$?
		      \begin{itemize}
			      \item Looks like an arithmetic problem!
		      \end{itemize}
	\end{itemize}
\end{frame}

\begin{frame}{The magic of reductions}
	\definitionbox{Problem Reduction}{
		A polynomial-time transformation that converts any instance of problem A into an equivalent instance of problem B.
	}
	\begin{itemize}[<+->]
		\item You can transform \textbf{any} Sudoku puzzle into a Boolean logic formula!
		      \begin{itemize}
			      \item The Sudoku has a solution $\Leftrightarrow$ the formula is satisfiable
		      \end{itemize}
		\item You can transform \textbf{any} traveling salesman instance into a graph coloring problem!
		\item You can transform \textbf{any} Boolean formula into a subset sum problem!
		\item These transformations preserve the ``yes/no'' answer and run in polynomial time.
		\item \textbf{Mind-blowing consequence}: Solve Sudoku efficiently = solve all of theoretical computer science!
	\end{itemize}
\end{frame}

\begin{frame}{Real-world impact of this equivalence}
	\begin{itemize}[<+->]
		\item \textbf{Good news}: Any algorithmic breakthrough on one NP-complete problem immediately applies to thousands of others!
		      \begin{itemize}
			      \item Better SAT solvers $\Rightarrow$ better protein folding, circuit design, AI planning\ldots
		      \end{itemize}
		\item \textbf{Sobering reality}: 50+ years of computer science research suggests these problems are fundamentally hard.
		      \begin{itemize}
			      \item Despite massive incentives (millions in prize money, practical applications worth billions)
		      \end{itemize}
		\item \textbf{Cryptographic relevance}: We rely on NP-complete problems being hard for certain security models.
		      \begin{itemize}
			      \item Though most practical cryptography uses different hard problems (factoring, discrete log)
		      \end{itemize}
		\item \textbf{Universal truth}: The computational universe has these deep, hidden connections that unite seemingly unrelated problems.
	\end{itemize}
\end{frame}

\begin{frame}{Fun fact: Nintendo games are NP-hard!}
	\begin{columns}[c]
		\begin{column}{1\textwidth}
			\begin{itemize}[<+->]
				\item \textbf{Games proven NP-hard\footnote{\url{https://appliedcryptography.page/paper/\#nintendo-hard}}}:
				      \begin{itemize}
					      \item Super Mario Bros. 1-3, The Lost Levels, Super Mario World
					      \item Donkey Kong Country 1-3
					      \item All classic Legend of Zelda games
					      \item All classic Metroid games
					      \item All classic Pokémon role-playing games
				      \end{itemize}
				\item \textbf{The catch}: ``Generalized versions'' with arbitrarily large levels.
				      \begin{itemize}
					      \item Real Nintendo levels are designed to be solvable by humans.
					      \item But the \textbf{mathematical structure} of these games is inherently complex.
				      \end{itemize}
				\item \textbf{Cool insight}: Video games naturally encode complex computational problems!
			\end{itemize}
		\end{column}
	\end{columns}
\end{frame}

\begin{frame}{The P vs. NP Problem}
	\definitionbox{The P vs. NP Problem}{
		One of the most important unsolved problems in computer science and mathematics.
		\begin{itemize}
			\item \textbf{Question}: Does P = NP?
			\item \textbf{Translation}: Are there problems that are easy to verify but fundamentally hard to solve?
		\end{itemize}
	}
	\begin{itemize}[<+->]
		\item If you could solve \textbf{any} NP-complete problem in polynomial time, then you could solve \textbf{all} NP problems in polynomial time.
		\item This would mean P = NP.
		\item \textbf{Intuition says}: Surely some problems are easy to check but hard to find!
		\item \textbf{Example}: Brute-force key recovery seems inherently exponential-time\ldots
		\item \textbf{Reality}: No one has proved this mathematically!
	\end{itemize}
\end{frame}

\begin{frame}{The million-dollar question}
	\begin{itemize}[<+->]
		\item The \textbf{Clay Mathematics Institute} offers \$1,000,000 for solving P vs. NP.
		\item One of seven ``Millennium Prize Problems''.
		\item Renowned complexity theorist Scott Aaronson called it \textit{``one of the deepest questions that human beings have ever asked''}.
		\item \textbf{To win}: Prove either $P = NP$ or $P \neq NP$.
		\item Over 50 years of research, no solution yet!
	\end{itemize}
\end{frame}

\begin{frame}{What if P = NP?}
	\begin{center}
		\Large\textbf{The cryptographic apocalypse scenario}
	\end{center}
	\begin{itemize}[<+->]
		\item If P = NP, then \textbf{any easily checked solution would also be easy to find}.
		\item \textbf{Symmetric cryptography} would be completely broken:
		      \begin{itemize}
			      \item Key recovery becomes polynomial-time.
			      \item AES, ChaCha20, all symmetric ciphers become useless.
		      \end{itemize}
		\item \textbf{Hash functions} would be invertible in polynomial time:
		      \begin{itemize}
			      \item Finding preimages becomes easy.
			      \item Digital signatures, password storage, all broken.
		      \end{itemize}
		\item \textbf{All of modern cryptography} would collapse overnight!
		\item \textbf{But also}: We could solve protein folding, optimize supply chains perfectly, solve climate modeling\ldots
	\end{itemize}
\end{frame}

\begin{frame}{Why we don't panic}
	\begin{itemize}[<+->]
		\item \textbf{Overwhelming consensus}: Most complexity theorists believe $P \neq NP$.
		\item \textbf{Intuitive reasoning}: Problems that look hard actually \textbf{are} hard.
		\item \textbf{The structure of reality}: Easy-to-verify but hard-to-solve problems seem fundamental to the universe.
		\item \textbf{50+ years of evidence}: Despite massive incentives, no polynomial-time algorithms found for NP-complete problems.
		\item \textbf{Current belief}: P is a strict subset of NP, with NP-complete problems outside P.
	\end{itemize}
	\definitionbox{The Challenge}{
		\begin{itemize}
			\item \textbf{Proving $P = NP$}: Need only one polynomial-time algorithm for one NP-complete problem
			\item \textbf{Proving $P \neq NP$}: Must prove no such algorithm can \textbf{ever} exist—much harder!
		\end{itemize}
	}
\end{frame}

\begin{frame}{Why NP-complete problems don't work for cryptography}
	\begin{itemize}[<+->]
		\item \textbf{Tempting idea}: Base cryptography on NP-complete problems for provable security!
		\item \textbf{The dream}: Prove that breaking some cipher is NP-hard.
		      \begin{itemize}
			      \item Security would be guaranteed as long as $P \neq NP$.
		      \end{itemize}
		\item \textbf{Reality is disappointing}: NP-complete problems are hard in the \textbf{worst case}, not the \textbf{average case}
		      \begin{itemize}
			      \item The structure that makes them hard can make specific instances easy.
			      \item Cryptography needs problems that are hard for \textbf{random} instances.
		      \end{itemize}
		\item \textbf{What we actually use}: Problems that are probably \textbf{not} NP-hard.
		      \begin{itemize}
			      \item Factoring, discrete logarithm, lattice problems.
			      \item Believed hard on average, but not proven NP-complete.
		      \end{itemize}
	\end{itemize}
\end{frame}

\begin{frame}{NP-complete vs. NP-hard}
	\begin{columns}[c]
		\begin{column}{0.5\textwidth}
			\textbf{NP-Complete Problems}
			\begin{itemize}[<+->]
				\item Must be decision problems (yes/no answers)
				\item You can verify solutions in polynomial time
				\item \textbf{Examples}: 3-SAT, graph coloring, subset sum
				\item The ``sweet spot'' of hardness
			\end{itemize}
		\end{column}
		\begin{column}{0.5\textwidth}
			\textbf{NP-Hard Problems}
			\begin{itemize}[<+->]
				\item Can be any type of problem (optimization, etc.)
				\item May not have polynomial-time verification
				\item \textbf{Examples}: Traveling salesman optimization, halting problem
				\item Can be even harder than NP-complete!
			\end{itemize}
		\end{column}
	\end{columns}
\end{frame}

\begin{frame}{Average-case vs. worst-case hardness}
	\begin{columns}[c]
		\begin{column}{0.5\textwidth}
			\textbf{Worst-case hardness (NP-complete)}
			\begin{itemize}[<+->]
				\item Some instances of the problem are very hard.
				\item Other instances might be easy.
				\item \textbf{Example}: 3-SAT has hard instances, but also trivial ones.
				\item Not suitable for cryptography.
			\end{itemize}
		\end{column}
		\begin{column}{0.5\textwidth}
			\textbf{Average-case hardness (Crypto)}
			\begin{itemize}[<+->]
				\item Random instances are typically hard.
				\item Few (if any) easy instances.
				\item \textbf{Example}: Factoring random large integers.
				\item Perfect for cryptographic applications.
			\end{itemize}
		\end{column}
	\end{columns}
	\vspace{0.5cm}
	\definitionbox{Hard Problems for Cryptography}{
		We need problems where almost every instance is hard, not just the worst ones.
	}
\end{frame}

\section{Diffie-Hellman}

\begin{frame}{Time for actual magic}
	\bigimagewithcaption{dh.png}{}
\end{frame}

\begin{frame}{The key exchange problem}
	\begin{itemize}[<+->]
		\item Alice and Bob want to communicate securely over the internet.
		\item They've never met before and share no secrets.
		\item How can they establish a shared secret key for encryption?
		\item Traditional approach: meet in person, exchange keys physically.
		\item \textbf{Problem}: This doesn't scale for the internet!
	\end{itemize}
	\definitionbox{The Challenge}{
		Create a shared secret between two parties who have never communicated before, even when an eavesdropper can see everything they send to each other.
	}
\end{frame}

\begin{frame}{The magic of Diffie-Hellman}
	\begin{itemize}[<+->]
		\item Whitfield Diffie and Martin Hellman solved this ``impossible'' problem.
		\item Their solution came one year \textbf{before} RSA (1977).
		\item Uses the \textbf{discrete logarithm problem} as its foundation.
		\item Allows two strangers to create a shared secret in public!
	\end{itemize}
	\vspace{0.5cm}
	\begin{center}
		\Large\textbf{This was the birth of modern cryptography}
	\end{center}
\end{frame}

\begin{frame}{What makes discrete logarithm hard?}
	\begin{itemize}[<+->]
		\item Remember: we need problems that are easy in one direction, hard in reverse.
		\item \textbf{Easy direction}: Given $g$ and $x$, compute $g^x \bmod p$
		      \begin{itemize}
			      \item Example: $2^{10} \bmod 17 = 1024 \bmod 17 = 4$
		      \end{itemize}
		\item \textbf{Hard direction}: Given $g$, $p$, and $g^x \bmod p$, find $x$
		      \begin{itemize}
			      \item Example: Given $g = 2$, $p = 17$, and result $= 4$, find $x = 10$
		      \end{itemize}
		\item For small numbers, this is easy. For huge numbers (thousands of bits), it's computationally infeasible!
	\end{itemize}
\end{frame}

\begin{frame}{A simple example}
	Let's work with small numbers to see the pattern:
	\vspace{0.5cm}
	\begin{itemize}[<+->]
		\item Let $p = 17$ (a prime) and $g = 2$ (a generator)
		\item \textbf{Computing powers is easy}:
		      \begin{itemize}
			      \item $2^1 \bmod 17 = 2$
			      \item $2^2 \bmod 17 = 4$
			      \item $2^3 \bmod 17 = 8$
			      \item $2^4 \bmod 17 = 16$
			      \item $2^5 \bmod 17 = 15$
		      \end{itemize}
		\item \textbf{Finding the exponent is harder}:
		      \begin{itemize}
			      \item Given result $15$, can you quickly find that the exponent was $5$?
			      \item With small numbers: yes, by trying all possibilities
			      \item With 2048-bit numbers: practically impossible!
		      \end{itemize}
	\end{itemize}
\end{frame}

\begin{frame}{Mathematical groups: the foundation}
	\definitionbox{What is a Mathematical Group?}{
		A set of elements with an operation that follows specific rules.
		\begin{itemize}
			\item Think of it as a \textbf{mathematical playground} with consistent rules.
			\item For cryptography, we use $\mathbb{Z}_p^*$: numbers $\{1, 2, 3, \ldots, p-1\}$ with multiplication mod $p$.
		\end{itemize}
	}
	\begin{itemize}[<+->]
		\item \textbf{Example}: $\mathbb{Z}_5^* = \{1, 2, 3, 4\}$ with multiplication mod 5
		      \begin{itemize}
			      \item $3 \times 4 = 12 \bmod 5 = 2$
			      \item $2 \times 3 = 6 \bmod 5 = 1$
		      \end{itemize}
		\item The ``rules'' ensure the math works consistently for cryptography.
	\end{itemize}
\end{frame}

\begin{frame}{Group rules (simplified)}
	For our cryptographic group $\mathbb{Z}_p^*$, these rules always hold:
	\begin{itemize}[<+->]
		\item \textbf{Closure}: Multiplying any two elements gives another element in the group
		      \begin{itemize}
			      \item In $\mathbb{Z}_5^*$: $2 \times 3 = 1$ (still in the group!)
		      \end{itemize}
		\item \textbf{Identity}: There's a special element (1) that doesn't change others
		      \begin{itemize}
			      \item $1 \times 4 = 4$, $1 \times 2 = 2$, etc.
		      \end{itemize}
		\item \textbf{Inverses}: Every element has a ``partner'' that multiplies to 1
		      \begin{itemize}
			      \item In $\mathbb{Z}_5^*$: $2 \times 3 = 1$, so 2 and 3 are inverses
		      \end{itemize}
		\item \textbf{Associativity}: $(a \times b) \times c = a \times (b \times c)$
	\end{itemize}
	\vspace{0.5cm}
	\textbf{Why care?} These rules guarantee that our cryptographic operations will behave predictably!
\end{frame}

\begin{frame}{Generators: allow exponentiation to target every element in the group}
	\definitionbox{Generator}{
		An element $g$ whose powers $g^1, g^2, g^3, \ldots$ produce \textbf{every element in the group}.
	}
	\begin{itemize}[<+->]
		\item In $\mathbb{Z}_5^*$, let's try $g = 2$:
		      \begin{itemize}
			      \item $2^1 \bmod 5 = 2$
			      \item $2^2 \bmod 5 = 4$
			      \item $2^3 \bmod 5 = 3$
			      \item $2^4 \bmod 5 = 1$
		      \end{itemize}
		\item We got $\{2, 4, 3, 1\}$ - that's all elements! So $g = 2$ is a generator.
		\item \textbf{Generators are crucial}: They let us express every group element as a power of $g$.
	\end{itemize}
\end{frame}

\begin{frame}{The discrete logarithm problem (DLP)}
	\definitionbox{Discrete Logarithm Problem}{
		Given $g$, $p$, and $h = g^x \bmod p$, find the secret exponent $x$.
	}
	\begin{itemize}[<+->]
		\item \textbf{``Discrete''} because we work with integers, not real numbers
		\item \textbf{``Logarithm''} because we're finding the exponent (like $\log_2(8) = 3$)
		\item \textbf{Example}: Given $g = 2$, $p = 17$, $h = 8$, find $x$ such that $2^x \equiv 8 \pmod{17}$
		      \begin{itemize}
			      \item Answer: $x = 3$ (since $2^3 = 8$)
			      \item Easy with small numbers, hard with large ones!
		      \end{itemize}
		\item For cryptographic-sized numbers (2048+ bits), no efficient algorithm is known.
	\end{itemize}
\end{frame}

\begin{frame}{Why does $p$ need to be prime?}
	\begin{itemize}[<+->]
		\item \textbf{Critical requirement}: In Diffie-Hellman, $p$ must be a prime number.
		\item \textbf{What happens with composite $p$?} Let's see with an example:
		      \begin{itemize}
			      \item Suppose $p = 15 = 3 \times 5$ (composite)
			      \item The group $\mathbb{Z}_{15}^*$ has only 8 elements: $\{1, 2, 4, 7, 8, 11, 13, 14\}$
			      \item Missing: $\{3, 5, 6, 9, 10, 12\}$ (share factors with 15)
		      \end{itemize}
		\item \textbf{With prime $p$}: The group $\mathbb{Z}_p^*$ has exactly $p-1$ elements
		      \begin{itemize}
			      \item Every non-zero element has an inverse
			      \item Clean mathematical structure
		      \end{itemize}
		\item \textbf{Small subgroup attacks}: When $p-1$ has small factors, attackers can exploit them!
		      \begin{itemize}
			      \item Example: If $p-1 = 2 \times q$ where $q$ is small
			      \item Attacker can solve DLP in the small subgroup of size $q$
			      \item This dramatically reduces security!
		      \end{itemize}
	\end{itemize}
\end{frame}

\begin{frame}{Why are these elements ``missing''?}
	\begin{itemize}[<+->]
		\item \textbf{The key requirement}: Elements in $\mathbb{Z}_n^*$ must be \textbf{coprime} to $n$
		      \begin{itemize}
			      \item Coprime means $\func{gcd}{a, n} = 1$ (no common factors)
			      \item Only coprime elements have multiplicative inverses mod $n$
		      \end{itemize}
		\item \textbf{Let's check the missing elements} (with $n = 15 = 3 \times 5$):
		      \begin{itemize}
			      \item $\func{gcd}{3, 15} = 3$ (shares factor 3)
			      \item $\func{gcd}{5, 15} = 5$ (shares factor 5)
			      \item $\func{gcd}{6, 15} = 3$ (because $6 = 2 \times 3$)
			      \item $\func{gcd}{9, 15} = 3$ (because $9 = 3^2$)
			      \item $\func{gcd}{10, 15} = 5$ (because $10 = 2 \times 5$)
			      \item $\func{gcd}{12, 15} = 3$ (because $12 = 4 \times 3$)
		      \end{itemize}
		\item \textbf{Why no inverse?} If $\func{gcd}{a, n} > 1$, then $ax \equiv 1 \pmod{n}$ has no solution
		      \begin{itemize}
			      \item Example: $3x \equiv 1 \pmod{15}$ is impossible because $3x$ is always divisible by 3
			      \item But $1 \bmod 15$ leaves remainder 1, which is not divisible by 3!
		      \end{itemize}
	\end{itemize}
\end{frame}

\begin{frame}{Small subgroup attack: a concrete example}
	\textbf{Scenario}: Suppose we use a prime $p = 23$ (small for illustration)
	\begin{itemize}[<+->]
		\item Note that $p - 1 = 22 = 2 \times 11$
		\item The group $\mathbb{Z}_{23}^*$ has subgroups of order 2, 11, and 22
		\item \textbf{The attack}: Force the victim to work in a small subgroup!
	\end{itemize}
	\vspace{0.5cm}
	\textbf{Attack execution}:
	\begin{enumerate}[<+->]
		\item \textbf{Attacker finds} an element of order 11: $g = 2$ has order 11 in $\mathbb{Z}_{23}^*$
		      \begin{itemize}
			      \item Check: $2^{11} \equiv 1 \pmod{23}$ (so order divides 11)
			      \item Since 11 is prime, the order is exactly 11
		      \end{itemize}
		\item \textbf{Victim picks} secret $x$ and sends $g^x = 2^x \bmod 23$
		\item \textbf{Problem}: There are only 11 possible values for $g^x$!
		      \begin{itemize}
			      \item Instead of checking all $\approx 22$ possible exponents
			      \item Attacker only needs to check 11 values
			      \item Security reduced by 50\%!
		      \end{itemize}
	\end{enumerate}
\end{frame}

\begin{frame}{Prime order groups: immunity to small subgroup attacks}
	\textbf{Question}: Can we have small subgroup attacks in a prime order group?
	\begin{itemize}[<+->]
		\item \textbf{Short answer}: No! Prime order groups have no small subgroups.
		\item \textbf{Lagrange's theorem}: In a group of order $n$, every subgroup's order divides $n$.
		\item \textbf{Prime order consequence}: If group order = prime $q$, the only divisors are 1 and $q$.
		      \begin{itemize}
			      \item Subgroup of order 1 = $\{1\}$ (just the identity)
			      \item Subgroup of order $q$ = the entire group
			      \item \textbf{No other subgroups exist!}
		      \end{itemize}
	\end{itemize}
	\vspace{0.5cm}
	\textbf{Example}: Consider a group of prime order $q = 11$
	\begin{itemize}[<+->]
		\item \textbf{Every non-identity element} generates the whole group!
		\item Pick any $g \neq 1$: the powers $g^1, g^2, \ldots, g^{10}, g^{11} = 1$ give all 11 elements
		\item No element can generate a ``small'' subgroup—it's all or nothing
	\end{itemize}
\end{frame}

\begin{frame}{$\mathbb{Z}_p^*$ vs. prime order subgroups}
	\textbf{Important distinction}: $\mathbb{Z}_p^*$ is usually \textbf{not} a prime order group!
	\begin{itemize}[<+->]
		\item For prime $p$, the group $\mathbb{Z}_p^*$ has order $p - 1$ (which is even for $p > 2$)
		\item Example: $\mathbb{Z}_{23}^*$ has order 22 = $2 \times 11$
		      \begin{itemize}
			      \item This allows subgroups of order 1, 2, 11, and 22
			      \item Elements of order 2 or 11 generate small subgroups!
		      \end{itemize}
	\end{itemize}
	\vspace{0.5cm}
	\textbf{Solution}: Work in a prime-order subgroup of $\mathbb{Z}_p^*$
	\begin{itemize}[<+->]
		\item Choose a safe prime: $p = 2q + 1$ where $q$ is also prime
		\item The subgroup of order $q$ has no small subgroups
		\item \textbf{Modern practice}: Use a generator of this order-$q$ subgroup
		\item \textbf{Result}: Automatic immunity to small subgroup attacks!
	\end{itemize}
\end{frame}

\begin{frame}{Small subgroup attack: devastating in practice}
	\textbf{Real-world example}: LogJam attack (2015)\footnote{Adrian et al., Imperfect Forward Secrecy: How Diffie-Hellman Fails in Practice, CCS 2015}
	\begin{itemize}[<+->]
		\item Many servers used 512-bit DH with \textbf{weak primes}
		\item Primes chosen where $p - 1 = 2 \times q$ with many small factors in $q$
		\item \textbf{Result}: Nation-state attackers could break ``secure'' connections!
	\end{itemize}
	\vspace{0.5cm}
	\textbf{Defense}: Use safe primes
	\begin{itemize}[<+->]
		\item A \textbf{safe prime} is a prime $p$ where $p = 2q + 1$ and $q$ is also prime
		\item Example: $p = 23 = 2 \times 11 + 1$ (both 23 and 11 are prime)
		\item This ensures the only subgroups have order 1, 2, $q$, or $2q$
		\item The subgroup of order $q$ is large, preventing small subgroup attacks
		\item Modern implementations require safe primes or equivalent protections
	\end{itemize}
\end{frame}

\begin{frame}{DLP vs. factoring: equally hard}
	\begin{columns}[c]
		\begin{column}{0.5\textwidth}
			\textbf{Factoring Problem}
			\begin{itemize}[<+->]
				\item Given $N = p \times q$, find $p$ and $q$
				\item Used in RSA (1977)
				\item Well-known, intuitive
			\end{itemize}
		\end{column}
		\begin{column}{0.5\textwidth}
			\textbf{Discrete Logarithm}
			\begin{itemize}[<+->]
				\item Given $g^x \bmod p$, find $x$
				\item Used in Diffie-Hellman (1976)
				\item Less intuitive, more mathematical
			\end{itemize}
		\end{column}
	\end{columns}
	\vspace{1cm}
	\begin{itemize}[<+->]
		\item \textbf{Security equivalence}: $n$-bit factoring $\approx$ $n$-bit discrete logarithm
		\item Both are vulnerable to Shor's quantum algorithm
		\item Both are \textbf{not} known to be NP-hard
		\item Algorithms for both problems share similar techniques
	\end{itemize}
\end{frame}

\begin{frame}{Diffie-Hellman: the mathematical version}
	\textbf{Setup}: Alice and Bob agree on public values $g$ (generator) and $p$ (large prime)
	\vspace{0.5cm}
	\begin{enumerate}[<+->]
		\item \textbf{Alice}: Chooses secret $a$, computes $A = g^a \bmod p$, sends $A$ to Bob
		\item \textbf{Bob}: Chooses secret $b$, computes $B = g^b \bmod p$, sends $B$ to Alice
		\item \textbf{Alice}: Computes shared secret $S = B^a \bmod p = (g^b)^a \bmod p = g^{ab} \bmod p$
		\item \textbf{Bob}: Computes shared secret $S = A^b \bmod p = (g^a)^b \bmod p = g^{ab} \bmod p$
	\end{enumerate}
	\vspace{0.5cm}
	\textbf{Result}: Alice and Bob both have $S = g^{ab} \bmod p$ without ever sharing $a$ or $b$!
\end{frame}

\begin{frame}{Diffie-Hellman example with small numbers}
	\textbf{Public parameters}: $g = 2$, $p = 17$
	\vspace{0.5cm}
	\begin{enumerate}[<+->]
		\item \textbf{Alice}: Picks secret $a = 6$
		      \begin{itemize}
			      \item Computes $A = 2^6 \bmod 17 = 64 \bmod 17 = 13$
			      \item Sends $A = 13$ to Bob
		      \end{itemize}
		\item \textbf{Bob}: Picks secret $b = 10$
		      \begin{itemize}
			      \item Computes $B = 2^{10} \bmod 17 = 1024 \bmod 17 = 4$
			      \item Sends $B = 4$ to Alice
		      \end{itemize}
		\item \textbf{Both compute shared secret}:
		      \begin{itemize}
			      \item Alice: $S = 4^6 \bmod 17 = 4096 \bmod 17 = 9$
			      \item Bob: $S = 13^{10} \bmod 17 = \ldots = 9$
		      \end{itemize}
	\end{enumerate}
	\textbf{Shared secret}: $S = 9$ (which equals $2^{6 \times 10} \bmod 17$)
\end{frame}

\begin{frame}{The computational Diffie-Hellman (CDH) problem}
	\definitionbox{Computational Diffie-Hellman (CDH) Problem}{
		Given $g^a \bmod p$ and $g^b \bmod p$, compute the shared secret $g^{ab} \bmod p$ without knowing the secret values $a$ and $b$.
	}
	\begin{itemize}[<+->]
		\item \textbf{Motivation}: Even if an eavesdropper captures the public values $g^a$ and $g^b$, they shouldn't be able to determine the shared secret $g^{ab}$.
		\item \textbf{Example}: Given $A = 13$ and $B = 4$ from our earlier example, can you compute $S = 9$?
		      \begin{itemize}
			      \item Without knowing $a = 6$ and $b = 10$, this becomes very difficult!
		      \end{itemize}
		\item \textbf{Real-world relevance}: This is exactly what an attacker faces when trying to break Diffie-Hellman.
	\end{itemize}
\end{frame}

\begin{frame}{CDH vs. DLP: the relationship}
	\begin{itemize}[<+->]
		\item \textbf{Key insight}: If you can solve DLP, then you can also solve CDH.
		      \begin{itemize}
			      \item Given $g^a$ and $g^b$, use DLP to find $a$ and $b$
			      \item Then compute $g^{ab}$ directly
		      \end{itemize}
		\item \textbf{Mathematical relationship}: DLP is \textbf{at least as hard} as CDH.
		      \begin{itemize}
			      \item CDH $\leq$ DLP (CDH reduces to DLP)
		      \end{itemize}
		\item \textbf{Open question}: Is CDH at least as hard as DLP?
		      \begin{itemize}
			      \item We don't know if solving CDH allows you to solve DLP!
			      \item Maybe there's a clever way to compute $g^{ab}$ without finding $a$ and $b$
		      \end{itemize}
		\item \textbf{Security assumption}: We assume CDH is hard even if it's easier than DLP.
	\end{itemize}
\end{frame}

\begin{frame}{The decisional Diffie-Hellman (DDH) problem}
	\definitionbox{Decisional Diffie-Hellman (DDH) Problem}{
		Given $g^a \bmod p$, $g^b \bmod p$, and a value $X$ that is either:
		\begin{itemize}
			\item $g^{ab} \bmod p$ (the real shared secret), or
			\item $g^c \bmod p$ for some random $c$
		\end{itemize}
		\ldots determine which one $X$ is (each choice has probability 1/2).
	}
	\begin{itemize}[<+->]
		\item \textbf{Why do we need this?} Indistinguishability!
		      \begin{itemize}
			      \item What if an attacker can compute the first 32 bits of $g^{ab}$?
			      \item CDH isn't completely broken, but the attacker learned something.
			      \item This partial information might compromise application security.
		      \end{itemize}
		\item \textbf{DDH ensures}: The shared secret $g^{ab}$ is \textbf{indistinguishable} from a random group element.
	\end{itemize}
\end{frame}

\begin{frame}{DDH vs. CDH: the hierarchy}
	\begin{itemize}[<+->]
		\item \textbf{Key relationship}: If you can solve CDH, then you can solve DDH.
		      \begin{itemize}
			      \item Given $(g^a, g^b, X)$, use CDH to compute $g^{ab}$
			      \item Check if $X = g^{ab}$; if yes, then $X$ is the real shared secret
		      \end{itemize}
		\item \textbf{Hardness hierarchy}: DDH $\leq$ CDH $\leq$ DLP
		      \begin{itemize}
			      \item DDH is fundamentally \textbf{easier} than CDH.
			      \item CDH is (probably) easier than DLP.
		      \end{itemize}
		\item \textbf{Surprising fact}: DDH is \textbf{not hard} in certain groups!
		      \begin{itemize}
			      \item In $\mathbb{Z}_p^*$, DDH can be broken using pairing-based techniques.
			      \item But CDH remains hard in the same group.
		      \end{itemize}
		\item \textbf{Solution}: Use elliptic curve groups where DDH is believed hard.
	\end{itemize}
\end{frame}

\begin{frame}{Why DDH matters in cryptography}
	\begin{itemize}[<+->]
		\item \textbf{Indistinguishability}: DDH ensures that shared secrets ``look random''.
		      \begin{itemize}
			      \item Critical for encryption schemes and key derivation.
			      \item Prevents attackers from learning partial information.
		      \end{itemize}
		\item \textbf{Security proofs}: Many cryptographic protocols prove security under DDH.
		      \begin{itemize}
			      \item ElGamal encryption.
			      \item Cramer-Shoup cryptosystem.
			      \item Various authenticated key exchange protocols.
		      \end{itemize}
		\item \textbf{Real-world impact}: Even though DDH is ``weaker'' than CDH, it's one of the most studied and used assumptions.
		      \begin{itemize}
			      \item Provides stronger security guarantees for applications.
			      \item Enables more sophisticated cryptographic constructions.
		      \end{itemize}
	\end{itemize}
\end{frame}

\begin{frame}{Real-world Diffie-Hellman}
	\begin{itemize}[<+->]
		\item \textbf{TLS/HTTPS}: Your browser uses Diffie-Hellman to establish secure connections.
		\item \textbf{Signal}: Uses elliptic-curve Diffie-Hellman for key exchange.
		\item \textbf{SSH}: Secure shell connections use Diffie-Hellman for key agreement.
		\item \textbf{VPNs}: Many VPN protocols rely on Diffie-Hellman for establishing tunnels.
	\end{itemize}
	\vspace{0.5cm}
	\definitionbox{Modern Diffie-Hellman Variants}{
		\begin{itemize}
			\item \textbf{Elliptic Curve Diffie-Hellman (ECDH)}: Same idea, different mathematical group.
			\item \textbf{Post-quantum alternatives}: New key exchange methods for the quantum era.
		\end{itemize}
		\vspace{0.1em}
		\begin{center}
			More on both of the above in future course topics!
		\end{center}
	}
\end{frame}

\begin{frame}{Diffie-Hellman key exchange in practice}
	\begin{center}
		\Large\textbf{How does this actually work in the real world?}
	\end{center}
	\vspace{0.5cm}
	\begin{enumerate}[<+->]
		\item \textbf{Parameter generation}: Choose secure values for $p$ and $g$
		      \begin{itemize}
			      \item $p$ must be a large prime (2048+ bits)
			      \item $g$ must be a generator of a large subgroup
		      \end{itemize}
		\item \textbf{Key generation}: Each party picks a random secret
		      \begin{itemize}
			      \item Alice picks $a$ randomly from $\{1, 2, \ldots, p-2\}$
			      \item Bob picks $b$ randomly from $\{1, 2, \ldots, p-2\}$
		      \end{itemize}
		\item \textbf{Public key computation}: Each party computes their public value
		\item \textbf{Key exchange}: Public values are sent over the network
		\item \textbf{Shared secret derivation}: Each party computes the final shared secret
	\end{enumerate}
\end{frame}

\begin{frame}{TLS handshake: Diffie-Hellman in action}
	\textbf{When you visit https://gmail.com, here's what happens:}
	\vspace{0.5cm}
	\begin{enumerate}[<+->]
		\item \textbf{Client Hello}: Your browser says ``I want to talk securely''
		\item \textbf{Server Hello}: Gmail's server responds with its certificate and DH parameters
		      \begin{itemize}
			      \item Includes $p$, $g$, and server's public value $g^b \bmod p$
		      \end{itemize}
		\item \textbf{Client Key Exchange}: Your browser generates its own secret $a$ and sends $g^a \bmod p$
		\item \textbf{Secret computation}: Both sides compute $g^{ab} \bmod p$
		\item \textbf{Key derivation}: The shared secret is used to derive encryption keys
		\item \textbf{Secure communication}: All further messages are encrypted with these keys
	\end{enumerate}
	\vspace{0.5cm}
	\textbf{Result}: Your password is encrypted before leaving your computer!
\end{frame}

\begin{frame}{Signal's double ratchet: DH everywhere}
	\begin{columns}[c]
		\begin{column}{0.7\textwidth}
			\begin{itemize}[<+->]
				\item \textbf{Initial key exchange}: Uses X3DH (Extended Triple DH)
				      \begin{itemize}
					      \item Combines \textbf{three} DH key exchanges for security.
					      \item Works even when recipient is offline (\textit{``asynchronous''} protocol).\footnote{Everything on this slide will be covered in much more detail later in the course.}
				      \end{itemize}
				\item \textbf{Ongoing communication}: Uses Double Ratchet
				      \begin{itemize}
					      \item New DH key exchange for every message!
					      \item Provides ``forward secrecy'' and ``post-compromise security''.
					      \item If your phone gets compromised today, yesterday's messages remain secure.
					      \item If your phone recovers from compromise, tomorrow's messages are secure again.
				      \end{itemize}
			\end{itemize}
		\end{column}
		\begin{column}{0.3\textwidth}
			\imagewithcaption{signal.jpg}{Signal uses DH key exchange dozens, hundreds of times per conversation.}
		\end{column}
	\end{columns}
\end{frame}

\begin{frame}{The dark side: unauthenticated Diffie-Hellman}
	\begin{center}
		\Large\textbf{But there's a serious problem\ldots}
	\end{center}
	\vspace{0.5cm}
	\begin{itemize}[<+->]
		\item \textbf{The vulnerability}: Basic DH has no authentication
		      \begin{itemize}
			      \item Alice can't verify she's talking to Bob
			      \item Bob can't verify he's talking to Alice
		      \end{itemize}
		\item \textbf{The attack}: Man-in-the-middle (MITM)
		      \begin{itemize}
			      \item Mallory sits between Alice and Bob
			      \item Alice does DH with Mallory, thinking it's Bob
			      \item Bob does DH with Mallory, thinking it's Alice
			      \item Mallory can read and modify everything!
		      \end{itemize}
		\item \textbf{Real-world impact}: This attack is practical and devastating!
	\end{itemize}
\end{frame}

\begin{frame}{Man-in-the-middle attack on DH}
	\textbf{How Mallory breaks ``secure'' communication:}
	\vspace{0.5cm}
	\begin{enumerate}[<+->]
		\item \textbf{Alice $\rightarrow$ Mallory}: Alice sends $g^a$ (thinking it goes to Bob)
		\item \textbf{Mallory $\rightarrow$ Bob}: Mallory sends $g^m$ (Bob thinks it's from Alice)
		\item \textbf{Bob $\rightarrow$ Mallory}: Bob sends $g^b$ (thinking it goes to Alice)
		\item \textbf{Mallory $\rightarrow$ Alice}: Mallory sends $g^m$ (Alice thinks it's from Bob)
		\item \textbf{Result}:
		      \begin{itemize}
			      \item Alice and Mallory share secret $g^{am}$
			      \item Bob and Mallory share secret $g^{bm}$
			      \item Alice and Bob don't share any secret!
		      \end{itemize}
		\item \textbf{Communication}: Alice encrypts with $g^{am}$, Mallory decrypts, reads/modifies, re-encrypts with $g^{bm}$ for Bob
	\end{enumerate}
	\vspace{0.5cm}
	\textbf{Alice and Bob never know they've been compromised!}
\end{frame}

\begin{frame}{Why MITM attacks succeed}
	\begin{itemize}[<+->]
		\item \textbf{Public values look random}: $g^a$ and $g^m$ are indistinguishable.
		      \begin{itemize}
			      \item Both appear to be random group elements.
			      \item No way to tell if they come from the intended party.
		      \end{itemize}
		\item \textbf{No identity verification}: DH only establishes a shared secret.
		      \begin{itemize}
			      \item Doesn't prove who you're sharing it with!
			      \item Like agreeing on a secret handshake with someone wearing a mask.
		      \end{itemize}
		\item \textbf{Active vs. passive attacks}:
		      \begin{itemize}
			      \item DH protects against \textbf{passive} eavesdropping.
			      \item Does nothing against \textbf{active} manipulation.
		      \end{itemize}
		\item \textbf{Historical impact}: This attack has compromised real systems for decades.
	\end{itemize}
\end{frame}

\begin{frame}{Solution: Authenticated Key Exchange}
	\definitionbox{Authenticated Key Exchange (AKE)}{
		Key exchange that verifies the identity of the parties involved, preventing man-in-the-middle attacks.
	}
	\begin{itemize}[<+->]
		\item \textbf{Core idea}: Combine DH with authentication mechanisms
		\item \textbf{Common approaches}:
		      \begin{itemize}
			      \item \textbf{Digital signatures}: Sign the DH public values (TLS).
			      \item \textbf{Pre-shared keys}: Use existing shared secrets (IPsec).
			      \item \textbf{Certificates}: Use a trusted third party (Certificate Authority in HTTPS).
			      \item \textbf{Password-based}: Derive authentication from passwords (SRP protocols).
		      \end{itemize}
		\item \textbf{Goal}: Ensure that Alice and Bob can verify they're really talking to each other.
	\end{itemize}
\end{frame}

\begin{frame}{TLS: authenticated DH with certificates}
	\textbf{How HTTPS prevents MITM attacks:}
	\begin{enumerate}[<+->]
		\item \textbf{Server authentication}: Gmail sends its certificate along with $g^b$
		      \begin{itemize}
			      \item Certificate proves ``this DH value really came from gmail.com''
			      \item Signed by a trusted Certificate Authority (CA)
		      \end{itemize}
		\item \textbf{Certificate verification}: Your browser checks:
		      \begin{itemize}
			      \item Is the signature valid?
			      \item Is the CA trusted?
			      \item Does the certificate match ``gmail.com''?
			      \item Has the certificate expired?
		      \end{itemize}
		\item \textbf{If verification passes}: You know you're really talking to Gmail
		\item \textbf{If verification fails}: Browser shows scary warnings!
	\end{enumerate}
	\textbf{Result}: MITM attacks become much harder (but not impossible!)
\end{frame}

\begin{frame}{Signal: authenticated DH with fingerprints}
	\begin{columns}[c]
		\begin{column}{0.7\textwidth}
			\begin{itemize}[<+->]
				\item \textbf{The bootstrapping problem}: How do Alice and Bob initially authenticate?
				      \begin{itemize}
					      \item No pre-existing certificates.
					      \item No trusted third parties.
				      \end{itemize}
				\item \textbf{Signal's solution}: Security numbers (fingerprints)
				      \begin{itemize}
					      \item Each conversation gets a unique 60-digit number.
					      \item Derived from both parties' long-term identity keys.
				      \end{itemize}
				\item \textbf{Manual verification}: Users compare numbers out-of-band.
				      \begin{itemize}
					      \item Read over the phone\ldots
					      \item Show in person\ldots
					      \item Send via different app\ldots
				      \end{itemize}
			\end{itemize}
		\end{column}
		\begin{column}{0.3\textwidth}
			\imagewithcaption{signal-verification.png}{Signal security number verification screen.}
		\end{column}
	\end{columns}
\end{frame}

\begin{frame}{SSH: authenticated DH with host keys}
	\textbf{How SSH prevents server impersonation:}
	\begin{itemize}[<+->]
		\item \textbf{First connection}: Server presents its ``host key'' along with DH public value
		      \begin{itemize}
			      \item SSH shows you a fingerprint: \texttt{SHA256:ABC123...}
			      \item You're supposed to verify this out-of-band (but nobody does!)
		      \end{itemize}
		\item \textbf{Trust on first use (TOFU)}: Client remembers the host key
		      \begin{itemize}
			      \item Stored in \texttt{\~{}/.ssh/known\_hosts}
		      \end{itemize}
		\item \textbf{Subsequent connections}: Client checks if host key matches
		      \begin{itemize}
			      \item If different, gives you a heart attack: \texttt{WARNING: REMOTE HOST IDENTIFICATION HAS CHANGED!}
			      \item If same: Connection proceeds normally
		      \end{itemize}
		\item \textbf{User authentication}: Usually with passwords or public keys
	\end{itemize}
	\textbf{Weakness}: TOFU is vulnerable on the very first connection!
\end{frame}

\begin{frame}{Modern implementations: elliptic curves}
	\begin{columns}[c]
		\begin{column}{0.5\textwidth}
			\textbf{Traditional DH}
			\begin{itemize}[<+->]
				\item Uses $\mathbb{Z}_p^*$ (integers mod $p$)
				\item Requires 2048+ bit numbers
				\item Slower computations
				\item Larger public keys
			\end{itemize}
		\end{column}
		\begin{column}{0.5\textwidth}
			\textbf{Elliptic Curve DH (ECDH)}
			\begin{itemize}[<+->]
				\item Uses elliptic curve groups
				\item 256-bit keys ≈ 2048-bit traditional DH
				\item Much faster computations
				\item Smaller public keys, less bandwidth
			\end{itemize}
		\end{column}
	\end{columns}
	\vspace{0.5cm}
	\begin{itemize}[<+->]
		\item \textbf{Popular curves}: P-256, P-384, X25519, X448
		\item \textbf{Same security}: Based on elliptic curve discrete logarithm problem
		\item \textbf{Real-world adoption}: ECDH is now standard in TLS, Signal, etc.
		\item \textbf{Performance matters}: Especially important for mobile devices and IoT
	\end{itemize}
\end{frame}

\begin{frame}{The quantum threat to Diffie-Hellman}
	\begin{center}
		\Large\textbf{All DH variants are doomed\ldots}
	\end{center}
	\vspace{0.1cm}
	\begin{itemize}[<+->]
		\item \textbf{Shor's algorithm} (1994) can break DH on quantum computers.
		      \begin{itemize}
			      \item Solves discrete logarithm in polynomial time.
			      \item Works for both traditional DH and ECDH.
		      \end{itemize}
		\item \textbf{Timeline concerns}:
		      \begin{itemize}
			      \item Large quantum computers don't exist yet.
			      \item But adversaries might store encrypted data now, decrypt later.
			      \item ``Harvest now, decrypt later'' attacks.
		      \end{itemize}
		\item \textbf{Post-quantum key exchange}: New algorithms under development.
		      \begin{itemize}
			      \item ML-KEM (based on lattice problems)
			      \item SIDH/SIKE (based on isogenies, but recently broken!)
			      \item Code-based and hash-based alternatives
		      \end{itemize}
	\end{itemize}
\end{frame}

\begin{frame}{Lessons from 50 years of Diffie-Hellman}
	\begin{itemize}[<+->]
		\item \textbf{Elegant mathematics}: Simple idea with profound implications.
		      \begin{itemize}
			      \item Two numbers raised to secret powers in a mathematical group.
		      \end{itemize}
		\item \textbf{Security requires more than math}: Authentication is crucial.
		      \begin{itemize}
			      \item Pure DH is vulnerable to active attacks.
			      \item Real systems need identity verification.
		      \end{itemize}
		\item \textbf{Efficiency drives adoption}: Elliptic curves made DH practical everywhere.
		      \begin{itemize}
			      \item Performance improvements enable new applications.
		      \end{itemize}
		\item \textbf{Future challenges}: Quantum computers will force reinvention.
		      \begin{itemize}
			      \item But the core insight—shared secrets from public exchanges—will survive.
		      \end{itemize}
		\item \textbf{Cryptography is a living field}: Continuous evolution and adaptation.
	\end{itemize}
\end{frame}

\begin{frame}{From hard problems to real-world security}
	\begin{center}
		\Large\textbf{The journey we've traced}
	\end{center}
	\vspace{0.5cm}
	\begin{enumerate}[<+->]
		\item \textbf{Mathematical insight}: Discrete logarithm is hard to compute.
		\item \textbf{Cryptographic innovation}: Diffie-Hellman key exchange leverages this hardness.
		\item \textbf{Real-world impact}: Secure communication for billions of people daily.
	\end{enumerate}
	\vspace{1cm}
	\textbf{This is the power of applied cryptography}: transforming abstract mathematical problems into tools that help people and protect our digital lives.
\end{frame}

\section{RSA}

\begin{frame}{RSA: The other public-key pioneer}
	\begin{center}
		\Large\textbf{One year after Diffie-Hellman came RSA (1977)}
	\end{center}
	\vspace{0.5cm}
	\begin{itemize}[<+->]
		\item \textbf{Named after}: Rivest, Shamir, and Adleman (MIT researchers)
		\item \textbf{Different approach}: Based on the hardness of factoring large numbers
		\item \textbf{Key innovation}: Combines public-key encryption AND digital signatures
		\item \textbf{Revolutionary impact}: Enabled secure communication without prior key exchange
	\end{itemize}
	\vspace{0.5cm}
	\definitionbox{RSA's Core Idea}{
		Anyone can encrypt messages to you using your public key, but only you can decrypt them with your private key.
	}
\end{frame}

\begin{frame}{How RSA works (simplified)}
	\textbf{The mathematical foundation}:
	\begin{enumerate}[<+->]
		\item \textbf{Key generation}:
		      \begin{itemize}
			      \item Choose two large primes $p$ and $q$
			      \item Compute $n = p \times q$ (public modulus)
			      \item Choose public exponent $e$ (commonly 65537)
			      \item Compute private exponent $d$ such that $e \times d \equiv 1 \pmod{\phi(n)}$
			            \begin{itemize}
				            \item $\phi(n)$ is Euler's totient function: counts integers less than $n$ that are coprime to $n$
				            \item For $n = p \times q$: $\phi(n) = (p-1)(q-1)$
			            \end{itemize}
		      \end{itemize}
		\item \textbf{Encryption}: $c = m^e \bmod n$ (anyone can do this)
		\item \textbf{Decryption}: $m = c^d \bmod n$ (only private key holder can do this)
	\end{enumerate}
	\vspace{0.5cm}
	\textbf{Security assumption}: Factoring $n$ back into $p$ and $q$ is computationally infeasible
\end{frame}

\begin{frame}{RSA's dual functionality}
	\begin{columns}[c]
		\begin{column}{0.5\textwidth}
			\textbf{Encryption}
			\begin{itemize}[<+->]
				\item Public key encrypts
				\item Private key decrypts
				\item Provides confidentiality
				\item Anyone can send you secrets
			\end{itemize}
		\end{column}
		\begin{column}{0.5\textwidth}
			\textbf{Digital Signatures}
			\begin{itemize}[<+->]
				\item Private key signs
				\item Public key verifies
				\item Provides authenticity
				\item You can prove you wrote something
			\end{itemize}
		\end{column}
	\end{columns}
	\vspace{0.5cm}
	\begin{itemize}[<+->]
		\item \textbf{Historical significance}: First algorithm to do both!
		\item \textbf{Widespread adoption}: Used in SSL/TLS, email encryption, code signing
		\item \textbf{Patent controversy}: Was patented until 2000, limiting early adoption
	\end{itemize}
\end{frame}

\begin{frame}{RSA in practice}
	\textbf{Where RSA dominated for decades}:
	\begin{itemize}[<+->]
		\item \textbf{HTTPS certificates}: Most websites used RSA signatures until recently
		\item \textbf{Email encryption}: PGP/GPG relied heavily on RSA
		\item \textbf{Software signing}: Windows, macOS, Linux packages
		\item \textbf{SSH keys}: RSA was the default for years
	\end{itemize}
	\vspace{0.5cm}
	\textbf{Key sizes and security}:
	\begin{itemize}[<+->]
		\item 1024-bit RSA: Broken, considered insecure since 2010
		\item 2048-bit RSA: Current minimum, but showing its age
		\item 4096-bit RSA: Secure but slow and unwieldy
		\item Compare to ECDSA: 256-bit keys offer similar security to 3072-bit RSA!
	\end{itemize}
\end{frame}

\begin{frame}{Why we're not covering RSA in depth}
	\begin{center}
		\Large\textbf{RSA is a cryptographic dinosaur}
	\end{center}
	\vspace{0.25cm}
	\begin{itemize}[<+->]
		\item \textbf{On its way out}: Being actively replaced everywhere
		      \begin{itemize}
			      \item TLS 1.3 prefers ECDSA and EdDSA
			      \item Modern systems default to elliptic curves
			      \item Even RSA Inc. recommends moving away from RSA!
		      \end{itemize}
		\item \textbf{Not a building block}: Unlike Diffie-Hellman
		      \begin{itemize}
			      \item DH principles apply to modern protocols (ECDH, X25519)
			      \item RSA's approach doesn't translate to newer systems
		      \end{itemize}
		\item \textbf{Performance issues}: Terribly slow compared to alternatives
		      \begin{itemize}
			      \item 4096-bit RSA signature: ~10ms
			      \item Ed25519 signature: ~0.1ms (100x faster!)
		      \end{itemize}
		\item \textbf{Implementation pitfalls}: Padding oracle attacks, timing attacks, etc.
		\item \textbf{The future is elliptic curves and post-quantum}: Time better spent there!
	\end{itemize}
\end{frame}

\begin{frame}[plain]
	\titlepage
\end{frame}
\end{document}
