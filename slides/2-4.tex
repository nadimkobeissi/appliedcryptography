\documentclass[aspectratio=169, lualatex, handout]{beamer}
\makeatletter\def\input@path{{theme/}}\makeatother\usetheme{cipher}

\title{Applied Cryptography}
\author{Nadim Kobeissi}
\institute{American University of Beirut}
\instituteimage{images/aub_white.png}
\date{\today}
\coversubtitle{CMPS 297AD/396AI\\Fall 2025}
\coverpartname{Part 2: Real-World Cryptography}
\covertopicname{2.4: End-to-End Encrypted Cloud Storage}
\coverwebsite{https://appliedcryptography.page}

\begin{document}
\begin{frame}[plain]
	\titlepage
\end{frame}

\incompleteslideswarning

\section{Introduction}

\incompleteslideswarning

\begin{frame}{Cloud storage today}
	\begin{columns}[c]
		\begin{column}{0.5\textwidth}
			\begin{itemize}
				\item Cloud storage is a prevalent computing paradigm today
				\item Outsources data storage to third party services
				\item Benefits for users:
				      \begin{itemize}
					      \item No need to worry about backups
					      \item High data availability
					      \item Access data from anywhere
				      \end{itemize}
				\item Major providers (Google, Microsoft, Apple, Amazon) control encryption keys
			\end{itemize}
		\end{column}
		\begin{column}{0.5\textwidth}
			\imagewithcaption{onedrive.png}{Microsoft OneDrive is an example of a popular cloud storage service.}
		\end{column}
	\end{columns}
\end{frame}

\begin{frame}{Regular vs End-to-End Encrypted Cloud Storage}
	\begin{columns}
		\begin{column}{0.5\textwidth}
			\textbf{Regular Cloud Storage}\\
			\textit{(e.g., OneDrive, Google Drive)}
			\vspace{0.5cm}
			\begin{itemize}
				\item \textbf{Transport encryption:} TLS protects data in transit
				\item \textbf{Encryption at rest:} Provider encrypts on their servers
				\item \textbf{Provider controls keys:} Can decrypt your data
			\end{itemize}
		\end{column}
		\begin{column}{0.5\textwidth}
			\textbf{E2E Encrypted Storage}\\
			\textit{(e.g., MEGA, Nextcloud)}
			\vspace{0.5cm}
			\begin{itemize}
				\item \textbf{Client-side encryption:} Data encrypted before upload
				\item \textbf{User controls keys:} Provider never sees plaintext
				\item \textbf{Confidentiality:} Provider learns nothing about content
			\end{itemize}
		\end{column}
	\end{columns}
\end{frame}

\begin{frame}{Regular vs End-to-End Encrypted Cloud Storage}
	\begin{columns}
		\begin{column}{0.5\textwidth}
			\textbf{Regular Cloud Storage}\\
			\textit{(e.g., OneDrive, Google Drive)}
			\\
			\vspace{0.5cm}
			\textbf{Risks:}
			\begin{itemize}
				\item Provider breach exposes data
				\item Government requests
				\item Insider threats
				\item Provider can analyze content
			\end{itemize}
		\end{column}
		\begin{column}{0.5\textwidth}
			\textbf{E2E Encrypted Storage}\\
			\textit{(e.g., MEGA, Nextcloud)}
			\\
			\vspace{0.5cm}
			\textbf{Guarantees:}
			\begin{itemize}
				\item Provider breach reveals nothing
				\item Government gets encrypted data
				\item No insider access to content
				\item Provider can only analyze metadata
			\end{itemize}
		\end{column}
	\end{columns}
\end{frame}

\begin{frame}{End-to-End Encryption (E2EE) for cloud storage}
	\begin{itemize}
		\item Alternative approach: Users encrypt before uploading
		\item Users control their own encryption keys
		\item Security guarantees:
		      \begin{itemize}
			      \item Cryptographic confidentiality
			      \item Authenticity and integrity protection
			      \item Provider can't read file contents
			      \item Provider can't modify files undetectably
		      \end{itemize}
		\item Security maintained even if service is compromised
		\item Ideally secure against malicious service providers
	\end{itemize}
\end{frame}

\begin{frame}{Key management challenges}
	\begin{itemize}
		\item Key management delegated to users
		\item Humans are poor at memorizing cryptographic keys
		\item Security typically bootstrapped from passwords
		\item Fundamental limitation: Dictionary attacks possible
		\item Trade-off between convenience and security
		\item Strong password policies can partially mitigate risks
	\end{itemize}
\end{frame}

\begin{frame}{Password hashing}
	\begin{columns}[c]
		\begin{column}{1\textwidth}
			\begin{itemize}
				\item Users typically access E2EE storage via passwords
				\item Passwords must be converted to cryptographic keys
				\item Common approach: password hashing\footnote{This was previously covered in our \textit{Collision-Resistant Hash Functions} topic.}
				\item Popular password hashing functions:
				      \begin{itemize}
					      \item \textbf{PBKDF2}: Widely deployed but aging
					      \item \textbf{scrypt}: Memory-hard against ASICs
					      \item \textbf{Argon2}: Winner of Password Hashing Competition
				      \end{itemize}
				\item KDFs slow down brute-force attacks
				\item Still vulnerable to dictionary attacks with weak passwords
			\end{itemize}
		\end{column}
	\end{columns}
\end{frame}

\begin{frame}{The authentication vs encryption dilemma}
	\begin{columns}[c]
		\begin{column}{1\textwidth}
			\begin{itemize}
				\item Traditional services: Server stores password hash
				\item E2EE requirement: Server must never see encryption key
				\item Two common approaches:
				      \begin{itemize}
					      \item \textbf{Dual-password:} Separate login and encryption passwords
					      \item \textbf{SRP (Secure Remote Password):} Zero-knowledge proof of password
					      \item \textbf{Hardware-backed authentication:} Passkeys, Apple Secure Enclave, etc.
				      \end{itemize}
				\item Trade-offs:
				      \begin{itemize}
					      \item Dual-password: Poor usability, users often reuse
					      \item SRP: Complex implementation
					      \item Hardware-backed: Requires specific devices/platforms
				      \end{itemize}
				\item MEGA uses custom authentication with encryption keys
			\end{itemize}
		\end{column}
	\end{columns}
\end{frame}

\begin{frame}{Password recovery challenges}
	\begin{itemize}
		\item Lost password = lost data in true E2EE
		      \begin{itemize}
			      \item The ``mud puddle test''
		      \end{itemize}
		\item Users expect password recovery options
		\item Common mitigation strategies:
		      \begin{itemize}
			      \item \textbf{Recovery keys:} Long random string users must store
			      \item \textbf{Social recovery:} Threshold sharing among contacts
			      \item \textbf{Security questions:} Weaker but familiar to users
		      \end{itemize}
	\end{itemize}
\end{frame}

\begin{frame}{Apple's Advanced Data Protection}
	\begin{columns}[c]
		\begin{column}{1\textwidth}
			\begin{itemize}
				\item Apple introduced optional E2EE for iCloud in 2023\footnote{\url{https://support.apple.com/en-us/102651}}
				\item Called ``Advanced Data Protection'' (ADP)
				\item Covers most iCloud data categories:
				      \begin{itemize}
					      \item Photos, Notes, iCloud Backup, Messages backup
					      \item iCloud Drive, Reminders, Safari bookmarks
				      \end{itemize}
				\item Apple's approach to recovery:
				      \begin{itemize}
					      \item \textbf{Option 1:} User-managed recovery key
					      \item \textbf{Option 2:} Apple escrow (default)
					      \item Users must explicitly opt-in to ADP
				      \end{itemize}
				\item Fundamental tension: Security vs recoverability
				\item Apple chose user choice over mandatory E2EE
			\end{itemize}
		\end{column}
	\end{columns}
\end{frame}

\begin{frame}{Multi-device key management}
	\begin{itemize}
		\item Users expect seamless multi-device access
		\item Each device needs access to encryption keys
		\item Key synchronization approaches:
		      \begin{itemize}
			      \item \textbf{Password-based:} Derive keys on each device
			      \item \textbf{Device pairing:} Transfer keys device-to-device
			      \item \textbf{Cloud keychain:} Encrypted key storage (circular problem)
		      \end{itemize}
		\item Additional challenges:
		      \begin{itemize}
			      \item Device revocation when lost/stolen
			      \item Key rotation after compromise
			      \item Offline device synchronization
		      \end{itemize}
	\end{itemize}
\end{frame}

\begin{frame}{Research directions in usable key management}
	\begin{itemize}
		\item \textbf{Threshold cryptography:} Split keys across multiple locations
		\item \textbf{Hardware tokens:} FIDO2/WebAuthn/Passkeys for encryption key storage
		\item \textbf{Biometric key derivation:} Fuzzy extractors from fingerprints
		\item \textbf{Social key recovery:} Friends/family as key custodians
		\item Open problems:
		      \begin{itemize}
			      \item Balancing security with grandmother-friendliness
			      \item Recovery without trusted third parties
			      \item Protecting against coercion/rubber-hose attacks
		      \end{itemize}
		\item Most solutions add complexity or trust assumptions
	\end{itemize}
\end{frame}

\begin{frame}{Passkeys: A new authentication paradigm}{Quick aside}
	\begin{columns}[c]
		\begin{column}{0.6\textwidth}
			\begin{itemize}
				\item \textbf{Passkeys} are a password replacement based on WebAuthn/FIDO2
				\item Built on public-key cryptography:
				      \begin{itemize}
					      \item Private key stored securely on device
					      \item Public key registered with service
					      \item Authentication via cryptographic signatures
				      \end{itemize}
				\item Key advantages over passwords:
				      \begin{itemize}
					      \item Phishing-resistant by design
					      \item No shared secrets with server
					      \item Automatic strong credentials
				      \end{itemize}
				\item Apple: iCloud Keychain, Microsoft: Windows Hello\ldots
			\end{itemize}
		\end{column}
		\begin{column}{0.4\textwidth}
			\imagewithcaption{passkeys_mac.png}{Passkey dialog on macOS.}
		\end{column}
	\end{columns}
\end{frame}

\begin{frame}{WebAuthn PRF: symmetric keys from passkeys}{Quick aside}
	\begin{itemize}
		\item WebAuthn's \textbf{PRF (Pseudo-Random Function)} extension enables key derivation
		\item Key derivation process:
		      \begin{itemize}
			      \item PRF can be evaluated during credential creation/assertion
			      \item Produces 32-byte outputs suitable for encryption keys
			      \item Supports two inputs per operation (for key rotation)
		      \end{itemize}
		\item Implementation typically uses HMAC-SHA256 internally
	\end{itemize}
\end{frame}

\begin{frame}{$\mathsf{PRF}: F_{k}= X \rightarrow Y$}{Reminder}
	\begin{columns}[c]
		\begin{column}{0.4\textwidth}
			\begin{itemize}
				\item We want the mapping to be:
				      \begin{itemize}
					      \item One-way
					      \item ``Randomized''
					      \item Relations between inputs not reflected in outputs
				      \end{itemize}
			\end{itemize}
		\end{column}

		\begin{column}{0.8\textwidth}
			\begin{tikzpicture}[scale=0.38]
				% Define colors
				\definecolor{domaingreen}{RGB}{102, 170, 68}
				\definecolor{rangegreen}{RGB}{170, 187, 136}
				\definecolor{circlecolor}{RGB}{235, 137, 85}
				\definecolor{purplearrow}{RGB}{160, 78, 160}
				\definecolor{redarrow}{RGB}{237, 50, 36}

				% Input space (domain) X - made square
				\draw[dashed, thick, domaingreen, fill=domaingreen]
				(0,0) rectangle (8,8);
				\node[text width=6.5cm, align=center, font=\normalsize]
				at
				(4,-0.8)
				{Size: infinite!};
				\node[font=\small] at (4,9) {Input space (domain) $X$};

				% Output (range) Y - made square - moved more to the right
				\draw[thick, rangegreen, fill=rangegreen] (15,2) rectangle (20,7);
				\node[text width=4cm, align=center, font=\normalsize]
				at
				(17.5,1.2)
				{Size: fixed};
				\node[font=\small] at (17.5,8.5) {Output (range) $Y$};
				% Input dots - adjusted positions for square domain
				\filldraw[circlecolor] (2,7) circle (0.3);
				\pause
				\draw[-{Stealth[length=6mm, width=4mm]}, thick, purplearrow]
				(2,7) -- (16.2,6.4);
				\pause
				\filldraw[circlecolor] (16.2,6.4) circle (0.3);
				\pause

				\filldraw[circlecolor] (3,6) circle (0.3);
				\pause
				\draw[-{Stealth[length=6mm, width=4mm]}, thick, purplearrow]
				(3,6) -- (18.6,5.3);
				\pause
				\filldraw[circlecolor] (18.6,5.3) circle (0.3);
				\pause

				\filldraw[circlecolor] (2,5) circle (0.3);
				\pause
				\draw[-{Stealth[length=6mm, width=4mm]}, thick, purplearrow]
				(2,5) -- (16.8,4.2);
				\pause
				\filldraw[circlecolor] (16.8,4.2) circle (0.3);
				\pause

				\filldraw[circlecolor] (4,3.5) circle (0.3);
				\pause
				\draw[-{Stealth[length=6mm, width=4mm]}, thick, purplearrow]
				(4,3.5) -- (18.4,3.2);
				\pause
				\filldraw[circlecolor] (18.4,3.2) circle (0.3);
				\pause

				\filldraw[circlecolor] (2,2) circle (0.3);
				\pause
				\draw[-{Stealth[length=6mm, width=4mm]}, thick, purplearrow]
				(2,2) -- (17.1,2.7);
				\pause
				\filldraw[circlecolor] (17.1,2.7) circle (0.3);
				\pause

				\filldraw[circlecolor] (3,1) circle (0.3);
				\pause
				\draw[-{Stealth[length=6mm, width=4mm]}, ultra thick, redarrow]
				(3,1) -- (16.8,4.2);
				\node[redarrow, font=\scriptsize\bfseries, rotate=14]
				at
				(10,3)
				{Collisions are inevitable};
			\end{tikzpicture}
		\end{column}
	\end{columns}
\end{frame}

\begin{frame}{Using passkeys for key management}{Quick aside}
	\begin{itemize}
		\item Potential architecture combining passkeys with E2EE:
		      \begin{itemize}
			      \item User authenticates with passkey (no password needed)
			      \item PRF extension derives encryption keys on-demand
			      \item Keys never leave the authenticator/device
		      \end{itemize}
		\item Benefits:
		      \begin{itemize}
			      \item No password-derived keys (stronger security)
			      \item Hardware-backed key storage
			      \item Seamless multi-device via platform sync
		      \end{itemize}
		\item Current limitations:
		      \begin{itemize}
			      \item PRF extension not universally supported
			      \item Security keys may not support PRF evaluation at creation
		      \end{itemize}
		\item Represents potential future direction for usable E2EE key management and authentication more generally
	\end{itemize}
\end{frame}

\begin{frame}{File sharing challenges}
	\begin{itemize}
		\item Primary feature: Sharing files with selected users
		\item Recipients need decryption keys
		\item Keys cannot be sent unprotected via untrusted provider
		\item Common solutions:
		      \begin{itemize}
			      \item Encrypt file key under recipient's public key
			      \item Use out-of-band mechanisms
		      \end{itemize}
	\end{itemize}
\end{frame}

\begin{frame}{E2EE cloud storage providers}
	\begin{itemize}
		\item Major providers:
		      \begin{itemize}
			      \item MEGA: 300M users, 150B files, 1000 PB data
			      \item Nextcloud
			      \item Apple iCloud (optional E2EE since 2023)
		      \end{itemize}
		\item Smaller providers:
		      \begin{itemize}
			      \item icedrive, pCloud, Seafile, Tresorit
			      \item Tens of millions of users collectively
		      \end{itemize}
	\end{itemize}
\end{frame}

\begin{frame}{Security concerns with current E2EE providers}
	\begin{itemize}
		\item Recent analyses found attacks on MEGA and Nextcloud
		\item Both providers use complex, ad hoc cryptographic designs
		\item Lack formal security guarantees
		\item MEGA's challenges:
		      \begin{itemize}
			      \item Migration infeasible - requires user participation
			      \item Initial minimal patches proved insufficient
			      \item Required second round of complex patching
		      \end{itemize}
		\item Nextcloud's response:
		      \begin{itemize}
			      \item Forced to disable E2EE file sharing entirely
			      \item Significant feature regression
			      \item Required full redesign
		      \end{itemize}
	\end{itemize}
\end{frame}

\begin{frame}{Current state of E2EE Cloud Storage}
	\begin{itemize}
		\item Most providers rely on proprietary, unanalyzed designs
		\item Highlights need for formally analyzed cryptographic designs
		\item Security confidence lags behind E2EE messaging and browsing
		\item Less design input from cryptographic research community
		\item Need for standardized, proven security approaches
		\item \textbf{In this topic, we'll cover:} \begin{enumerate}
			      \item What went wrong with MEGA
			      \item What went wrong with Nextcloud
			      \item A quick look at WhatsApp encrypted cloud backups
			      \item An attempt by cryptographers to design a provably secure E2EE Cloud Storage protocol
		      \end{enumerate}
	\end{itemize}
\end{frame}

\incompleteslideswarning

\section{Case Study: MEGA}

\begin{frame}{MEGA: Overview}
	\begin{columns}[c]
		\begin{column}{0.8\textwidth}
			\begin{itemize}
				\item Founded in 2013 by Kim Dotcom (post-Megaupload)
				\item One of the largest E2EE cloud storage providers
				\item \textbf{Key statistics}:
				      \begin{itemize}
					      \item Over 300 million registered users\footnote{\url{https://blog.mega.io/mega-reaches-300-million-users-milestone}}
					      \item 150+ billion files stored
				      \end{itemize}
				\item \textbf{Core features}:
				      \begin{itemize}
					      \item Client-side encryption before upload
					      \item Secure file sharing via encrypted links
					      \item Cross-platform support (web, desktop, mobile)
					      \item Free tier with 20 GB storage
				      \end{itemize}
				\item All encryption performed in JavaScript (web client)
			\end{itemize}
		\end{column}
		\begin{column}{0.2\textwidth}
			\imagewithcaption{mega_logo.png}{MEGA's logo.}
		\end{column}
	\end{columns}
\end{frame}

\begin{frame}{MEGA: Key hierarchy}
	\begin{columns}[c]
		\begin{column}{0.6\textwidth}
			\begin{itemize}
				\item Password serves as root of trust for entire key hierarchy
				\item Key derivation from password:
				      \begin{itemize}
					      \item \textbf{Authentication key:} Identifies user to MEGA
					      \item \textbf{Encryption key:} Encrypts the master key
				      \end{itemize}
				\item \textbf{Master key:} Randomly generated, encrypts all other keys
			\end{itemize}
		\end{column}
		\begin{column}{0.4\textwidth}
			\imagewithcaption{mega_keys.png}{MEGA's key hierarchy. The master key encrypts the share, chat, sign
				and node keys using AES-ECB. Source: \url{https://appliedcryptography.page/papers/\#mega-awry}}
		\end{column}
	\end{columns}
\end{frame}

\begin{frame}{MEGA: Key hierarchy}
	\begin{columns}[c]
		\begin{column}{0.6\textwidth}
			\begin{itemize}
				\item Asymmetric key pairs for each account:
				      \begin{itemize}
					      \item \textbf{RSA key pair:} For data sharing
					      \item \textbf{Curve25519 key pair:} For chat key exchange
					      \item \textbf{Ed25519 key pair:} For signing other keys
				      \end{itemize}
				\item \textbf{Per-node keys:} New key for each file/folder
				\item All keys encrypted with master key using AES-ECB
				\item Encrypted keys stored on MEGA servers
				\item Enables multi-device access with password only
			\end{itemize}
		\end{column}
		\begin{column}{0.4\textwidth}
			\imagewithcaption{mega_keys.png}{MEGA's key hierarchy. The master key encrypts the share, chat, sign
				and node keys using AES-ECB. Source: \url{https://appliedcryptography.page/papers/\#mega-awry}}
		\end{column}
	\end{columns}
\end{frame}

\begin{frame}{MEGA: Where's the attack?}
	\bigimagewithcaption{mega_keys.png}{MEGA's key hierarchy. The master key encrypts the share, chat, sign
		and node keys using AES-ECB. Source: \url{https://appliedcryptography.page/papers/\#mega-awry}}
\end{frame}

\begin{frame}{MEGA: Key hierarchy}
	\begin{columns}[c]
		\begin{column}{1\textwidth}
			\begin{itemize}
				\item Password-to-key derivation
				\item MEGA uses HKDF for deriving keys from password
				\item \textbf{Problem:} HKDF is not a password hash!
				      \begin{itemize}
					      \item HKDF designed for key derivation from high-entropy inputs
					      \item No computational hardness against brute-force
					      \item Passwords have low entropy
				      \end{itemize}
				\item \textbf{Should use:} Argon2, scrypt, or PBKDF2
				      \begin{itemize}
					      \item Memory-hard functions
					      \item Configurable iteration counts
					      \item Resistance to GPU/ASIC attacks
				      \end{itemize}
				\item Enables offline dictionary attacks on user passwords
			\end{itemize}
		\end{column}
	\end{columns}
\end{frame}

\begin{frame}{Beyond password hashing: Zero-knowledge protocols}{Quick aside}
	\begin{itemize}
		\item Password hashing has fundamental limitations:
		      \begin{itemize}
			      \item Server must receive password (or hash) to verify
			      \item Vulnerable to offline dictionary attacks
			      \item Trade-off between security and performance
		      \end{itemize}
		\item \textbf{Zero-knowledge password protocols} offer better security:
		      \begin{itemize}
			      \item Server never sees password or password-equivalent
			      \item No offline dictionary attacks possible
			      \item Cryptographic proof of password knowledge
		      \end{itemize}
	\end{itemize}
\end{frame}

\begin{frame}{SRP (Secure Remote Password)}{Quick aside}
	\begin{itemize}
		\item Developed by Tom Wu at Stanford (1998)
		\item \textbf{Key properties:}
		      \begin{itemize}
			      \item Password never transmitted, even encrypted
			      \item Mutual authentication between client and server
			      \item Generates session key for subsequent encryption
			      \item Patent-free and standardized (RFC 2945)
		      \end{itemize}
		\item \textbf{How it works:}
		      \begin{itemize}
			      \item Registration: Client sends verifier derived from password
			      \item Authentication: Zero-knowledge proof using verifier
			      \item Based on discrete logarithm problem
		      \end{itemize}
		\item \textbf{Adoption:} Used by Apple iCloud, 1Password, ProtonMail
	\end{itemize}
\end{frame}

\begin{frame}{OPAQUE: Next-generation password authentication}{Quick aside}
	\begin{itemize}
		\item Developed by Jarecki, Krawczyk, and Xu (2018)
		\item \textbf{Advantages over SRP:}
		      \begin{itemize}
			      \item Stronger security proofs in UC framework
			      \item Protection against pre-computation attacks
			      \item Forward secrecy for passwords
			      \item Quantum-resistant variants possible
		      \end{itemize}
		\item \textbf{Key innovation:} Oblivious PRF (OPRF)
		      \begin{itemize}
			      \item Server helps compute PRF without learning input
			      \item Prevents server from testing password guesses
		      \end{itemize}
		\item Being standardized by IETF (RFC 9497)
		\item Early adoption by WhatsApp for backup encryption
	\end{itemize}
\end{frame}

\begin{frame}{Signal's SVR3: Multi-enclave key recovery}{Quick aside}
	\begin{columns}[c]
		\begin{column}{1\textwidth}
			\begin{itemize}
				\item Signal faced same challenge: E2EE backups need secure key recovery
				\item \textbf{SVR3 (Secure Value Recovery 3)\footnote{\url{https://appliedcryptography.page/papers/\#signal-recovery}}:} PIN-based key recovery
				      \begin{itemize}
					      \item Distributes trust across 3 different hardware enclaves
					      \item Intel SGX (Azure), AMD SEV-SNP (Google Cloud), Nitro (AWS)
					      \item Attacker must compromise ALL three to access keys
				      \end{itemize}
				\item \textbf{Key features:}
				      \begin{itemize}
					      \item Uses Password Protected Secret Sharing (PPSS)
					      \item Limits PIN guesses via secure hardware
					      \item No single point of failure
					      \item Cost: \$0.0025/user/year
				      \end{itemize}
				\item Already deployed to millions of Signal users
			\end{itemize}
		\end{column}
	\end{columns}
\end{frame}

\begin{frame}{Comparing authentication approaches}{Quick aside}
	\begin{columns}
		\begin{column}{0.33\textwidth}
			\textbf{Password Hashing}
			\vspace{0.3cm}
			\begin{itemize}
				\item[\textcolor{green}{\mycheckmark}] Simple to implement
				\item[\textcolor{green}{\mycheckmark}] Well understood
				\item[\textcolor{red}{$\times$}] Offline attacks
				\item[\textcolor{red}{$\times$}] Server sees password
			\end{itemize}
		\end{column}
		\begin{column}{0.33\textwidth}
			\textbf{SRP}
			\vspace{0.3cm}
			\begin{itemize}
				\item[\textcolor{green}{\mycheckmark}] No offline attacks
				\item[\textcolor{green}{\mycheckmark}] Proven secure
				\item[\textcolor{orange}{$\sim$}] Complex protocol
				\item[\textcolor{red}{$\times$}] Pre-computation attacks
			\end{itemize}
		\end{column}
		\begin{column}{0.34\textwidth}
			\textbf{OPAQUE}
			\vspace{0.3cm}
			\begin{itemize}
				\item[\textcolor{green}{\mycheckmark}] Strongest security
				\item[\textcolor{green}{\mycheckmark}] Forward secrecy
				\item[\textcolor{red}{$\times$}] New protocol
				\item[\textcolor{red}{$\times$}] Limited deployment
			\end{itemize}
		\end{column}
	\end{columns}
	\vspace{0.5cm}
	\begin{center}
		\textit{For E2EE storage: Can derive encryption keys from authentication protocol outputs}
	\end{center}
\end{frame}

\incompleteslideswarning

\section{Case Study: Nextcloud}

\incompleteslideswarning

\section{Case Study: WhatsApp End-to-End Encrypted Backups}

\incompleteslideswarning

\section{Designing an End-to-End Cloud Storage Protocol}

\incompleteslideswarning

\begin{frame}[plain]
	\titlepage
\end{frame}
\end{document}
