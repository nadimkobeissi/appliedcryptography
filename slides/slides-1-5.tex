\documentclass[aspectratio=169, lualatex, handout]{beamer}
\makeatletter\def\input@path{{theme/}}\makeatother\usetheme{cipher}

\title{Applied Cryptography - 1.5: Chosen-Plaintext \& Chosen-Ciphertext Attacks}
\author{Nadim Kobeissi}
\subject{An introduction to chosen-plaintext and chosen-ciphertext attacks in cryptography, covering CPA/CCA security definitions, real-world vulnerabilities, and authenticated encryption.}
\keywords{CPA security, CCA security, chosen-plaintext attacks, chosen-ciphertext attacks, format-oracle attacks, malleability, message authentication codes, authenticated encryption, encrypt-then-MAC}
\institute{American University of Beirut}
\instituteimage{images/aub_white.png}
\date{\today}
\coversubtitle{CMPS 297AD/396AI\\Fall 2025}
\coverpartname{Part 1: Provable Security}
\covertopicname{1.5: Chosen-Plaintext \& Chosen-Ciphertext Attacks}
\coverwebsite{https://appliedcryptography.page}

\begin{document}
\begin{frame}[plain]
	\titlepage
\end{frame}

\section{Chosen-Plaintext Attacks}

\begin{frame}{CPA Security}
	\begin{columns}[c]
		\begin{column}{0.6\textwidth}
			\definitionbox{Security Against Chosen-Plaintext Attacks (CPA Security)}{
				Let $\Sigma$ be an encryption scheme, and let $\Sigma.\mathcal{C}(\ell)$ denote the set of possible ciphertexts for plaintexts of length $\ell$. If $\Sigma$ supports only plaintexts of a single length, we can simply write $\Sigma.\mathcal{C}$ to denote the entire set of ciphertexts.\footnote{i.e. ``just forget about the length''.}\\[0.5em]

				$\Sigma$ has security against \textbf{chosen-plaintext attacks} if the following two libraries are indistinguishable:
			}
		\end{column}
		\begin{column}{0.4\textwidth}
			\sslinked{
				\sslibrary{\Sigma}{cpa-real}{
					$K \twoheadleftarrow \Sigma.\mathcal{K}$\\[1em]
					\sslibrarysubroutine{cpa.enc}{M}{
						$C \coloneq \Sigma.\texttt{Enc}(K, M)$\\
						return $C$
					}{1}
				}{0.8}
			}{\approxeq}{
				\sslibrary{\Sigma}{cpa-rand}{
					\sslibrarysubroutine{cpa.enc}{M}{
						$C \twoheadleftarrow \Sigma.\mathcal{C}(|M|)$\\
						return $C$
					}{1}
				}{0.8}
			}
		\end{column}
	\end{columns}
\end{frame}

\begin{frame}{Why CPA security matters}
	\begin{itemize}[<+->]
		\item CPA security means: \textit{``even if I let you encrypt any message you want, you can't obtain any distinguisher regarding my scheme.''}
		\item \textbf{Why this matters in real life:}
		      \begin{itemize}[<+->]
			      \item Attackers often can trick systems into encrypting data they choose.
			      \item Without CPA security, seeing these encryptions could reveal your secrets.
			      \item Example: If bank transactions always encrypt to the same ciphertexts, attackers could identify your purchases.
		      \end{itemize}
		\item The only thing that's allowed to leak is the length of messages/ciphertext.
		\item Most modern encryption is designed to have this important property.
	\end{itemize}
\end{frame}

\begin{frame}{Message length: not important, not unimportant}
	\begin{itemize}[<+->]
		\item Even with CPA security, the \textbf{length} of messages is still leaked.
		\item This seemingly minor leak can reveal surprising information:
		      \begin{itemize}[<+->]
			      \item \textbf{Encrypted VoIP calls}: Length patterns can reveal which language is spoken.
			      \item \textbf{Encrypted web traffic}: Sizes of requests/responses identify websites.
			      \item \textbf{Encrypted messages}: Length patterns can reveal the type of content (document, image, etc.)
			      \item \textbf{Encrypted commands}: Length often reveals which command was issued.
		      \end{itemize}
		\item Mitigation typically involves padding messages to fixed lengths or standard increments.
		\item This is why many secure protocols use fixed-size packets or add random padding.
	\end{itemize}
\end{frame}

\begin{frame}{CPA security is why we avoid deterministic encryption}
	\begin{columns}[c]
		\begin{column}{0.5\textwidth}
			\begin{itemize}[<+->]
				\item Deterministic encryption will always fail CPA security!
				\item If the same message always encrypts to the same ciphertext:
				      \begin{itemize}
					      \item Attacker can recognize repeat messages.
					      \item Can build a ``dictionary'' of known plaintexts.
					      \item etc.
				      \end{itemize}
				\item ECB mode is a classic example of insecure deterministic encryption.
			\end{itemize}
		\end{column}
		\begin{column}{0.5\textwidth}
			\begin{center}
				\ssprogram{}{
					$M \twoheadleftarrow \mathcal{M}$\\
					$C_1 \coloneq \func{cpa.enc}{M}$\\
					$C_2 \coloneq \func{cpa.enc}{M}$\\
					return $C_1 == C_2$
				}{1}
			\end{center}
		\end{column}
	\end{columns}
\end{frame}

\begin{frame}{Non-deterministic encryption isn't hard\ldots}
	\begin{columns}[c]
		\begin{column}{0.5\textwidth}
			\begin{itemize}[<+->]
				\item AES-CTR turns AES into a non-deterministic PRF\ldots
				\item AES-GCM even turns it into a non-deterministic authenticated cipher\ldots\footnote{More on these later. An authenticated cipher is one where the ciphertext can't be modified by the adversary without that being detected.}
				\item We can just as easily make a PRF non-deterministic:
			\end{itemize}
		\end{column}
		\begin{column}{0.5\textwidth}
			\sssubroutine{enc}{K, M}{
				$R \twoheadleftarrow \bits^\lambda$\\
				$S \coloneq F(K, R) \oplus M$\\
				return $R\|S$
			}{1}
		\end{column}
	\end{columns}
\end{frame}

\begin{frame}{\ldots but it can be fragile}
	\begin{columns}[c]
		\begin{column}{0.5\textwidth}
			\begin{itemize}[<+->]
				\item We switch M and R's places in \lib{}{cpa-real}'s encryption function.
				\item Is the scheme still secure?
				\item \textbf{No!}
			\end{itemize}
		\end{column}
		\begin{column}{0.5\textwidth}
			\sslinked{
				\sslibrary{}{cpa-real}{
					$K \twoheadleftarrow \bits^\lambda$\\[1em]
					\sslibrarysubroutine{cpa.enc}{M}{
						$R \twoheadleftarrow \bits^\lambda$\\
						$S \coloneq F(K, \hl{M}) \oplus \hl{R}$\\
						return $R\|S$
					}{1}
				}{1}
			}{\overset{?}{\approxeq}}{
				\sslibrary{}{cpa-rand}{
					\sslibrarysubroutine{cpa.enc}{M}{
						$R\|S \twoheadleftarrow \bits^{2\lambda}$\\
						return $R\|S$
					}{1}
				}{1}
			}
		\end{column}
	\end{columns}
\end{frame}

\begin{frame}{The Golden Rule of PRFs}
	\definitionbox{The Golden Rule of PRFs}{
		If a PRF $F$ is being used as a component in a larger construction $H$, then security usually rests on how well $H$ can ensure distinct inputs to $F$.
	}
	\begin{itemize}[<+->]
		\item When analyzing PRF security, focus on input uniqueness.
		\item Repeated inputs to a PRF create exploitable patterns.
		\item Even if $F$ is secure, $H$ can be broken if it causes $F$ to receive duplicate inputs.
		\item Don't try to directly distinguish $F$'s outputs from uniform.
		\item Instead, exploit how $H$ uses $F$ incorrectly.
		\item Find input patterns that force collisions within $F$.
	\end{itemize}
\end{frame}

\begin{frame}{\ldots but it can be fragile}
	\begin{columns}[c]
		\begin{column}{0.5\textwidth}
			\begin{itemize}
				\item $F(K, M)$ is now constant.
				\item \prob{\prog{}\link\lib{}{cpa-real} \Rightarrow \texttt{true}} $ = 1$
			\end{itemize}
		\end{column}
		\begin{column}{0.5\textwidth}
			\sslinked{
				\ssprogram{}{
					$M \twoheadleftarrow \mathcal{M}$\\
					$R_1\|S_1 \coloneq \func{cpa.enc}{M}$\\
					$R_2\|S_2 \coloneq \func{cpa.enc}{M}$\\
					return $S_1 \oplus S_2 == R_1 \oplus R_2$
				}{0.8}
			}{\link}{
				\sslibrary{}{cpa-real}{
					$K \twoheadleftarrow \bits^\lambda$\\[1em]
					\sslibrarysubroutine{cpa.enc}{M}{
						$R \twoheadleftarrow \bits^\lambda$\\
						$S \coloneq F(K, \hl{M}) \oplus \hl{R}$\\
						return $R\|S$
					}{1}
				}{0.8}
			}
		\end{column}
	\end{columns}
\end{frame}

\begin{frame}{AES is a block cipher}{Reminder}
	\begin{itemize}[<+->]
		\item AES takes a 16-byte input, produces a 16-byte output.
		\item Key can be 16, 24 or 32 bytes. \pause
		\item OK, so what if we want to encrypt more than 16 bytes? \pause
		\item \textbf{Proposal}: split the plaintext into 16 byte chunks, encrypt
		      each of them with the same key.
	\end{itemize}
\end{frame}

\begin{frame}{Block cipher examples}
	\begin{columns}
		\begin{column}{0.33\textwidth}
			\imagewithcaption{tux_plaintext.png}{What we start with}
		\end{column}
		\pause
		\begin{column}{0.33\textwidth}
			\imagewithcaption{tux_encrypted_ecb.png}{What we get}
		\end{column}
		\pause
		\begin{column}{0.33\textwidth}
			\imagewithcaption{tux_encrypted_ctr.png}{What we actually want}
		\end{column}
	\end{columns}
\end{frame}

\begin{frame}{AES by itself is deterministic}
	\begin{columns}[c]
		\begin{column}{0.7\textwidth}
			\begin{itemize}[<+->]
				\item AES as a block cipher is inherently \textbf{deterministic}:
				      \begin{itemize}
					      \item $\texttt{AES}(K, M) = C$ will always produce the same $C$.
					      \item This makes raw AES fail CPA security.
				      \end{itemize}
				\item What makes AES secure in practice?
				      \begin{itemize}
					      \item \textbf{Block cipher modes} (CBC, CTR, GCM) add randomness/state using ``initialization vectors''
					            \begin{itemize}
						            \item IV is fancy name for ``random bytes used once''.\footnote{Also referred to as ``nonces'' for ``number used once'', although in some block cipher modes such as AES-CBC, they can technically be used multiple times without problems.}
					            \end{itemize}
					      \item IVs ensure the same plaintext encrypts differently each time.
				      \end{itemize}
				\item Without a proper mode, AES is like ECB mode: predictable patterns
			\end{itemize}
		\end{column}
		\begin{column}{0.3\textwidth}
			\begin{center}
				\ssprogram{AES}{
					$K \twoheadleftarrow \bits^{\lambda}$\\
					$M \twoheadleftarrow \mathcal{M}$\\[0.5em]
					$C_1 \coloneq \texttt{AES}(K, M)$\\
					$C_2 \coloneq \texttt{AES}(K, M)$\\[0.5em]
					$C_1 == C_2$
				}{0.85}
			\end{center}
		\end{column}
	\end{columns}
\end{frame}

\begin{frame}{Block cipher modes of operation}
	\bigimagewithcaption{block_cipher_modes.png}{Source: Wikipedia}
\end{frame}

\begin{frame}{CBC mode: a closer look}
	\begin{columns}[c]
		\begin{column}{0.6\textwidth}
			\begin{itemize}[<+->]
				\item CBC (Cipher Block Chaining) uses previous ciphertext to randomize current block.
				\item Requires a random initialization vector (IV).
				\item Each block's encryption depends on all previous blocks.
				\item Changes to one block affect all subsequent ciphertext blocks.
				\item Sequential encryption (can't parallelize).
			\end{itemize}
		\end{column}
		\begin{column}{0.4\textwidth}
			\sssubroutine{Enc}{K, M_1\|\cdots\|M_\ell}{
				$C_0 \twoheadleftarrow \bits^\lambda$\\
				$\text{for } i=1 \text{ to } \ell:$\\
				$\quad C_i \coloneq F(K, C_{i-1} \oplus M_i)$\\
				$\textbf{return } C_0\|C_1\|\cdots\|C_\ell$
			}{1}
		\end{column}
	\end{columns}
\end{frame}

\begin{frame}{CBC mode: a closer look}
	\vspace{-0.3cm}
	\bigimagewithcaption{cbc_with_dec.pdf}{Source: The Joy of Cryptography}
\end{frame}

\begin{frame}{CTR mode: a closer look}
	\begin{columns}[c]
		\begin{column}{0.6\textwidth}
			\begin{itemize}[<+->]
				\item CTR (Counter) mode turns a block cipher into a stream cipher.
				\item Uses a nonce ($C_0$) plus a counter to create unique inputs to $F$.
				\item Each block's encryption is independent of other blocks.
				\item Highly parallelizable (unlike CBC).
				\item The nonce must never be reused with the same key.
				\item Widely used in modern protocols (TLS, SSH, etc.)
			\end{itemize}
		\end{column}
		\begin{column}{0.4\textwidth}
			\sssubroutine{Enc}{K, M_1\|\cdots\|M_\ell}{
				$C_0 \twoheadleftarrow \bits^\lambda$\\
				$\text{for } i=1 \text{ to } \ell:$\\
				$\quad C_i \coloneq F(K, C_0 + i - 1) \oplus M_i$\\
				$\textbf{return } C_0\|C_1\|\cdots\|C_\ell$
			}{1}
		\end{column}
	\end{columns}
\end{frame}

\begin{frame}{CTR mode: a closer look}
	\bigimagewithcaption{ctr_with_dec.pdf}{Source: The Joy of Cryptography}
\end{frame}

\section{Chosen-Ciphertext Attacks}

\begin{frame}{Introducing chosen-ciphertext attacks}
	\begin{itemize}[<+->]
		\item Imagine this scenario at your company:
		      \begin{itemize}[<+->]
			      \item You discover a bug: software crashes when decrypting ciphertexts that yield plaintexts with null characters.
			      \item You think: ``We never encrypt messages with null bytes, so no problem!''
		      \end{itemize}
		\item An attacker can:
		      \begin{itemize}[<+->]
			      \item Send specially crafted ciphertexts to your system
			      \item Observe which ones crash your software (contain null bytes) and which don't
			      \item Use this information to completely break your encryption scheme!
		      \end{itemize}
	\end{itemize}
\end{frame}

\begin{frame}{Introducing chosen-ciphertext attacks}
	\begin{itemize}
		\item This is a \textbf{chosen-ciphertext attack} (CCA):
		      \begin{itemize}[<+->]
			      \item Attacker can submit arbitrary ciphertexts for decryption.
			      \item System leaks information about the decryption results (the crash).
			      \item Even small information leaks can completely compromise security.
			      \item All encryption schemes we've seen so far are vulnerable to this!
		      \end{itemize}
	\end{itemize}
\end{frame}

\begin{frame}{CPA vs CCA security definitions}
	\begin{columns}[c]
		\begin{column}{0.5\textwidth}
			\definitionbox{CPA Security (informal)}{
				\textit{``Even if I let you encrypt any message you want, you can't obtain any distinguisher regarding my scheme.''}
			}
		\end{column}
		\begin{column}{0.5\textwidth}
			\definitionbox{CCA Security (informal)}{
				\textit{``Even if I let you encrypt any message you want AND decrypt any ciphertext you want, you can't obtain any distinguisher regarding my scheme.''}
			}
		\end{column}
	\end{columns}
\end{frame}

\begin{frame}{CPA vs CCA security definitions}{To be more precise\ldots}
	\begin{columns}[c]
		\begin{column}{0.5\textwidth}
			\definitionbox{CPA Security (more precise)}{
				\textit{``Even if I let you choose any messages to encrypt, you can't tell which of two chosen messages my challenge ciphertext encrypts, except with probability close to $\frac{1}{2}$.''}
			}
		\end{column}
		\begin{column}{0.5\textwidth}
			\definitionbox{CCA Security (more precise)}{
				\textit{``Even if I let you choose any messages to encrypt and decrypt (except the challenge ciphertext itself), you can't tell which of two chosen messages my challenge ciphertext encrypts, except with probability close to $\frac{1}{2}$.''}
			}
		\end{column}
	\end{columns}
\end{frame}

\begin{frame}{Format-oracle attacks \& malleability}
	\begin{columns}[c]
		\begin{column}{0.6\textwidth}
			\begin{itemize}[<+->]
				\item Scenario: victim using CTR mode decryption reveals if result contains a null character.
				\item Information can leak through crashes, error messages, or behavior differences.
				\item A single yes/no bit of information can be enough to break encryption!
				\item Malleability: many encryption schemes allow attackers to make predictable changes to plaintexts by modifying ciphertexts.
				\item CTR mode is especially vulnerable because XOR allows targeted bit-flipping.
			\end{itemize}
		\end{column}
		\begin{column}{0.4\textwidth}
			\sssubroutine{nulloracle}{C}{
				$M \coloneq \texttt{CTR.Dec}(K, C)$\\
				if $M$ contains $\bit{0}\bit{0}$ character:\\
				$\quad$ return $\texttt{true}$\\
				else return $\texttt{false}$
			}{1}
		\end{column}
	\end{columns}
\end{frame}

\begin{frame}{Why is it called \func{nulloracle}{c}?}
	\begin{columns}[c]
		\begin{column}{0.5\textwidth}
			\begin{itemize}
				\item In mythology and colloquial language:
				      \begin{itemize}
					      \item An oracle was a mystic who answered questions on behalf of the gods.
					      \item Often gave enigmatic answers requiring interpretation.
				      \end{itemize}
			\end{itemize}
		\end{column}
		\begin{column}{0.5\textwidth}
			\vspace{-0.4cm}
			\bigimagewithcaption{oracle_painting_zoom.jpg}{\textit{The Oracle} by Camillo Miola, 1880\footnote{Absolutely beautiful painting, well-worth viewing in high resolution: \url{https://upload.wikimedia.org/wikipedia/commons/9/94/Camillo_Miola_\%28Biacca\%29_-_The_Oracle_-_72.PA.32_-_J._Paul_Getty_Museum.jpg}}}
		\end{column}
	\end{columns}
\end{frame}

\begin{frame}{Why is it called \func{nulloracle}{c}?}
	\begin{itemize}[<+->]
		\item In cryptography and computer science:
		      \begin{itemize}
			      \item An oracle is an algorithm that solves a specific problem.
			      \item You can query it but can't see its internal workings.
			      \item Oracles are often used as theoretical tools in security proofs.
		      \end{itemize}
		\item Our \func{nulloracle}{c} reveals only:
		      \begin{itemize}
			      \item Whether a decrypted ciphertext contains null (\bit{0}\bit{0}) characters.
			      \item Just a single bit of information (yes/no).
		      \end{itemize}
	\end{itemize}
\end{frame}

\begin{frame}{Null-oracle attack}
	\begin{columns}[c]
		\begin{column}{0.6\textwidth}
			\begin{itemize}[<+->]
				\item An adversary who has access to \func{nulloracle}{C} can efficiently compute \texttt{Dec}$(K, C)$ for any $C$!
				\item Yes, it really is enough.
			\end{itemize}
		\end{column}
		\begin{column}{0.4\textwidth}
			\sssubroutine{nulloracle}{C}{
				$M \coloneq \texttt{CTR.Dec}(K, C)$\\
				if $M$ contains $\bit{0}\bit{0}$ character:\\
				$\quad$ return $\texttt{true}$\\
				else return $\texttt{false}$
			}{1}
		\end{column}
	\end{columns}
\end{frame}

\begin{frame}{CTR mode is malleable}{Defining malleability}
	\begin{columns}[c]
		\begin{column}{0.5\textwidth}
			\definitionbox{Malleability}{
				An encryption scheme is \textbf{malleable} if, given an encryption $C$ of an unknown plaintext $M$, it is possible to create a new ciphertext $C' \neq C$ where $M' = \texttt{Dec}(K, C')$ is somehow related to $M$, so that $M'$ reveals some information about $M$.
			}
		\end{column}
		\begin{column}{0.5\textwidth}
			\begin{itemize}[<+->]
				\item In CTR mode: $C_i = F(K, C_0 + i - 1) \oplus M_i$
				\item Flipping a bit in $C_i$ flips exactly the same bit in $M_i$.
				\item Attackers can make targeted modifications without knowing the key.
				\item Example: change ``transfer \$100'' to ``transfer \$900'' by modifying just one byte.
			\end{itemize}
		\end{column}
	\end{columns}
\end{frame}

\begin{frame}{CTR mode is malleable}{Defining malleability}
	\begin{columns}[c]
		\begin{column}{0.5\textwidth}
			\begin{itemize}
				\item In CTR mode: $C_i = F(K, C_0 + i - 1) \oplus M_i$
				\item Flipping a bit in $C_i$ flips exactly the same bit in $M_i$.
				\item Attackers can make targeted modifications without knowing the key.
				\item Example: change ``transfer \$100'' to ``transfer \$900'' by modifying just one byte.
			\end{itemize}
		\end{column}
		\begin{column}{0.5\textwidth}
			\imagewithcaption{ctr_dec.pdf}{CTR mode decryption.\\Source: The Joy of Cryptography}
		\end{column}
	\end{columns}
\end{frame}

\begin{frame}{CTR mode is malleable}{Defining malleability}
	\begin{columns}[c]
		\begin{column}{0.5\textwidth}
			\begin{itemize}
				\item In CTR mode: $C_i = F(K, C_0 + i - 1) \oplus M_i$
				\item Flipping a bit in $C_i$ flips exactly the same bit in $M_i$.
				\item Attackers can make targeted modifications without knowing the key.
				\item Example: change ``transfer \$100'' to ``transfer \$900'' by modifying just one byte.
			\end{itemize}
		\end{column}
		\begin{column}{0.5\textwidth}
			\imagewithcaption{ctr_dec_delta.pdf}{CTR mode decryption.\\Source: The Joy of Cryptography}
		\end{column}
	\end{columns}
\end{frame}

\begin{frame}{CTR mode is malleable}{The attack}
	\begin{columns}[c]
		\begin{column}{0.5\textwidth}
			\begin{itemize}[<+->]
				\item Suppose adversary captures ciphertext $C$ and wants to find plaintext $M = \texttt{Dec}(K, C)$.
				\item Let's focus on discovering just the last byte of $M$.
				\item If the last byte of $M$ is $b$, then $M = M' \| b$.
				\item What if we modify the ciphertext by XORing the last byte with $b$?
				\item This would turn the last byte of the plaintext into all zeros (null byte)!
			\end{itemize}
		\end{column}
		\begin{column}{0.5\textwidth}
			\begin{align*}
				C \oplus (\ldots \bit{0}\bit{0} \| b) & \text{ decrypts to:}                            \\
				                                      & M \oplus (\ldots \bit{0}\bit{0} \| b)           \\
				                                      & = (M' \| b) \oplus (\ldots \bit{0}\bit{0} \| b) \\
				                                      & = M' \| (b \oplus b)                            \\
				                                      & = M' \| \bit{0}\bit{0}
			\end{align*}
		\end{column}
	\end{columns}
\end{frame}

\begin{frame}{CTR mode is malleable}{Finding one byte}
	\begin{columns}[c]
		\begin{column}{0.5\textwidth}
			\begin{itemize}[<+->]
				\item Problem: We don't know $b$ in advance!
				\item Solution: Try all 255 possible values for $b$.
				\item For each guess $g$, we:
				      \begin{itemize}
					      \item Create $C' = C \oplus (\ldots \bit{0}\bit{0} \| g)$
					      \item Send $C'$ to the \func{nulloracle}{}
					      \item If oracle returns \texttt{true}, then $g = b$!
				      \end{itemize}
				\item When we find the right value, we've discovered the last byte of $M$.
			\end{itemize}
		\end{column}
		\begin{column}{0.5\textwidth}
			\begin{center}
				Try these ciphertexts:\\
				$C \oplus (\ldots \bit{0}\bit{0} \| \bit{0}\bit{1})$\\
				$C \oplus (\ldots \bit{0}\bit{0} \| \bit{0}\bit{2})$\\
				$C \oplus (\ldots \bit{0}\bit{0} \| \bit{0}\bit{3})$\\
				$\vdots$\\
				$C \oplus (\ldots \bit{0}\bit{0} \| \bit{f}\bit{e})$\\
				$C \oplus (\ldots \bit{0}\bit{0} \| \bit{f}\bit{f})$
			\end{center}
		\end{column}
	\end{columns}
\end{frame}

\begin{frame}{CTR mode is malleable}{Finding all bytes}
	\begin{columns}[c]
		\begin{column}{0.6\textwidth}
			\begin{itemize}[<+->]
				\item We can use the same approach for any byte in the plaintext.
				\item To find the $i$-th byte:
				      \begin{itemize}
					      \item Create a string of all zeros
					      \item Set only the $i$-th byte to our guess $g$
					      \item XOR this with the ciphertext
					      \item Query the oracle
				      \end{itemize}
				\item By repeating for all bytes, we can recover the entire plaintext!
			\end{itemize}
		\end{column}
		\begin{column}{0.4\textwidth}
			\ssprogram{null-oracle}{
			$n := |C_1\|\cdots\|C_{\ell}|$\\
			$M := ""$\\
			\\
			for $i = 1$ to $n$:\\
			$\quad$ for each $b \in \{\bit{0}\bit{1}, \ldots, \bit{f}\bit{f}\}$:\\
			$\quad\quad$ $\Delta := (\bit{0}\bit{0})^n$\\
			$\quad\quad$ $\Delta[i] := b$\\
			$\quad\quad$ if $\func{nulloracle}{C \oplus (0^\lambda \| \Delta)}$:\\
			$\quad\quad\quad$ $M := M \| b$\\
			$\quad\quad\quad$ next $i$\\
			return $M$
			}{0.8}
		\end{column}
	\end{columns}
\end{frame}

\begin{frame}{CTR mode is malleable}{The power of chosen-ciphertext attacks}
	\begin{itemize}[<+->]
		\item With just 255 queries per byte, we can completely decrypt any ciphertext!
		\item For context: decrypting a 1KB file would take about 250,000 queries.
		\item This is extremely practical for an attacker.
		\item All from a single bit of information about the plaintext (contains null or not).
		\item This attack works because:
		      \begin{itemize}
			      \item CTR mode is malleable (we can make predictable changes to plaintext)
			      \item The system leaks a tiny bit of information about decrypted plaintexts
			      \item Together, these flaws completely break the encryption
		      \end{itemize}
	\end{itemize}
\end{frame}

\begin{frame}{Is the null-oracle attack just brute-force?}
	\begin{itemize}[<+->]
		\item Yes and no!
		\item It's brute-force on each byte independently:
		      \begin{itemize}
			      \item To recover $n$-byte plaintext: at most $255n$ oracle queries
			      \item True brute-force on entire plaintext: $255^n$ (exponentially worse!)
		      \end{itemize}
		\item For a 16-byte message:
		      \begin{itemize}
			      \item Null-oracle attack: ~\approx,080 queries (16 \times 255)
			      \item True brute-force: \approx$10^{38}$ queries ($255^{16}$)
		      \end{itemize}
		\item This attack is exponentially more efficient than traditional brute-force.
	\end{itemize}
\end{frame}

\begin{frame}{Can we just rate-limit the number of queries?}
	\begin{columns}[c]
		\begin{column}{0.6\textwidth}
			\begin{itemize}[<+->]
				\item Rate-limiting might help, but:
				      \begin{itemize}
					      \item It only increases attack time, doesn't prevent it.
					      \item Attackers can be patient or use multiple accounts.
					      \item Legitimate users suffer from the rate-limiting.
				      \end{itemize}
				\item Better approach: cryptographic solutions!
				      \begin{itemize}
					      \item Fix the fundamental vulnerability, not just limit its exploitation.
					      \item Create systems that are mathematically proven to resist chosen-ciphertext attacks.
				      \end{itemize}
			\end{itemize}
		\end{column}
		\begin{column}{0.4\textwidth}
			\imagewithcaption{yoda_meme.jpg}{}
		\end{column}
	\end{columns}
\end{frame}

\begin{frame}{Is fixing the null-byte bug enough?}
	\begin{itemize}[<+->]
		\item The null-character bug is just one example of a broader class of attacks.
		\item Format-oracle attacks can exist without implementation bugs!
		\item They only need:
		      \begin{itemize}[<+->]
			      \item A system that accepts untrusted ciphertexts
			      \item Decrypts them
			      \item Behaves differently based on the decryption result
			      \item In a way the attacker can observe
		      \end{itemize}
		\item This behavior can come from:
		      \begin{itemize}[<+->]
			      \item Accidental bugs (null-byte crashes)
			      \item Intentional features (error messages, timing differences)
			      \item Normal application logic (web app behaving differently based on decrypted data)
		      \end{itemize}
	\end{itemize}
\end{frame}

\begin{frame}{Real-world format-oracle attacks}{Padding oracle}
	\begin{columns}[c]
		\begin{column}{1\textwidth}
			\begin{itemize}[<+->]
				\item CBC mode requires padding to handle plaintexts that aren't block-aligned.
				\item Many implementations crash when encountering invalid padding.
				\item This exposes an oracle that tells attackers: ``Does $\texttt{Dec}(K, C)$ have valid padding?''
				\item Attackers can systematically exploit this to decrypt arbitrary ciphertexts.
				\item Has led to major vulnerabilities in SSH and SSL/TLS protocols.
				\item Example: POODLE attack against SSL 3.0 affected millions of websites.\footnote{\url{https://appliedcryptography.page/paper/\#google-poodle}}
			\end{itemize}
		\end{column}
	\end{columns}
\end{frame}

\begin{frame}{Real-world format-oracle attacks}{Timing side-channel}
	\begin{columns}[c]
		\begin{column}{1\textwidth}
			\begin{itemize}[<+->]
				\item Victim interprets plaintext as a number $n$ and performs $n$ operations.
				\item Attackers can measure how long the system takes to respond.
				\item Response time reveals approximate numerical values inside $\texttt{Dec}(K, C)$.
				\item Extremely subtle - even microsecond differences can leak information.
				\item Successfully used to break older SSH and SSL/TLS implementations.
				\item Example: Lucky Thirteen attack against TLS revealed message contents through timing differences.\footnote{\url{https://appliedcryptography.page/paper/\#lucky-thirteen}}
			\end{itemize}
		\end{column}
	\end{columns}
\end{frame}

\begin{frame}{Real-world format-oracle attack}{Apple iMessage}
	\begin{columns}[c]
		\begin{column}{1\textwidth}
			\begin{itemize}[<+->]
				\item Older Apple iMessage versions used gzip compression.
				\item System responded differently when a ciphertext decrypted to:
				      \begin{itemize}
					      \item A valid gzip file (processed normally)
					      \item An invalid gzip file (error reported)
				      \end{itemize}
				\item This created an oracle revealing: ``Is $\texttt{Dec}(K, C)$ a valid gzip file?''\footnote{\url{https://appliedcryptography.page/paper/\#jhu-imessage}}
				\item Attackers who understood the gzip format could exploit this to:
				      \begin{itemize}
					      \item Silently recover private messages
					      \item Bypass encryption entirely
				      \end{itemize}
			\end{itemize}
		\end{column}
	\end{columns}
\end{frame}

\begin{frame}{Real-world format-oracle attack}{XML format Oracles}
	\begin{itemize}[<+->]
		\item Many systems decrypt data expecting valid XML format.
		\item If decrypted data isn't valid XML, system returns an error.
		\item This exposes an oracle: ``Is $\texttt{Dec}(K, C)$ valid XML?''
		\item XML has complex syntax rules that attackers can exploit.
		\item Can allow complete decryption of arbitrary ciphertexts.
		\item Similar attacks exist for other formats (JSON, HTML, etc.)
	\end{itemize}
\end{frame}

\begin{frame}{I thought CTR mode was secure?}{CPA vs. CCA security}
	\begin{itemize}[<+->]
		\item CTR mode \textit{is} secure against chosen-plaintext attacks (CPA-secure).
		      \begin{itemize}
			      \item It uses randomness to ensure identical messages encrypt differently each time.
			      \item The adversary cannot distinguish encryptions of known plaintexts.
		      \end{itemize}
		\item But CPA security isn't enough in many real-world scenarios!
		      \begin{itemize}
			      \item CPA security only considers attackers who can request encryptions.
			      \item It doesn't protect against attackers who can submit chosen ciphertexts.
		      \end{itemize}
		\item In the null-oracle attack:
		      \begin{itemize}
			      \item The victim decrypts ciphertexts chosen by the adversary.
			      \item Even leaking one bit about the plaintext (contains nulls or not) is fatal.
			      \item CPA security doesn't model or prevent this type of attack.
		      \end{itemize}
		\item This is why we need stronger security notions (CCA security).
	\end{itemize}
\end{frame}

\begin{frame}{CCA Security}
	\begin{columns}[c]
		\begin{column}{0.5\textwidth}
			\definitionbox{Security Against Chosen-Ciphertext Attacks (CCA Security)}{
				An encryption scheme $\Sigma$ has security against \textbf{chosen-ciphertext attacks} if the following two libraries are indistinguishable:
			}
		\end{column}
		\begin{column}{0.5\textwidth}
			\sslinked{
				\sslibrary{\Sigma}{cca-real}{
					$K \twoheadleftarrow \bits^\lambda$\\[1em]
					\sslibrarysubroutine{cca.enc}{M}{
						$C \coloneq \Sigma.\texttt{Enc}(K, M)$\\
						return $C$
					}{1}\\[1em]
					\sslibrarysubroutine{cca.dec}{C}{
						return $\Sigma.\texttt{Dec}(K, C)$
					}{1}
				}{0.7}
			}{\approxeq}{
				\sslibrary{\Sigma}{cca-rand}{
					$K \twoheadleftarrow \bits^\lambda$\\[1em]
					\sslibrarysubroutine{cca.enc}{M}{
						$C \twoheadleftarrow \Sigma.\mathcal{C}(|M|)$\\
						$\mathcal{D}[C] \coloneq M$\\
						return $C$
					}{1}\\[1em]
					\sslibrarysubroutine{cca.dec}{C}{
						if $\mathcal{D}[C]$ defined: return $\mathcal{D}[C]$\\
						return $\Sigma.\texttt{Dec}(K, C)$
					}{1}
				}{0.7}
			}
		\end{column}
	\end{columns}
\end{frame}

\begin{frame}{Remember our CPA-secure encryption scheme?}
	\begin{columns}[c]
		\begin{column}{0.5\textwidth}
			\begin{itemize}
				\item Not CCA-secure!
			\end{itemize}
		\end{column}
		\begin{column}{0.5\textwidth}
			\sssubroutine{enc}{K, M}{
				$R \twoheadleftarrow \bits^\lambda$\\
				$S \coloneq F(K, R) \oplus M$\\
				return $R\|S$
			}{2}
		\end{column}
	\end{columns}
\end{frame}

\begin{frame}{Remember our CPA-secure encryption scheme?}
	\begin{columns}[c]
		\begin{column}{0.3\textwidth}
			\begin{itemize}
				\item Not CCA-secure!
				\item In other words, we can trivially distinguish between these libraries:
			\end{itemize}
		\end{column}
		\begin{column}{0.7\textwidth}
			\sslinked{
				\sslibrary{}{cca-real}{
					$K \twoheadleftarrow \bits^\lambda$\\[1em]
					\sslibrarysubroutine{cca.enc}{M}{
						$R \twoheadleftarrow \bits^\lambda$\\
						$S \coloneq F(K, R) \oplus M$\\
						return $R\|S$
					}{1}\\[1em]
					\sslibrarysubroutine{cca.dec}{R\|S}{
						$M \coloneq F(K, R) \oplus S$\\
						return $M$
					}{1}
				}{0.9}
			}{\cancel{\approxeq}}{
				\sslibrary{}{cca-rand}{
					$K \twoheadleftarrow \bits^\lambda$\\[1em]
					\sslibrarysubroutine{cca.enc}{M}{
						$R\|S \twoheadleftarrow \bits^{2\lambda}$\\
						$\mathcal{D}[R\|S] \coloneq M$\\
						return $R\|S$
					}{1}\\[1em]
					\sslibrarysubroutine{cca.dec}{R\|S}{
						if $\mathcal{D}[R\|S]$ defined: return $\mathcal{D}[R\|S]$\\
						$M \coloneq F(K, R) \oplus S$\\
						return $M$
					}{1}
				}{0.9}
			}
		\end{column}
	\end{columns}
\end{frame}

\begin{frame}{Malleability strategy}
	To break the CCA security of a scheme:
	\begin{enumerate}
		\item Study the decryption algorithm of the scheme. It often helps to draw a schematic diagram.
		\item See whether any changes to a ciphertext make a \textit{predictable} change to the plaintext.
		\item Formalize an attack in which the adversary:
		      \begin{enumerate}
			      \item Requests the encryption of a chosen plaintext,
			      \item Modifies the ciphertext as above,
			      \item Asks for the modified ciphertext to be decrypted.
		      \end{enumerate}
	\end{enumerate}
\end{frame}

\begin{frame}{Malleability once again}
	\begin{columns}[c]
		\begin{column}{0.5\textwidth}
			\begin{itemize}
				\item Not CCA-secure!
				\item Here's a distinguisher:
			\end{itemize}
			\imagewithcaption{prf_enc_malleable}{Source: The Joy of Cryptography}
		\end{column}
		\begin{column}{0.5\textwidth}
			\ssprogram{}{
				$M \twoheadleftarrow \bits^\lambda$\\
				$\Delta := \text{arbitrary, nonzero, $\lambda$-bit string}$\\
				$R\|S \coloneq \func{cca.enc}{M}$\\
				$M' \coloneq \func{cca.dec}{R\|(S \oplus \Delta)}$\\
				return $M' == M \oplus \Delta$
			}{1}
		\end{column}
	\end{columns}
\end{frame}

\begin{frame}{Another example}
	\begin{columns}[c]
		\begin{column}{0.6\textwidth}
			\begin{itemize}
				\item Let's try a harder challenge.
			\end{itemize}
			\imagewithcaption{prf_enc_2}{Source: The Joy of Cryptography}
		\end{column}
		\begin{column}{0.4\textwidth}
			\sslibrary{}{cca-real}{
				$K \twoheadleftarrow \bits^\lambda$\\[1em]
				\sslibrarysubroutine{cca.enc}{M}{
					$R \twoheadleftarrow \bits^\lambda$\\
					$S \coloneq F(K, R) \oplus F(K, M)$\\
					return $R\|S$
				}{1}\\[1em]
				\sslibrarysubroutine{cca.dec}{R\|S}{
					$M \coloneq F^{-1}(K, F(K, R) \oplus S)$\\
					return $M$
				}{1}
			}{0.85}
		\end{column}
	\end{columns}
\end{frame}

\begin{frame}{Another example}
	\begin{columns}[c]
		\begin{column}{0.3\textwidth}
			\begin{itemize}
				\item Let's try a harder challenge.
			\end{itemize}
			\imagewithcaption{prf_enc_2}{Source: The Joy of Cryptography}
		\end{column}
		\begin{column}{0.7\textwidth}
			\sslinked{
				\sslibrary{}{cca-real}{
					$K \twoheadleftarrow \bits^\lambda$\\[1em]
					\sslibrarysubroutine{cca.enc}{M}{
						$R \twoheadleftarrow \bits^\lambda$\\
						$S \coloneq F(K, R) \oplus F(K, M)$\\
						return $R\|S$
					}{1}\\[1em]
					\sslibrarysubroutine{cca.dec}{R\|S}{
						$M \coloneq F^{-1}(K, F(K, R) \oplus S)$\\
						return $M$
					}{1}
				}{0.85}
			}{\cancel{\approxeq}}{
				\sslibrary{}{cca-rand}{
					$K \twoheadleftarrow \bits^\lambda$\\[1em]
					\sslibrarysubroutine{cca.enc}{M}{
						$R\|S \twoheadleftarrow \bits^{2\lambda}$\\
						return $R\|S$
					}{1}\\[1em]
					\sslibrarysubroutine{cca.dec}{R\|S}{
						if $\mathcal{D}[R\|S]$ defined: return $\mathcal{D}[R\|S]$\\
						$M \coloneq F^{-1}(K, F(K, R) \oplus S)$\\
						return $M$
					}{1}
				}{0.85}
			}
		\end{column}
	\end{columns}
\end{frame}

\begin{frame}{Frankenstein strategy}
	Try the following approach in a chosen-ciphertext attack:
	\begin{enumerate}
		\item Request two separate encryptions of chosen plaintexts; it often helps to use the same plaintext.
		\item Mix and match parts of the resulting ciphertexts to obtain two Frankenstein ciphertexts.
		\item Ask for the Frankenstein ciphertexts to be decrypted, and see whether anything interesting happens.
	\end{enumerate}
\end{frame}

\begin{frame}{Another example}
	\begin{columns}[c]
		\begin{column}{0.5\textwidth}
			\begin{itemize}
				\item Not CCA-secure!
				\item Here's a distinguisher:
			\end{itemize}
		\end{column}
		\begin{column}{0.5\textwidth}
			\ssprogram{}{
				$M \twoheadleftarrow \bits^\lambda$\\
				$R_1\|S_1 \coloneq \func{cca.enc}{M}$\\
				$R_2\|S_2 \coloneq \func{cca.enc}{M}$\\
				$M_1^* \coloneq \func{cca.dec}{R_1\|S_2}$\\
				$M_2^* \coloneq \func{cca.dec}{R_2\|S_1}$\\
				return $M_1^* == M_2^*$
			}{1}
		\end{column}
	\end{columns}
\end{frame}

\subsection{Message Authentication Codes}

\begin{frame}{Symmetric primitive example: hash functions}{Reminder}
	\begin{columns}[c]
		\begin{column}{0.55\textwidth}
			\definitionbox{Hash Function Properties}{
				\begin{itemize}\item Takes input of \textbf{any size}
					\item Produces output of \textbf{fixed size}
					\item Is \textbf{deterministic} (same input $\rightarrow$ same output)
					\item Even a \textbf{tiny change} in input creates completely different output
					\item Is \textbf{efficient} to compute\end{itemize}
			}
		\end{column}
		\begin{column}{0.45\textwidth}
			\begin{tcolorbox}
				[colback=black!5!white,colframe=ciphergray] $\mathsf{SHA256}(\texttt{hello}) =$ \\ {\small\texttt{2cf24dba5fb0a30e26e83b2ac5}\\ \texttt{b9e29e1b161e5c1fa7425e7304}\\
				\texttt{3362938b9824}}\\[1em]
				$\mathsf{SHA256}(\texttt{hullo}) =$ \\ {\small\texttt{7835066a1457504217688c8f5d}\\
				\texttt{06909c6591e0ca78c254ccf174}\\ \texttt{50d0d999cab0}}
			\end{tcolorbox}
			\textcolor{cipherprimary}{\textbf{Note:} \small One character change $\rightarrow$
				completely different hash!}
		\end{column}
	\end{columns}
\end{frame}
\begin{frame}{Expected properties of a hash function}{Reminder}
	\begin{columns}[c]
		\begin{column}{0.6\textwidth}
			\begin{itemize}[<+->]
				\item \textbf{Collision resistance}: computationally infeasible to find
				      two different inputs producing the same hash.
				\item \textbf{Preimage resistance}: given the output of a hash function,
				      it is computationally infeasible to reconstruct the original input.
				\item \textbf{Second preimage resistance}: given an input and an output,
				      it's computationally infeasible to find another different input
				      producing the same output.
			\end{itemize}
		\end{column}
		\begin{column}{0.4\textwidth}
			\imagewithcaption{sha2.png}{SHA-2 compression function. Source: Wikipedia}
		\end{column}
	\end{columns}
\end{frame}

\begin{frame}{Hash functions: what are they good for?}{Reminder}
	\begin{itemize}[<+->]
		\item \textbf{Password storage}: Store the hash of the password on the server,
		      not the password itself. Then check candidate passwords against the hash.
		\item \textbf{Data integrity verification}: Hash a file. Later hash it
		      again and compare hashes to check if the file has changed, suffered storage
		      degradation, etc.
		\item \textbf{Proof of work}: Server asks client to hash something a lot of
		      times before they can access some resource. Useful for anti-spam, Bitcoin
		      mining, etc.
	\end{itemize}
\end{frame}

\begin{frame}{Message authentication codes}
	\begin{columns}[c]
		\begin{column}{0.5\textwidth}
			\definitionbox{Message Authentication Code (MAC)}{
				A MAC is a cryptographic function that takes a key $K$ and a message $M$ and produces a tag $T$ that authenticates the message. Only someone with the same key can verify the tag.
			}
		\end{column}
		\begin{column}{0.5\textwidth}
			\begin{itemize}[<+->]
				\item A MAC provides \textbf{integrity} and \textbf{authenticity} for messages.
				\item Unlike hash functions, MACs require a secret key
				\item MACs address the malleability problem we saw with encryption schemes. Without a MAC, attackers could modify ciphertexts.
				\item A secure MAC should be unforgeable, even after seeing MACs for chosen messages.
			\end{itemize}
		\end{column}
	\end{columns}
\end{frame}

\begin{frame}{PRFs as MACs}
	\begin{columns}[c]
		\begin{column}{0.4\textwidth}
			\begin{itemize}[<+->]
				\item A pseudorandom function (PRF) can be used directly as a MAC!
				\item The MAC key is the PRF key $K$.
				\item To authenticate a message $X$:
				      \begin{itemize}
					      \item Compute the tag $T = F(K, X)$
					      \item Send both $X$ and $T$ to the recipient
				      \end{itemize}
			\end{itemize}
		\end{column}
		\begin{column}{0.6\textwidth}
			\begin{flushright}
				\sslinked{
					\sslibrary{}{mac-real}{
						$K \twoheadleftarrow \bits^\lambda$\\[1em]
						\sslibrarysubroutine{mac.guess}{X, Y}{
							return $Y == F(K, X)$
						}{1}\\[1em]
						\sslibrarysubroutine{mac.reveal}{X}{
							return $F(K, X)$
						}{1}
					}{0.75}
				}{\approxeq}{
					\sslibrary{}{mac-ideal}{
						\sslibrarysubroutine{mac.guess}{X, Y}{
							if $L[X]$ undefined: return $\texttt{false}$\\
							return $Y == L[X]$
						}{1}\\[1em]
						\sslibrarysubroutine{mac.reveal}{X}{
							if $L[X]$ undefined: $L[X] \twoheadleftarrow \bits^\lambda$\\
							return $L[X]$
						}{1}
					}{0.75}
				}
			\end{flushright}
		\end{column}
	\end{columns}
\end{frame}

\begin{frame}{From CPA to CCA security}{The Encrypt-then-MAC approach}
	\begin{itemize}[<+->]
		\item CPA security isn't enough in the real world.
		\item We need protection against chosen-ciphertext attacks.
		\item Solution: combine encryption with a MAC!.
		\item \textbf{Encrypt-then-MAC}:
		      \begin{itemize}
			      \item Encrypt the message normally.
			      \item Compute a MAC tag of the \textit{ciphertext}.
			      \item Send both ciphertext and tag.
			      \item Receiver verifies the tag before decrypting.
		      \end{itemize}
		\item This prevents adversaries from creating valid modified ciphertexts.
	\end{itemize}
\end{frame}

\begin{frame}{Encrypt-then-MAC: formal construction}
	\begin{columns}[c]
		\begin{column}{0.45\textwidth}
			\definitionbox{Encrypt-then-MAC Construction}{
				Let $\Sigma$ be an SKE scheme and $F$ be a PRF with output length $\lambda$ whose domain includes $\Sigma.\mathcal{C}$.
				Define a new encryption scheme:
				\begin{align*}
					\mathcal{K}       & = \Sigma.\mathcal{K} \times \bits^\lambda       \\
					\mathcal{M}       & = \Sigma.\mathcal{M}                            \\
					\mathcal{C}(\ell) & = \Sigma.\mathcal{C}(\ell) \times \bits^\lambda
				\end{align*}
			}
		\end{column}
		\begin{column}{0.55\textwidth}
			\sssubroutine{Enc}{(K_e, K_m), M}{
				$C \coloneq \Sigma.\texttt{Enc}(K_e, M)$\\
				$T \coloneq F(K_m, C)$\\
				return $C\|T$
			}{1}
			\sssubroutine{Dec}{(K_e, K_m), C\|T}{
				if $F(K_m, C) \neq T$: return $\texttt{err}$\\
				return $\Sigma.\texttt{Dec}(K_e, C)$
			}{1}
		\end{column}
	\end{columns}
\end{frame}

\begin{frame}{CCA security of Encrypt-then-MAC}
	\begin{columns}[c]
		\begin{column}{0.45\textwidth}
			\definitionbox{CCA Security Claim}{
				If $\Sigma$ is a CPA-secure encryption scheme and $F$ is a secure PRF, then the Encrypt-then-MAC construction is CCA-secure.
			}
			\begin{itemize}[<+->]
				\item Key insight: \textbf{the MAC prevents tampering}
				\item Without the MAC key, adversary can't create valid tags.
				\item \textbf{Decryption oracle only returns plaintexts for ciphertexts with valid tags.}
			\end{itemize}
		\end{column}
		\begin{column}{0.55\textwidth}
			\sslinked{
				\sslibrary{}{cca-real}{
					$K_e \twoheadleftarrow \Sigma.\mathcal{K}$\\
					$K_m \twoheadleftarrow \bits^\lambda$\\[1em]
					\sslibrarysubroutine{cca.enc}{M}{
						$C \coloneq \Sigma.\texttt{Enc}(K_e, M)$\\
						$T \coloneq F(K_m, C)$\\
						return $C\|T$
					}{1}\\[1em]
					\sslibrarysubroutine{cca.dec}{C\|T}{
						if $F(K_m, C) \neq T$: return $\texttt{err}$\\
						return $\Sigma.\texttt{Dec}(K_e, C)$
					}{1}
				}{0.65}
			}{\approxeq}{
				\sslibrary{}{cca-rand}{
					$K_e \twoheadleftarrow \Sigma.\mathcal{K}$\\
					$K_m \twoheadleftarrow \bits^\lambda$\\[1em]
					\sslibrarysubroutine{cca.enc}{M}{
						$C \twoheadleftarrow \Sigma.\mathcal{C}(|M|)$\\
						$T \twoheadleftarrow \bits^\lambda$\\
						$\mathcal{D}[C\|T] \coloneq M$\\
						return $C\|T$
					}{1}\\[1em]
					\sslibrarysubroutine{cca.dec}{C\|T}{
						if $\mathcal{D}[C\|T]$ defined: return $\mathcal{D}[C\|T]$\\
						if $F(K_m, C) \neq T$: return $\texttt{err}$\\
						return $\Sigma.\texttt{Dec}(K_e, C)$
					}{1}
				}{0.65}
			}
		\end{column}
	\end{columns}
\end{frame}

\begin{frame}{Combining MACs and encryption}
	\begin{itemize}[<+->]
		\item Not every way of combining a MAC and CPA-secure encryption achieves CCA security.
		\item There are three common approaches to combining them:
		      \begin{itemize}
			      \item \textbf{Encrypt-then-MAC}: Encrypt message, then MAC the ciphertext.
			      \item \textbf{Encrypt-and-MAC}: Encrypt message and MAC the plaintext separately.
			      \item \textbf{MAC-then-encrypt}: MAC the plaintext, then encrypt both message and tag.
		      \end{itemize}
		\item Only one approach guarantees CCA security when using any CPA-secure encryption.
	\end{itemize}
\end{frame}

\begin{frame}{Encrypt-then-MAC}{CCA-secure}
	\begin{columns}[c]
		\begin{column}{0.5\textwidth}
			\begin{itemize}[<+->]
				\item \textbf{MAC verifies ciphertext integrity} before decryption.
				\item Prevents attackers from submitting modified ciphertexts.
				\item \textbf{Always CCA-secure} if encryption is CPA-secure and MAC is secure.
			\end{itemize}
		\end{column}
		\begin{column}{0.5\textwidth}
			\sssubroutine{Enc}{(K_e, K_m), M}{
				$C \coloneq \Sigma.\texttt{Enc}(K_e, M)$\\
				$T \coloneq F(K_m, C)$\\
				return $C\|T$
			}{1}
			\sssubroutine{Dec}{(K_e, K_m), C\|T}{
				if $F(K_m, C) \neq T$: return $\texttt{err}$\\
				return $\Sigma.\texttt{Dec}(K_e, C)$
			}{1}
		\end{column}
	\end{columns}
\end{frame}

\begin{frame}{Encrypt-and-MAC}{Not even CPA-secure!}
	\begin{columns}[c]
		\begin{column}{0.5\textwidth}
			\begin{itemize}[<+->]
				\item \textbf{MAC is computed on the plaintext}
				\item Same plaintext always produces same tag, leaking equality information.
				\item \textbf{Not even CPA-secure}, let alone CCA-secure.
			\end{itemize}
		\end{column}
		\begin{column}{0.5\textwidth}
			\sssubroutine{Enc}{(K_e, K_m), M}{
				$C \coloneq \Sigma.\texttt{Enc}(K_e, M)$\\
				$T \coloneq F(K_m, M)$\\
				return $C\|T$
			}{1}
			\sssubroutine{Dec}{(K_e, K_m), C\|T}{
				$M \coloneq \Sigma.\texttt{Dec}(K_e, C)$\\
				if $F(K_m, M) \neq T$: return $\texttt{err}$\\
				return $M$
			}{1}
		\end{column}
	\end{columns}
\end{frame}

\begin{frame}{MAC-then-encrypt}{It's complicated}
	\begin{columns}[c]
		\begin{column}{0.5\textwidth}
			\begin{itemize}[<+->]
				\item \textbf{Tag is hidden inside the ciphertext}
				\item Whether this is CCA-secure depends on the specific encryption scheme.
				\item \textbf{Not generally CCA-secure} for all CPA-secure encryption schemes.
			\end{itemize}
		\end{column}
		\begin{column}{0.5\textwidth}
			\sssubroutine{Enc}{(K_e, K_m), M}{
				$T \coloneq F(K_m, M)$\\
				$C \coloneq \Sigma.\texttt{Enc}(K_e, M\|T)$\\
				return $C$
			}{1}
			\sssubroutine{Dec}{(K_e, K_m), C}{
				$M\|T \coloneq \Sigma.\texttt{Dec}(K_e, C)$\\
				if $F(K_m, M) \neq T$: return $\texttt{err}$\\
				return $M$
			}{1}
		\end{column}
	\end{columns}
\end{frame}

\begin{frame}{Encrypt + MAC security comparison}
	\begin{center}
		\begin{tabular}{|l|c|c|}
			\hline
			\textbf{Construction} & \textbf{CPA-secure?} & \textbf{CCA-secure?} \\
			\hline
			Encrypt-then-MAC      & Yes                  & Yes                  \\
			\hline
			Encrypt-and-MAC       & No                   & No                   \\
			\hline
			MAC-then-encrypt      & Yes                  & Maybe                \\
			\hline
		\end{tabular}
	\end{center}
	\vspace{1em}
	\begin{itemize}[<+->]
		\item \textbf{Encrypt-then-MAC} is the safest option.
		\item \textbf{Encrypt-and-MAC} should never be used.
		\item \textbf{MAC-then-encrypt} requires case-by-case analysis.
	\end{itemize}
\end{frame}

\subsection{Authenticated Encryption}

\begin{frame}{Authenticated Encryption: beyond CCA security}
	\begin{itemize}[<+->]
		\item CCA security is stronger than CPA security, but still not the gold standard.
		\item CCA security says: adversary-generated ciphertexts won't reveal useful information.
		\item But CCA security doesn't require that they decrypt to \texttt{err}.
		\item For many applications, we want a stronger guarantee:
		      \begin{itemize}
			      \item Only key-holders can create valid ciphertexts.
			      \item All other ciphertexts should be rejected as invalid.
		      \end{itemize}
		\item This property is called \textbf{Authenticated Encryption (AE)}.
		\item Authenticated encryption provides both confidentiality and authenticity.
	\end{itemize}
\end{frame}

\begin{frame}{Authenticated Encryption: formal definition}
	\begin{columns}[c]
		\begin{column}{0.4\textwidth}
			\definitionbox{Authenticated Encryption}{
				A SKE scheme $\Sigma$ is a secure authenticated encryption (AE) scheme if the following two libraries are indistinguishable:
			}
		\end{column}
		\begin{column}{0.6\textwidth}
			\begin{flushright}
				\sslinked{
					\sslibrary{\Sigma}{ae-real}{
						$K \twoheadleftarrow \Sigma.\mathcal{K}$\\[1em]
						\sslibrarysubroutine{ae.enc}{M}{
							return $\Sigma.\texttt{Enc}(K, M)$
						}{1}\\[1em]
						\sslibrarysubroutine{ae.dec}{C}{
							return $\Sigma.\texttt{Dec}(K, C)$
						}{1}
					}{0.75}
				}{\approxeq}{
					\sslibrary{\Sigma}{ae-rand}{
						\sslibrarysubroutine{ae.enc}{M}{
							$C \twoheadleftarrow \Sigma.\mathcal{C}(|M|)$\\
							$\mathcal{D}[C] \coloneq M$\\
							return $C$
						}{1}\\[1em]
						\sslibrarysubroutine{ae.dec}{C}{
							if $\mathcal{D}[C]$ defined: return $\mathcal{D}[C]$\\
							else: return $\texttt{err}$
						}{1}
					}{0.75}
				}
			\end{flushright}
		\end{column}
	\end{columns}
\end{frame}

\begin{frame}{AE vs. CCA security}
	\begin{columns}[c]
		\begin{column}{0.5\textwidth}
			\begin{itemize}[<+->]
				\item Key difference: \textbf{how we handle adversary-created ciphertexts}.
				\item In \lib{\Sigma}{ae-rand}, any ciphertext not created by the library always decrypts to \texttt{err}.
				\item In \lib{\Sigma}{cca-rand}, such ciphertexts could decrypt to anything (not necessarily \texttt{err}).
				\item So AE requires:
				      \begin{itemize}
					      \item Adversary cannot tell real from random ciphertexts (as in CPA)
					      \item Adversary cannot create new valid ciphertexts (authentication)
				      \end{itemize}
			\end{itemize}
		\end{column}
		\begin{column}{0.5\textwidth}
			\begin{itemize}[<+->]
				\item AE is \textbf{strictly stronger} than CCA security.
				\item Every AE scheme is CCA-secure, but not every CCA-secure scheme is an AE.
				\item Making the distinction explicit helps us design better protocols.
				\item AE is what you should aim for in practice.
			\end{itemize}
		\end{column}
	\end{columns}
\end{frame}

\begin{frame}{How Encrypt-then-MAC achieves AE}
	\begin{columns}[c]
		\begin{column}{0.6\textwidth}
			\begin{itemize}[<+->]
				\item Remember our Encrypt-then-MAC construction:
				      \begin{itemize}
					      \item Encrypt the plaintext: $C \coloneq \Sigma.\texttt{Enc}(K_e, M)$
					      \item MAC the ciphertext: $T \coloneq F(K_m, C)$
					      \item Send both: $C\|T$
				      \end{itemize}
				\item It achieves AE security because:
				      \begin{itemize}
					      \item Without $K_m$, adversary can't forge valid tags.
					      \item Any ciphertext not created by the system will fail MAC verification.
					      \item MAC verification failures lead to \texttt{err}.
				      \end{itemize}
				\item The proof is nearly identical to CCA security proof.
			\end{itemize}
		\end{column}
		\begin{column}{0.4\textwidth}
			\definitionbox{AE Security of Encrypt-then-MAC}{
				Encrypt-then-MAC is a secure AE, if the underlying $\Sigma$ is a CPA-secure SKE and $F$ is a secure PRF.
			}
			\vspace{0.5em}
			\definitionbox{AE implies CCA}{
				If an encryption scheme $\Sigma$ is a secure AE, then it is also CCA-secure.
			}
		\end{column}
	\end{columns}
\end{frame}

\begin{frame}{Authenticated Encryption: in practice}
	\begin{itemize}[<+->]
		\item Modern cryptographic protocols almost always use authenticated encryption.
		\item Common AE implementations:
		      \begin{itemize}[<+->]
			      \item \textbf{AES-GCM} (Galois/Counter Mode): most widely used, combines CTR mode with a MAC.
			      \item \textbf{ChaCha20-Poly1305}: popular alternative to AES-GCM, especially on devices without AES hardware.
			      \item \textbf{AES-CBC + HMAC-SHA256}: older approach, uses Encrypt-then-MAC with AES in CBC mode.
		      \end{itemize}
		\item Important implementation rule: \textbf{verify before decrypt}!
		      \begin{itemize}[<+->]
			      \item Always check the MAC before decrypting.
			      \item Prevents timing side-channels based on decryption behavior.
			      \item Helps protect against padding oracle attacks and similar vulnerabilities.
		      \end{itemize}
	\end{itemize}
\end{frame}

\begin{frame}{AES-GCM: Galois/Counter Mode}
	\begin{columns}[c]
		\begin{column}{0.6\textwidth}
			\begin{itemize}[<+->]
				\item AES-GCM (Galois/Counter Mode) is the most widely used AEAD scheme.
				\item Combines AES in CTR mode (for encryption) with GMAC (for authentication).
				\item Extremely efficient:
				      \begin{itemize}
					      \item Single pass over the data.
					      \item Parallelizable.
					      \item Hardware acceleration widely available.
				      \end{itemize}
				\item Used in: TLS 1.2/1.3, IPsec, SSH, and many other protocols.
			\end{itemize}
		\end{column}
		\begin{column}{0.4\textwidth}
			\begin{center}
				\sssubroutine{AES-GCM.Enc}{K, N, A, M}{
					$H \coloneq \texttt{AES}(K, 0^{128})$\\
					$\text{for } i=0 \text{ to } \lceil|M|/128\rceil-1:$\\
					$\quad C_i \coloneq M_i \oplus \texttt{AES}(K, N\|i)$\\
					$T \coloneq \texttt{GHASH}_H(A, C) \oplus$\\
					$\quad\quad \texttt{AES}(K, N\|0)$\\
					$\textbf{return } C\|T$
				}{0.87}
			\end{center}
		\end{column}
	\end{columns}
\end{frame}

\begin{frame}{AES-GCM: Galois/Counter Mode}
	\begin{columns}[c]
		\begin{column}{0.5\textwidth}
			\begin{itemize}[<+->]
				\item \textbf{Inputs}:
				      \begin{itemize}
					      \item Key $K$ (128, 192, or 256 bits)
					      \item Nonce $N$ (usually 96 bits)
					      \item Associated data $A$ (optional)
					      \item Plaintext $M$
				      \end{itemize}
				\item \textbf{Encryption process}:
				      \begin{itemize}
					      \item AES-CTR for confidentiality.
					      \item GHASH (Galois field multiplication) for authentication.
				      \end{itemize}
				\item Authentication tag $T$ protects both ciphertext and associated data.
			\end{itemize}
		\end{column}
		\begin{column}{0.5\textwidth}
			\imagewithcaption{gcm.pdf}{Source: The Joy of Cryptography}
		\end{column}
	\end{columns}
\end{frame}

\begin{frame}{AES-GCM: Galois/Counter Mode}
	\begin{columns}[c]
		\begin{column}{1\textwidth}
			\begin{itemize}[<+->]
				\item \textbf{Security strengths}:
				      \begin{itemize}
					      \item Provides confidentiality, integrity, and authenticity
					      \item Formally proven secure (assuming AES is secure)
					      \item Fast and widely trusted
				      \end{itemize}
				\item \textbf{Critical implementation requirements}:
				      \begin{itemize}
					      \item \textbf{Never reuse a nonce} with the same key!
					      \item A repeated nonce can lead to complete loss of confidentiality and authentication
					      \item 96-bit nonces are recommended (other sizes are less efficient)
					      \item Authentication tag should be at least 128 bits long
				      \end{itemize}
			\end{itemize}
		\end{column}
	\end{columns}
\end{frame}

\begin{frame}{Key commitment in authenticated encryption}
	\begin{columns}[c]
		\begin{column}{1\textwidth}
			\begin{itemize}[<+->]
				\item \textbf{Key commitment}: a ciphertext should only decrypt to a valid plaintext under the key used to generate it.
				\item Most AEAD schemes (including AES-GCM) don't guarantee this property!\footnote{\url{https://appliedcryptography.page/paper/\#key-commitment}}
				\item Attack scenario:
				      \begin{enumerate}
					      \item Attacker creates special ciphertext $C$.
					      \item When decrypted with key $K_1$: harmless message.
					      \item When decrypted with key $K_2$: malicious content!
					      \item Enables plausible deniability, content smuggling, etc.
				      \end{enumerate}
				\item Practical impact:
				      \begin{itemize}
					      \item Attacker can create ciphertexts that decrypt differently under different keys.
					      \item Enables attacks in multi-recipient contexts.
					      \item Affects real applications (e.g., messaging, encrypted files).
				      \end{itemize}
			\end{itemize}
		\end{column}
	\end{columns}
\end{frame}

\begin{frame}{ChaCha20-Poly1305}
	\begin{columns}[c]
		\begin{column}{0.6\textwidth}
			\begin{itemize}[<+->]
				\item ChaCha20-Poly1305 is a modern AEAD construction:
				      \begin{itemize}
					      \item ChaCha20 stream cipher for encryption.
					      \item Poly1305 MAC for authentication.
				      \end{itemize}
				\item Designed by Daniel J. Bernstein.\footnote{Does the class want to hear about Bernstein vs. United States?}
				\item Key characteristics:
				      \begin{itemize}
					      \item No table lookups (better resistance to timing attacks)
					      \item Excellent performance on devices without AES hardware.
				      \end{itemize}
				\item Widely used in TLS 1.3, Signal, WireGuard\ldots
			\end{itemize}
		\end{column}
		\begin{column}{0.4\textwidth}
			\begin{center}
				\sssubroutine{ChaCha20-Poly1305.Enc}{K, N, A, M}{
				$\text{key}_p \coloneq \texttt{ChaCha20}(K, N, 0)_{0..31}$\\
				$C \coloneq M \oplus \texttt{ChaCha20}(K, N, 1)$\\
				$\text{data} \coloneq \text{pad}(A) \| \text{pad}(C) \|$\\
				$\quad\quad \text{len}(A) \| \text{len}(C)$\\
				$T \coloneq \texttt{Poly1305}_{\text{key}_p}(\text{data})$\\
				$\textbf{return } C\|T$
				}{0.87}
			\end{center}
		\end{column}
	\end{columns}
\end{frame}

\begin{frame}{Comparing AEAD implementations}
	\begin{center}
		\begin{tabular}{|l|c|c|c|}
			\hline
			\textbf{Property}       & \textbf{AES-GCM}  & \textbf{ChaCha20-Poly1305} & \textbf{AES+HMAC} \\
			\hline
			Performance (HW accel.) & Excellent         & Good                       & Good              \\
			\hline
			Performance (no accel.) & Poor              & Excellent                  & Moderate          \\
			\hline
			Security level          & 128-256 bits      & 256 bits                   & 128-256 bits      \\
			\hline
			Side-channel resistance & Moderate          & Excellent                  & Moderate          \\
			\hline
			Parallelizable          & Yes               & Partially                  & No                \\
			\hline
			Nonce sensitivity       & Very high         & High                       & Moderate          \\
			\hline
			Overhead                & Low (single pass) & Low (single pass)          & Higher (two pass) \\
			\hline
			Implem. complexity      & Moderate          & Low                        & High              \\
			\hline
		\end{tabular}
	\end{center}
\end{frame}

\begin{frame}{Replay attacks}{Step 1: Alice sends original message}
	\bigimagewithcaption{aead_scenario_1.pdf}{Source: The Joy of Cryptography}
\end{frame}

\begin{frame}{Replay attacks}{Step 2: Attacker replays message in a different context}
	\bigimagewithcaption{aead_scenario_2.pdf}{Source: The Joy of Cryptography}
\end{frame}

\begin{frame}{Replay attacks}{Authenticated encryption didn't save us}
	\begin{itemize}
		\item Even with authenticated encryption, context matters!
		\item Scenario: Alice sends Bob encrypted commands.
		      \begin{itemize}
			      \item Each ciphertext contains Alice's genuine intent.
			      \item Bob trusts any ciphertext that decrypts successfully.
		      \end{itemize}
		\item Vulnerability: An adversary can replay legitimate ciphertexts.
		      \begin{itemize}
			      \item Alice once sent ``Delete temporary files''.
			      \item Adversary replays it when Alice meant to say ``Display files''.
		      \end{itemize}
	\end{itemize}
\end{frame}

\begin{frame}{Associated data}
	\begin{itemize}[<+->]
		\item Solution 1: Include context in the plaintext
		      \begin{itemize}
			      \item ``\texttt{ACTION: DISPLAY}'' before the actual message.
			      \item Inefficient - increases message size.
			      \item Both parties already know the context.
		      \end{itemize}
		\item Better solution: Associated Data (AD)
		      \begin{itemize}
			      \item Context that sender and receiver already know.
			      \item Used during encryption and decryption.
			      \item Doesn't increase ciphertext size.
		      \end{itemize}
		\item How it works:
		      \begin{itemize}
			      \item $\texttt{Enc}(K, A, M) \rightarrow C$ where $A$ is associated data.
			      \item $\texttt{Dec}(K, A, C) \rightarrow M$ or $\texttt{err}$
		      \end{itemize}
	\end{itemize}
\end{frame}

\begin{frame}{AEAD: formal definition}
	\begin{columns}[c]
		\begin{column}{0.5\textwidth}
			\definitionbox{AEAD}{
				An encryption scheme $\Sigma$ with associated data is a secure authenticated encryption with associated data (AEAD) scheme if the following two libraries are indistinguishable:
			}
		\end{column}
		\begin{column}{0.5\textwidth}
			\begin{flushright}
				\sslinked{
					\sslibrary{\Sigma}{aead-real}{
						$K \twoheadleftarrow \Sigma.\mathcal{K}$\\[1em]
						\sslibrarysubroutine{aead.enc}{A, M}{
							return $\Sigma.\texttt{Enc}(K, A, M)$
						}{1}\\[1em]
						\sslibrarysubroutine{aead.dec}{A, C}{
							return $\Sigma.\texttt{Dec}(K, A, C)$
						}{1}
					}{0.75}
				}{\approxeq}{
					\sslibrary{\Sigma}{aead-rand}{
						\sslibrarysubroutine{aead.enc}{A, M}{
							$C \twoheadleftarrow \Sigma.\mathcal{C}(|M|)$\\
							$\mathcal{D}[A, C] \coloneq M$\\
							return $C$
						}{1}\\[1em]
						\sslibrarysubroutine{aead.dec}{A, C}{
							if $\mathcal{D}[A, C]$ defined:\\
							\quad return $\mathcal{D}[A, C]$\\
							else: return $\texttt{err}$
						}{1}
					}{0.75}
				}
			\end{flushright}
		\end{column}
	\end{columns}
\end{frame}

\begin{frame}{Usefulness of AEADs}
	\begin{itemize}[<+->]
		\item AEAD (Authenticated Encryption with Associated Data) scheme guarantees:
		      \begin{itemize}
			      \item Ciphertexts reveal nothing about plaintexts (confidentiality).
			      \item Only ciphertexts created by the legitimate sender will decrypt without error (authenticity).
			      \item Decryption only succeeds when the correct associated data is used (context binding).
		      \end{itemize}
		\item What to use as associated data?
		      \begin{itemize}
			      \item Protocol information: ``\texttt{DISPLAY}'' vs ``\texttt{DELETE}''.
			      \item Session identifiers or timestamps.
			      \item Previous messages in the conversation.
			      \item Any contextual information both parties already know.
		      \end{itemize}
		\item Use as much associated data as relevant - it's cryptographically ``free''!
	\end{itemize}
\end{frame}

\begin{frame}{Solution: use associated data to provide context}
	\bigimagewithcaption{aead_scenario_4.pdf}{Source: The Joy of Cryptography}
\end{frame}

\begin{frame}{The Fujisaki-Okamoto Transform}{From CPA to CCA in public-key encryption}
	\begin{columns}[c]
		\begin{column}{0.6\textwidth}
			\begin{itemize}[<+->]
				\item What about public-key encryption?
				\item Many public-key schemes (RSA, ElGamal) are only CPA-secure.
				\item The \textbf{Fujisaki-Okamoto (FO) transform} converts CPA-secure PKE to CCA-secure PKE.
				\item Basic idea:
				      \begin{itemize}
					      \item Use randomness derived from the message itself.
					      \item Add redundancy that can be verified during decryption.
					      \item Invalid ciphertexts will fail verification.
				      \end{itemize}
			\end{itemize}
		\end{column}
		\begin{column}{0.4\textwidth}
			\definitionbox{FO Transform (simplified)}{
				Given CPA-secure PKE scheme $\Pi$:
				\begin{itemize}
					\item $\texttt{Enc}_{FO}(pk, M)$:
					      \begin{enumerate}
						      \item $R \coloneq H(M)$
						      \item $C_1 \coloneq \Pi.\texttt{Enc}(pk, M; R)$
						      \item $C_2 \coloneq G(M) \oplus R'$
						      \item Return $(C_1, C_2)$
					      \end{enumerate}
					\item Decryption verifies consistency
				\end{itemize}
			}
		\end{column}
	\end{columns}
\end{frame}

\begin{frame}[plain]
	\titlepage
\end{frame}
\end{document}
