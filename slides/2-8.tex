\documentclass[aspectratio=169, lualatex, handout]{beamer}
\makeatletter\def\input@path{{theme/}}\makeatother\usetheme{cipher}

\title{Applied Cryptography}
\author{Nadim Kobeissi}
\institute{American University of Beirut}
\instituteimage{images/aub_white.png}
\date{\today}
\coversubtitle{CMPS 297AD/396AI\\Fall 2025}
\coverpartname{Part 2: Real-World Cryptography}
\covertopicname{2.8: Zero-Knowledge Proofs}
\coverwebsite{https://appliedcryptography.page}

\begin{document}
\begin{frame}[plain]
	\titlepage
\end{frame}

% zkSNARKs, arithmetization
% Sigma, Fiat-Shamir
% That recent paper about limitations of Fiat-Shamir
% OPAQUE, since it was mentioned 100 times in earlier topics
% SRP? Socialist Millionaire? Or too basic?
% Much more, TBD
%
% Use cases:
% - Zcash
% - Penumbra

\incompleteslideswarning

\begin{frame}{Zero-knowledge proofs: magic, but real}
	\begin{columns}[c]
		\begin{column}{0.5\textwidth}
			\textbf{Imagine this scenario:}
			\begin{itemize}
				\item I claim to know the password to a secret vault
				\item You need to verify I really know it
				\item But I don't want to tell you the password!
			\end{itemize}
			\vspace{0.5em}
			\textbf{Seems impossible?}
			\begin{itemize}
				\item How can I prove knowledge without revealing it?
				\item This is the magic of zero-knowledge proofs
			\end{itemize}
		\end{column}
		\begin{column}{0.5\textwidth}
			\begin{block}{Zero-knowledge property}
				A proof that reveals \textbf{nothing} except the truth of the statement being proven
			\end{block}
			\vspace{0.5em}
			\begin{alertblock}{The paradox}
				Convince you completely while teaching you nothing
			\end{alertblock}
		\end{column}
	\end{columns}
\end{frame}

\begin{frame}{What makes zero-knowledge proofs special?}
	\textbf{Traditional proof:}
	\begin{itemize}
		\item ``Here's my password: \texttt{hunter2}''
		\item You learn the secret itself
		\item You could now impersonate me
	\end{itemize}
	\vspace{1em}
	\textbf{Zero-Knowledge proof:}
	\begin{itemize}
		\item ``I'll convince you I know the password''
		\item You learn \textit{only} that I know it
		\item You gain no ability to prove it yourself
		\item You can't replay my proof to others
	\end{itemize}
\end{frame}

\begin{frame}{The three zero-knowledge properties}
	\begin{block}{1. Completeness}
		If the statement is true, an honest prover can convince an honest verifier
	\end{block}
	\begin{block}{2. Soundness}
		If the statement is false, no cheating prover can convince an honest verifier (except with negligible probability)
	\end{block}
	\begin{block}{3. Zero-Knowledge}
		The verifier learns nothing beyond the validity of the statement
	\end{block}
	\vspace{0.25em}
	\begin{center}
		\textbf{These three properties together create the ``magic''}
	\end{center}
\end{frame}

\begin{frame}{Example: Proving Knowledge of a Private Key}
	\begin{columns}[c]
		\begin{column}{0.5\textwidth}
			\textbf{The Setup:}
			\begin{itemize}
				\item Public key: $A = g^{a}$
				\item I claim to know $a$
				\item But revealing $a$ would compromise security
			\end{itemize}
			\vspace{0.5em}
			\textbf{Zero-Knowledge Solution:}
			\begin{enumerate}
				\item Interactive protocol
				\item Multiple rounds of challenges
				\item Statistically convincing
				\item Reveals nothing about $a$
			\end{enumerate}
		\end{column}
		\begin{column}{0.5\textwidth}
			\begin{alertblock}{Can't prove to others}
				After our interaction:
				\begin{itemize}
					\item You're 100\% convinced I know $a$
					\item You've learned nothing about $a$
					\item You can't prove to others that I know it
				\end{itemize}
			\end{alertblock}
			\vspace{0.5em}
			\begin{exampleblock}{No ``trace'' or ``evidence''}
				The interaction leaves no transferable evidence
			\end{exampleblock}
		\end{column}
	\end{columns}
\end{frame}

\begin{frame}{Schnorr Identification Protocol}
	\vspace{-1.5cm}
	\begin{center}\resizebox{0.8\textwidth}{!}{
			\begin{msc}{}
				\setmscvalues{small}
				\drawframe{none}
				\setlength{\instdist}{8cm}
				\setlength{\instwidth}{2cm}
				\declinst{prover}{}{Prover}
				\declinst{verifier}{}{Verifier}
				\action*{$\begin{array}{l}
							\text{Knows generator}~g         \\
							\text{Knows prime}~n             \\
							\text{Knows secret}~a            \\
							y \twoheadleftarrow \mathbb{Z}_n \\
							Y = g^y
						\end{array}$}{prover}
				\action*{$\begin{array}{l}
							\text{Knows generator}~g                 \\
							\text{Knows prime}~n                     \\
							\text{Knows prover's public key}~A = g^a \\
						\end{array}$}{verifier}
				\nextlevel[7]
				\mess{$Y$}{prover}{verifier}
				\nextlevel[1]
				\action*{$c \twoheadleftarrow \mathbb{Z}_n$}{verifier}
				\nextlevel[2]
				\mess{$c$}{verifier}{prover}
				\nextlevel[1]
				\action*{$r = (y + ca) \bmod n$}{prover}
				\nextlevel[2]
				\mess{$r$}{prover}{verifier}
				\nextlevel[1]
				\action*{$g^r \overset{?}{\equiv} Y \cdot A^c$}{verifier}
				\nextlevel[2]
			\end{msc}
		}\end{center}
\end{frame}

\begin{frame}{Schnorr Identification Protocol}
	\begin{columns}[c]
		\begin{column}{0.6\textwidth}
			\begin{itemize}
				\item \textbf{Interactive proof:} Multiple rounds of communication
				\item \textbf{Commitment phase:} Prover commits to random $Y = g^y$
				\item \textbf{Challenge phase:} Verifier sends random challenge $c$
				\item \textbf{Response phase:} Prover computes $r = y + ca$
				\item \textbf{Verification:} Check if $g^r = Y \cdot A^c$
				\item \textbf{Zero-knowledge property:} $r$ reveals nothing about $a$ due to random masking by $y$
				\item \textbf{Soundness:} Without knowing $a$, prover can't answer random challenges correctly
			\end{itemize}
		\end{column}
		\begin{column}{0.4\textwidth}
			\vspace{-1.5cm}
			\begin{center}\resizebox{1\textwidth}{!}{
					\begin{msc}{}
						\setmscvalues{small}
						\drawframe{none}
						\setlength{\instdist}{8cm}
						\setlength{\instwidth}{2cm}
						\declinst{prover}{}{Prover}
						\declinst{verifier}{}{Verifier}
						\action*{$\begin{array}{l}
									\text{Knows generator}~g         \\
									\text{Knows prime}~n             \\
									\text{Knows secret}~a            \\
									y \twoheadleftarrow \mathbb{Z}_n \\
									Y = g^y
								\end{array}$}{prover}
						\action*{$\begin{array}{l}
									\text{Knows generator}~g                 \\
									\text{Knows prime}~n                     \\
									\text{Knows prover's public key}~A = g^a \\
								\end{array}$}{verifier}
						\nextlevel[7]
						\mess{$Y$}{prover}{verifier}
						\nextlevel[1]
						\action*{$c \twoheadleftarrow \mathbb{Z}_n$}{verifier}
						\nextlevel[2]
						\mess{$c$}{verifier}{prover}
						\nextlevel[1]
						\action*{$r = (y + ca) \bmod n$}{prover}
						\nextlevel[2]
						\mess{$r$}{prover}{verifier}
						\nextlevel[1]
						\action*{$g^r \overset{?}{\equiv} Y \cdot A^c$}{verifier}
						\nextlevel[2]
					\end{msc}
				}\end{center}
		\end{column}
	\end{columns}
\end{frame}

\begin{frame}{The verifier learned literally nothing new}
	\begin{columns}[c]
		\begin{column}{0.55\textwidth}
			\begin{itemize}
				\item After the protocol, you're convinced I know the secret
				\item But you can't convince anyone else!
				\item Why? The transcript looks identical to \textbf{something you could have generated alone}
				\item \textbf{The simulation argument:}
				      \begin{itemize}
					      \item You could have picked $r$ and $c$ randomly
					      \item Then computed $Y = g^r / A^c$
					      \item This produces a valid-looking transcript!
					      \item No way to distinguish from real interaction
				      \end{itemize}
			\end{itemize}
		\end{column}
		\begin{column}{0.45\textwidth}
		\resizebox{0.8\textwidth}{!}{%
			\begin{minipage}{\textwidth}
				\begin{alertblock}{Self-generated transcripts}
					For any challenge $c$ and response $r$:
					\begin{itemize}
						\item Set $Y = g^r / A^c$
						\item Now $(Y, c, r)$ is a valid transcript
						\item Indistinguishable from real protocol
						\item Real interaction: Prover knows secret
						\item Simulated transcript: No secret needed
						\item Both look exactly the same!
						\item This is what makes it ``zero-knowledge''
					\end{itemize}
				\end{alertblock}
			\end{minipage}%
		}
		\end{column}
	\end{columns}
\end{frame}

\begin{frame}{ZK can scale to entire systems}{Reminder}
	\begin{columns}[c]
		\begin{column}{0.6\textwidth}
			\begin{itemize}[<+->]
				\item As seen previously, you can build entire \textbf{zero-knowledge protocols}, not just primitives that prove knowledge of certain numbers.
				\item ``Zero-knowledge virtual machines'' where you can execute
				      an entire program that runs as a zero-knowledge proof.
				\item ZKP battleship game: server proves to the players that its
				      output to their battleship guesses is correct, without revealing any
				      additional information (e.g. ship location).
			\end{itemize}
		\end{column}

		\begin{column}{0.4\textwidth}
			\imagewithcaption{battleship.jpg}{Battleship board game. Source: Hasbro}
		\end{column}
	\end{columns}
\end{frame}

\begin{frame}{Beyond authentication: the power of zero-knowledge}
	\begin{center}
		\textbf{Zero-knowledge is not limited to proving identity!}
	\end{center}
	\vspace{0.25em}
	\begin{columns}[c]
		\begin{column}{0.5\textwidth}
			\begin{itemize}
				\item \textbf{Can prove statements like:}
				      \begin{itemize}
					      \item \textit{``I know a solution to this Sudoku puzzle''}
					      \item \textit{``This encrypted value is in range [0, 100]''} (range proofs)
					      \item \textit{``I performed this computation correctly''}
					      \item \ldots without revealing any information!
				      \end{itemize}
			\end{itemize}
		\end{column}
		\begin{column}{0.5\textwidth}
			\begin{itemize}
				\item \textbf{Applications:}
				      \begin{itemize}
					      \item Private transactions (Zcash)
					      \item Anonymous credentials
					      \item Secure voting systems
					      \item Privacy-preserving audits
				      \end{itemize}
			\end{itemize}
		\end{column}
	\end{columns}
\end{frame}

\incompleteslideswarning

\begin{frame}[plain]
	\titlepage
\end{frame}
\end{document}
